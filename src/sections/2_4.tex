% Copyright 2021 Joel Feldman, Andrew Rechnitzer and Elyse Yeager, except where noted.
% This work is licensed under a Creative Commons Attribution-NonCommercial-ShareAlike 4.0 International License.
% https://creativecommons.org/licenses/by-nc-sa/4.0/


 \begin{frame}{Table of Contents }
\mapofcontentsB{\bd}
 \end{frame}
%----------------------------------------------------------------------------------------

%----------------------------------------------------------------------------------------
\section{Introduction to Differential Equations}
%----------------------------------------------------------------------------------------
\begin{frame}\begin{block}{Differential Equation}
A \alert{differential equation} is an equation for an unknown function that
involves the derivative of the unknown function.
\end{block} \pause
 Differential equations
play a central role in modelling a huge number of different phenomena.
Here is a table giving a bunch of named differential equations and what
they are used for. It is far from complete.

\begin{center}\tiny
\renewcommand{\arraystretch}{1.4}
     \begin{tabular}{|c|c|}
        \hline
  Newton's Law of Motion
      & describes motion of particles \\ \hline
  Maxwell's equations
      &describes electromagnetic radiation \\ \hline
  Navier--Stokes equations
      &describes fluid motion \\ \hline
  Heat equation
      &describes heat flow \\ \hline
  Wave equation
      &describes wave motion \\ \hline
  Schr\"odinger equation
      &describes atoms, molecules and crystals \\ \hline
  Stress-strain equations
      &describes elastic materials \\ \hline
  Black--Scholes models
      &used for pricing financial options \\ \hline
  Predator--prey equations
      &describes ecosystem populations  \\ \hline
  Einstein's equations
      &connects gravity and geometry  \\ \hline
  Ludwig--Jones--Holling's equation
      &models spruce budworm/Balsam fir ecosystem  \\ \hline
  Zeeman's model
      &models heart beats and nerve impulses \\ \hline
  Sherman--Rinzel--Keizer model
      &for electrical activity in Pancreatic $\beta$--cells  \\ \hline
  Hodgkin--Huxley equations
      &models nerve action potentials  \\
  \hline
     \end{tabular}
\renewcommand{\arraystretch}{1.0}
\end{center}
\end{frame}
%----------------------------------------------------------------------------------------

%----------------------------------------------------------------------------------------
%-------------------------------------------------------------
\begin{frame}
\label{note2.4a}
\centering
Disclaimer:

We are dipping our toes into a vast topic. Most universities offer half a dozen different undergraduate
courses on various aspects of differential equations. We will just look
at one special, but important, type of equation.
\vfill

\begin{itemize}[<+->]
\item We will first learn to \alert{verify} solutions without \alert{finding} them. (If you learned about differential equations last semester, this will be review.)\vfill
\item \alert{Then,} we will learn to solve one particular type of differential equation.\vfill

\end{itemize}
\vfill

\end{frame}
%-------------------------------------------------------------
\begin{frame}{Differential Equations}
\begin{defn}
	A \textbf{differential equation} is an equation involving the derivative of an unknown function.
	\end{defn}
\pause \vfill
Examples: $\diff{y}{x}=2x$; \qquad $x\diff{y}{x}=7xy+y$
\pause \vfill
\begin{defn}
	If a \alert{function} makes a differential equation true, we say it \textbf{satisfies} the differential equation, or is a solution to the differential equation.
\end{defn}
\pause\vfill
Example: $y=x^2$ and $y=x^2+1$ both satisfy the first differential equation%;
%\pause\vfill
%the function $y=xe^{7x+9}$ satisfies the second.\vfill
\end{frame}
%-------------------------------------------------------------

\begin{frame}[t]{Verifying Solutions}
\AnswerNo
Consider the equation 
\[x+2=x^3-x^2\]

How would you verify whether $x=1$ satisfies the equation?\\


How would you verify whether $x=2$ satisfies the equation?\\
\pause

\color{W1} Plug $x$ into the equation, check whether the left-hand side and the right-hand side are the same \textbf{number}.
\end{frame}
%-------------------------------------------------------------

\begin{frame}[t]{Verifying Solutions}
\label{note2.4b}
\AnswerYes<1-2>
\NoSpace<2>
Consider the differential equation 
\[\diff{y}{x}=2y+4x\]

\only<-2>{
How would you verify whether $\textcolor{C2}{y=e^{2x}-2x}$ satisfies the equation?\\
How would you verify whether $\textcolor{C2}{y=e^{2x}-2x-1}$ satisfies the equation?\\}


\onslide<2->{\color{W1} Replace $y$ and $\diff{y}{x}$ in the equation, check whether the left-hand side and the right-hand side are the same \textbf{function}. } 

\sonly<3->{
	\begin{itemize}\color{spoilercolor}
	\only<3>{\item If $\textcolor{C2}{y=e^{2x}-2x}$, then $\textcolor{C3}{\diff{y}{x}=2e^{2x}-2}$. Plug these into both sides of the differential equation, replacing anything depending on $y$:
	\begin{align*}
	\textcolor{C3}{\diff{y}{x}}&=2\textcolor{C2}{y}+4x\\
	\textcolor{C3}{2e^{2x}-2}&\stackrel{?}{=} 2\textcolor{C2}{(e^{2x}-2x)}+4x\\
	2e^{2x}-2&\stackrel{?}{=} 2e^{2x}
	\end{align*}
	Since the functions on the left and right are not the same function, $y=e^{2x}-2x$ is \alert{not} a solution to the differential equation.}
	\item<4> If $\textcolor{C2}{y=e^{2x}-2x-1}$, then $\textcolor{C3}{\diff{y}{x}=2e^{2x}-2}$. Plug these into both sides of the differential equation, replacing anything depending on \textcolor{C3}{$y$}:
	\begin{align*}
	\textcolor{C3}{\diff{y}{x}}&=2\textcolor{C2}{y}+4x\\
	\textcolor{C3}{2e^{2x}-2}&\stackrel{?}{=} 2\textcolor{C2}{(e^{2x}-2x-1)}+4x\\
	2e^{2x}-2&\stackrel{?}{=} 2e^{2x}-4x-2+4x
	\end{align*}
	Since the functions on the left and right are the same function, $y=e^{2x}-2x-1$ \alert{is} a solution to the differential equation.
	\end{itemize}}
	
\end{frame}
%-------------------------------------------------------------

\begin{frame}[t]
\AnswerYes<2>\MoreSpace<2>
\begin{multicols}{2}\alert{Differential equation:}
\[x\diff{y}{x}=7xy+y\]
\vfill

\columnbreak

\alert{Interpretation:}\\[1em]
There is a function $y(x)$ that makes the left-hand side and the right-hand side into the same function.\\[1em]
To check whether a given function satisfies the differential equation, plug it in for everything with a ``$y$": $y$ itself and $\diff{y}{x}$.
\end{multicols}

\onslide<2->{Is $y=xe^{7x+9}$  a solution to the differential equation?}

\vfill
\end{frame}
%-------------------------------------------------------------
\begin{frame}<beamer>[t]
\AnswerYes<1>
Differential equation: $x\diff{y}{x}=7xy+y$\\
Function: $y=xe^{7x+9}$

\sonslide<2->{
If $\textcolor{C2}{y=xe^{7x+9}}$, then $\textcolor{C3}{\diff{y}{x}}=x\left(7e^{7x+9}\right)+e^{7x+9}=\textcolor{C3}{(7x+1)e^{7x+9}}$. We replace all terms depending on $\textcolor{C3}{y}$ in the differential equation with these functions.
\begin{align*}
x\textcolor{C3}{\diff{y}{x}}&=7x\textcolor{C2}{y}+\textcolor{C2}y\\
x\textcolor{C3}{(7x+1)e^{7x+9}}&=7x\textcolor{C2}{(xe^{7x+9})}+\textcolor{C2}{xe^{7x+9}}\\
x(7x+1)e^{7x+9}&=x(7x+1)e^{7x+9}
\end{align*}
The left and right hand side of the equation give the same function, so our function $y=xe^{7x+9}$ satisfies the differential equation.}
\end{frame}
%----------------------------------------------------------------------------------------
\begin{frame}[t]
\AnswerYes<1>\NoSpace<1>
Which of the following solve the differential equation $\diff{y}{x}=\frac{x}{y}$ ?\\[10pt]

\salert<2>{A. $y=-x$}\hfill  B. $y=x+5 $ \hfill  \salert<2>{C. $y=\sqrt{x^2+5}$}

\sonslide<2->{
	
	\begin{itemize}\color{spoilercolor}
	\item If $\textcolor{C2}{y=-x}$, then $\textcolor{C3}{\diff{y}{x}=-1}$. Plugging into the differential equation yields:
	$\textcolor{C3}{-1}\stackrel{?}{=}\frac{x}{\textcolor{C2}{-x}}$.
	Since the left and right are the same function (except for the single point when $x=0$), we say $y=-x$ \alert{solves} the differential equation.
	\vfill
	\item If $\textcolor{C2}{y=x+5}$, then $\textcolor{C3}{\diff{y}{x}=1}$. Plugging into the differential equation yields:
	$\textcolor{C3}{1}\stackrel{?}{=}\frac{x}{\textcolor{C2}{x+5}}$.
	Since the left and right are \alert{not} the same function, $y=x+5$ \alert{does not solve} the differential equation.
	\vfill
	\item If $\textcolor{C2}{y=\sqrt{x^2+5}}$, then $\textcolor{C3}{\diff{y}{x}}=\frac{2x}{2\sqrt{x^2+5}}=\textcolor{C3}{\frac{x}{\sqrt{x^2+5}}}$. Plugging into the differential equation yields:
	$
	\textcolor{C3}{\frac{x}{\sqrt{x^2+5}}}\stackrel{?}{=}\frac{x}{\textcolor{C2}{\sqrt{x^2+5}}}
	$.
	Since the left and right are the same function, we say $y=\sqrt{x^2+5}$ \alert{solves} the differential equation.
	\end{itemize}
	}
\end{frame}

%----------------------------------------------------------------------------------------
%----------------------------------------------------------------------------------------
%----------------------------------------------------------------------------------------
\section{2.4 Separable Differential Equations}
%----------------------------------------------------------------------------------------
%----------------------------------------------------------------------------------------
%----------------------------------------------------------------------------------------

%----------------------------------------------------------------------------------------
%----------------------------------------------------------------------------------------
\begin{frame}[t]{First Example of a Separable DE}
\label{note2.4c}
\AnswerYes<2>
\only<1>{\begin{block}{Definition}
A separable differential equation is an equation for a function y(x) that can be written in the form
\[g(y)\cdot\diff{y}{x}=f(x) \]
\end{block}
(It may take some rearranging to get the equation into this form.)\\[1em]
For example:
\MoreSpace}


\begin{align*}
\color{black}y^2\cdot\diff{y}{x}&\color{black}=4x\\
\sonslide<3->{
\int \left(y^2 \cdot \diff{y}{x}\right)\ \dee{x}&=\int 4x \ \dee{x}\\
\int y^2 \ \dee{y} &=2x^2+C\\
\frac{1}{3}y^3 &=2x^2+C\\
y^3&=6x^2+3C\\
y(x)&=\sqrt[3]{6x^2+3C}\\
y(x)&=\sqrt[3]{6x^2+D}
}
\end{align*}
\sonslide<3->{Here $C$ and $D$ are arbitrary constants.}
\end{frame}
%----------------------------------------------------------------------------------------
\begin{frame}[t]{General Method for Solving Separable DEs}
\AnswerYes<1>
\[g(y)\cdot\diff{y}{x}=f(x) \]
\sonly<2>{
\begin{align*}
g(y(x))\cdot \diff{y}{x}&=f(x)\\
\int \left(g(y(x))\cdot \diff{y}{x}\right)\,\dee{x}&=\int f(x)\,\dee{x}\intertext{$y$-substitution:}
\int g(y)\,\dee{y}&=\int f(x)\,\dee{x}
\end{align*}}
\snshonly{3}{2}{2}{\color{C2}Shorthand:
\begin{align*}
g(y)\cdot \diff{y}{x}&=f(x)\\
g(y)\,\dee{y}&=f(x)\,\dee{x}\\
\int g(y)\,\dee{y}&=\int f(x)\,\dee{x}
\end{align*}}

\end{frame}
%----------------------------------------------------------------------------------------
\begin{frame}[t]
\QuestionBar<1>{1}{3}
\AnswerBar<2->{1}{3}
\AnswerYes<-6>
\nsAnswerYes<-6>
\StatusBar{1}{7}
\[\diff{y}{x}=y^2x\]\pause
\begin{enumerate}
\item ``Separate" $y$'s from $x$'s. \pause \\
\only<beamer>{\textcolor{spoilercolor}{$\underbrace{\frac{1}{y^2}\dee{y}}_{\text{only }y}=\underbrace{x\dee{x}}_{\text{only }x}$}\pause}
\item Integrate.\pause
\only<beamer>{\textcolor{spoilercolor}{\begin{align*}
\int\frac{1}{y^2}\dee{y}&=\int x\dee{x} &&\implies&& -\frac1y=\frac12x^2+C
\end{align*}}\pause}
\item Solve explicitly for $y$.\hfill \pause
\only<beamer>{\textcolor{spoilercolor}{$\displaystyle y=\frac{1}{-\frac12x^2-C}$}
\hfill~}
\end{enumerate}
\end{frame}
%----------------------------------------------------------------------------------------
\begin{frame}[t]
\QuestionBar<1>{2}{3}\AnswerYes<1>
\AnswerBar<2>{2}{3}

\[\diff{y}{x}=(xy)^4,\qquad y(0)=\frac12 \]
\sonslide<2>{\footnotesize
\begin{multicols}{2}\allowdisplaybreaks
\begin{align*}
\diff{y}{x}&=x^4y^4\\
y^{-4}\ \dee y & = x^4 \dee x\\
\int y^{-4} \dee y & =\int  x^4 \dee x\\
\frac{1}{-3}y^{-3} & = \frac15 x^5 +C\\
\frac{1}{y^3}&=-3\left(\frac{1}{5}x^5+C\right)\\
y&=\frac{1}{-\sqrt[3]{3\left(\frac{1}{5}x^5+C\right)}}\\
y(0)&=-\left.\sqrt[3]{\frac{1}{3\big(\frac{1}{5}x^5+C\big)}}\right|_{x=0}\\
\frac12&=-\sqrt[3]{\frac{1}{3C}}\\
2&=-\sqrt[3]{3C}\\
3C&=-8\\
y(x)&=-\sqrt[3]{\frac{1}{\frac{3}{5}x^5-8}}\\
&=\sqrt[3]{\frac{1}{8-\frac{3}{5}x^5}}\\
\end{align*}\end{multicols}}
\end{frame}
%---------------------------------------------------
%----------------------------------------------------------------------------------------
\begin{frame}[t]\[\diff{y}{x}=y(4x^3-1)\qquad y(0)=-2\]
\QuestionBar<1>{3}{3}\AnswerYes<1>
\AnswerBar<2>{3}{3}
\sonslide<2>{\begin{align*}
\frac1y\ \dee y&=\left(4x^3-1\right)\dee x\\
\int \frac1y\ \dee y&=\int \left(4x^3-1\right)\dee x\\
\log|y|&=x^4-x+C\\
\text{When $x=0$,\ \ }\log|-2|&=0^4-0+C\\
C&=\log 2\\
|y(x)|&=e^{x^4-x+\log 2}\\
y(x)&=e^{x^4-x+\log 2} \quad\mbox{ or }\quad y(x)=-e^{x^4-x+\log 2}\\
y(x)&=-e^{x^4-x+\log 2}=-2e^{x^4-x}\quad\text{to make $y(0)=-2$}
\end{align*}}

\end{frame}
%----------------------------------------------------------------------------------------
%----------------------------------------------------------------------------------------
%----------------------------------------------------------------------------------------

\begin{frame}[t]
\AnswerYes<1>
Let $a$ and $b$ be any two constants. We'll now solve the family of differential
equations
\begin{equation*}
\diff{y}{x} =a(y-b)
\end{equation*}
using our mnemonic device.
\sonslide<2->{
\begin{align*}
\frac{\dee{y}}{y-b} &=a\,\dee{x}\\
\int \frac{\dee{y}}{y-b}&= \int a\,\dee{x}\\
\log|y-b|&= ax+c\\
|y-b|&= e^{ax+c} =e^c e^{ax} \\
y-b &=\pm e^c e^{ax} = C e^{ax}
\end{align*}
where the constant $C$ can be any real number. (Even $C=0$ works, i.e. $y(x)=b$  solves $\diff{y}{x}=a(y-b)$.) Note that when $y(x)=C e^{ax}+b$ we have $y(0)=C+b$. So $C=y(0)-b$ and the general solution is
\begin{equation*}
y(x) = \{y(0)-b\}\,e^{ax} + b
\end{equation*}}
	\unote{Example~\eref{text}{eg:SDEsdeFrist}}
\end{frame}
%-------------------------------------------------------------%----------------------------------------------------------------------------------------

\begin{frame}[t]
\AnswerYes<2>\NoSpace<2>
\QuestionBar<2>{1}{2}
\AnswerBar<3>{1}{2}
\only<-2>{
\begin{block}{Linear First-Order Differential Equations}
Let $a$ and $b$ be constants.
The differentiable function $y(x$) obeys the differential equation
\begin{equation*}
\diff{y}{x} =a(y-b)
\end{equation*}
if and only if
\begin{equation*}
y(x) = \{y(0)-b\}\,e^{ax} + b
\end{equation*}
\end{block}}

\pause
Find a function $y(x)$ with $y'=3y+7$ and $y(2)=5$.

\sonly<3>{
\begin{multicols}{2}

To avoid re-inventing the wheel, we'll use our equation. But first, we should re-write our differential equation so the formatting matches.

\begin{align*}
\diff{y}{x}&=3\left(y+\frac73\right)\\
a&=3,\quad b=-\frac73\\
y(x)&=Ce^{3x}-\frac73
\intertext{Since we aren't given $y(0)$, we can't use the theorem as a shortcut to find $C$. We'll do it the old-fashioned way.}
5=y(2)&=Ce^{3(2)}-\frac73\\
\frac{22}{3}&=Ce^6\\
C&=\frac{22}{3e^6}\\
y(x)&=\frac{22}{3e^6}e^{3x}-\frac73
\end{align*}
\end{multicols}
}
\unote{Theorem~\eref{text}{thm:linearODE}}

\end{frame}
%----------------------------------------------------------------------------------------
%----------------------------------------------------------------------------------------
\begin{frame}[t]
\QuestionBar<1>{2}{2}\AnswerYes<1>
\AnswerBar{2}{2}<2->
\sMoreSpace<2->
\nsMoreSpace<1>
The rate at which a medicine is metabolized (broken down) in the body depends on how much of it is in the bloodstream. Suppose a certain medicine is metabolized at a rate of $\frac{1}{10}A$ $\mu$g/hr, where $A$ is the amount of medicine in the patient. The medicine is being administered to the patient at a constant rate of 2 $\mu$g/hr.

If the patient starts with no medicine in their blood at $t=0$, give the formula for the amount of medicine in the patient at time $t$. What happens to the amount over time?\vfill

\sonslide<2->{The rate of change of the amount of medicine in the patient is given by how quickly the medicine is being administered, minus how quickly it is metabolized:
\[\diff{A}{t}=2-\frac{1}{10}A\]}
\end{frame}
%----------------------------------------------------------------------------------------
\begin{frame}[t]
\AnswerBar{2}{2}
\AnswerYes<1>\NoSpace<1>
\begin{block}{Linear First-Order Differential Equations}
Let $a$ and $b$ be constants.
The differentiable function $y(x$) obeys the differential equation
\begin{equation*}
\diff{y}{x} =a(y-b)
\end{equation*}
if and only if
\begin{equation*}
y(x) = \{y(0)-b\}\,e^{ax} + b
\end{equation*}
\end{block}

\[\diff{A}{t}=2-\frac{1}{10}A=-\frac{1}{10}(A-20)\qquad A(0)=0\]

\sonslide<2->{
\begin{multicols}{2}
\begin{align*}
a&=-\frac{1}{10},\quad b=20\\
A(t)&=(A(0)-20)e^{-t/10}+20\\
A(t)&=-20e^{-t/10}+20\\
\end{align*}
This is an increasing function, with $\lim\limits_{t \to \infty}A(t)=20$. So the amount of medicine initially increases, but eventually almost holds steady at 20 $\mu$g.
\end{multicols}}


\end{frame}

%----------------------------------------------------------------------------------------
%----------------------------------------------------------------------------------------
%----------------------------------------------------------------------------------------

