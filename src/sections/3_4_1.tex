% Copyright 2021 Joel Feldman, Andrew Rechnitzer and Elyse Yeager, except where noted.
% This work is licensed under a Creative Commons Attribution-NonCommercial-ShareAlike 4.0 International License.
% https://creativecommons.org/licenses/by-nc-sa/4.0/


 \begin{frame}{Table of Contents }
\mapofcontentsC{\cd}
 \end{frame}
%%----------------------------------------------------------------------------------------
%%----------------------------------------------------------------------------------------
%----------------------------------------------------------------------------------------
\section{3.4 Absolute and Conditional Convergence}
%---------------------------------------------------------------------------------
%----------------------------------------------------------------------------------------
\begin{frame}[t]{Four Series}
\only<1>{\AnswerYes}
\AnswerSpace
Let $a_n = \left(-\frac{2}{3}\right)^n$. Do the following series converge or diverge?
\[\sum_{n=0}^\infty a_n \hspace{2cm} \sum_{n=0}^\infty |a_n|\]
\sonslide<2->{\[ \text{converge} \hspace{2cm} \text{converge}\]}
\vfill
Let $b_n = \frac{(-1)^n}{n}$. Do the following series converge or diverge?
\[\sum_{n=1}^\infty b_n \hspace{2cm} \sum_{n=1}^\infty |b_n|\]
\sonslide<2->{\[ \text{converge} \hspace{2cm} \text{diverge}\]}
\end{frame}
%----------------------------------------------------------------------------------------
%----------------------------------------------------------------------------------------
%----------------------------------------------------------------------------------------
\begin{frame}[t]
The series
\[\sum_{n=0}^\infty \left(-\frac23\right)^n\]
is called \alert{absolutely convergent}, because the series converges and if we replace the terms being added by their absolute values, that series \textit{still} converges.

\vfill

The series
\[\sum_{n=0}^\infty\frac{(-1)^n}{n}\]
is called \alert{conditionally convergent}, because the series converges, but  if we replace the terms being added by their absolute values, that series \textit{diverges}.
\end{frame}
%----------------------------------------------------------------------------------------
%----------------------------------------------------------------------------------------
\begin{frame}[t]
\unote{Definition~\eref{text}{def:SRabsCond} and Theorem~\eref{text}{thm:SRabs}}

\begin{block}{Absolute and conditional convergence}
\begin{enumerate}[(a)]
\item A series $\sum\limits_{n=1}^\infty a_n$ is said to \textbf{converge absolutely}
if the series\linebreak $\sum\limits_{n=1}^\infty |a_n|$ converges.
\item If $\sum\limits_{n=1}^\infty a_n$ converges but
$\sum\limits_{n=1}^\infty |a_n|$
diverges we say that\linebreak $\sum\limits_{n=1}^\infty a_n$ is \textbf{conditionally convergent}.
\end{enumerate}

\end{block}
\vfill
\begin{block}{Theorem}
If the series $\sum\limits_{n=1}^\infty |a_n|$ converges then the
series $\sum\limits_{n=1}^\infty a_n$ also converges. That is,
absolute convergence implies convergence.\end{block}
\end{frame}
%----------------------------------------------------------------------------------------
\newcommand{\converges}{\textcolor{M3}{converges}}
\newcommand{\diverges}{\textcolor{W1}{diverges}}
%----------------------------------------------------------------------------------------
\begin{frame}
\StatusBar{1}{6}
\centering
\begin{tabular}{|c|c|c|}\hline
\vphantom{\LARGE $\dfrac12$}
If $\sum a_n$ ... & and $\sum |a_n|$ ... &then we say $\sum a_n$ is ...\\ \hline \hline
%
\vphantom{\LARGE $\dfrac12$}
\onslide<2->{\converges} & \onslide<2->{\converges} & \onslide<3-|handout:0>{\color{spoilercolor} absolutely convergent}\\\hline
%
\vphantom{\LARGE $\dfrac12$}
\onslide<3->{\converges} & \onslide<3->{\diverges} & \onslide<4-|handout:0>{\color{spoilercolor} conditionally convergent}\\ \hline
%
\vphantom{\LARGE $\dfrac12$}
\onslide<4->{\diverges} & \onslide<4->{\diverges} & \onslide<5-|handout:0>{\color{spoilercolor} divergent}\\ \hline
%
\vphantom{\LARGE $\dfrac12$}
\onslide<5->{\diverges} & \onslide<5->{\converges} & \onslide<6-|handout:0>{\alert{not possible!}}\\ \hline
\end{tabular}
\end{frame}
%----------------------------------------------------------------------------------------

%----------------------------------------------------------------------------------------
\begin{frame}[t]
\only<1>{\AnswerYes\QuestionBar{1}{2}}\AnswerSpace
\unote{Example~\eref{text}{eg:SRabsCondB}}
Does the series
\[\sum_{n=1}^\infty \frac{(-1)^n}{n^2}\]
converge or diverge?\vfill

\AnswerBar<2->{1}{2}
\sonslide<2->{
\begin{Ldescription}
\item[\color{C1}Alternating series test:] Let $a_n=\frac{1}{n^2}$. Note $a_n$ has positive, decreasing terms, approaching 0 as $n$ grows. Then $\sum\limits_{n=1}^\infty \frac{(-1)^n}{n^2}$ converges by the alternating series test.
\item[\color{C1}Absolute convergence implies convergence:] The series
 $\sum\limits_{n=1}^\infty\left| \frac{(-1)^n}{n^2}\right|$ is the same as the $p$-series 
 $\sum\limits_{n=1}^\infty \frac{1}{n^2}$, which converges by the $p$-test. Then $\sum\limits_{n=1}^\infty \frac{(-1)^n}{n^2}$  converges absolutely, therefore it converges.
\end{Ldescription}
}
\end{frame}



%---------------------------------------------------------------------------------------
\begin{frame}[t]
\only<1>{\AnswerYes\QuestionBar{2}{2}}\AnswerSpace
Does the series
\[\sum_{n=1}^\infty \frac{\sin(n)}{n^2}\]
converge or diverge?\\[1em]
\vfill

\AnswerBar<2->{2}{2}
\sonslide<2->{

The terms of this series are sometimes positive and sometimes negative, but they do not strictly alternate, so the alternating series test does not apply.

Note that $\sum\limits_{n=1}^\infty \frac{1}{n^2}$ is a convergent series, and $\frac{|\sin n |}{n^2} \leq \frac{1}{n^2}$ for all $n$. Then by the comparison test, $\sum\limits_{n=1}^\infty \frac{|\sin n |}{n^2}$ converges.


Then $\sum\limits_{n=1}^\infty \frac{\sin(n)}{n^2}$ converges absolutely, hence it converges.
}
\end{frame}
%----------------------------------------------------------------------------------------

