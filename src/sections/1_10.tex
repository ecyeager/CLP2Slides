% Copyright 2021 Joel Feldman, Andrew Rechnitzer and Elyse Yeager, except where noted.
% This work is licensed under a Creative Commons Attribution-NonCommercial-ShareAlike 4.0 International License.
% https://creativecommons.org/licenses/by-nc-sa/4.0/


 \begin{frame}{Table of Contents }
\mapofcontentsA{\aj,\atech}
 \end{frame}
%----------------------------------------------------------------------------------------

%----------------------------------------------------------------------------------------
\section{1.10 Partial Fractions}
%----------------------------------------------------------------------------------------
\begin{frame}{Motivation}
\StatusBar{1}{4}
How to integrate $\ds\int\frac{x-2}{(x+1)(2x-1)}~\dee x$? \pause\\\vfill
Useful fact: $\ds\frac{x-2}{(x+1)(2x-1)}=\frac{1}{x+1}-\frac{1}{2x-1}$\\\vfill\pause
So: 
\begin{align*}
\int\frac{x-2}{(x+1)(2x-1)}~\dee x &= \int\frac{1}{x+1}~\dee x-\int\frac{1}{2x-1}~\dee x\\
&=\log|x+1|-\frac12\log|2x-1|+C\end{align*}\vfill\pause

Method of Partial Fractions: \alert{Algebraic method} to turn any rational function (i.e. ratio of two polynomials) into the sum of easier-to-integrate rational functions.
\end{frame}
%----------------------------------------------------------------------------------------
%\section{Linear Factors}
%----------------------------------------------------------------------------------------
\begin{frame}{Distinct Linear Factors}
The rational function \[\frac{\text{numerator}}{K(x-a_1)(x-a_2)\cdots(x-a_j)}\]
can be written as \[\frac{A_1}{x-a_1}+\frac{A_2}{x-a_2}+\cdots+\frac{A_j}{x-a_j}\]
 for some constants $A_1,A_2,\ldots,A_j$, provided 
 \vfill
 (1) the linear roots $a_1,\cdots a_j$ are distinct, and 
 \vfill
 (2) the degree of the numerator is strictly less than the degree of the denominator.

\unote{Equation~\eref{text}{eq:PFdecompa}}
\end{frame}
%----------------------------------------------------------------------------------------
%----------------------------------------------------------------------------------------
\begin{frame}[t]{Distinct Linear Factors}\only<1-2>{\AnswerYes}
$\displaystyle{\frac{7x+13}{(2x+5)(x-2)}=}\onslide<2-|handout:0>{\color{spoilercolor}\frac{A}{2x+5}+\frac{B}{x-2}}$
\vfill

\sonslide<3->{To find $A$ and $B$, simplify the right-hand side by finding a common denominator.
\begin{align*}
\frac{7x+13}{2x^2+x-10}&=\frac{A}{2x+5}+\frac{B}{x-2}=\frac{A(x-2)}{(2x+5)(x-2)}+\frac{B(2x+5)}{(2x+5)(x-2)}\\
&=\frac{A(x-2)+B(2x+5)}{2x^2+x-10}
\intertext{ Cancel denominators}
 7x+13& =A(x-2)+B(2x+5)
\end{align*}
}
\QuestionBar{1}{2}
\end{frame}
%----------------------------------------------------------------------------------------
%----------------------------------------------------------------------------------------
\begin{frame}[t]{Distinct Linear Factors}
\NoSpace<1>
\AnswerYes<1-2>
We found $\textcolor{C3}{
 7x+13=A(x-2)+B(2x+5)}$ for some constants $A$ and $B$. What are $A$ and $B$?\\[1em]
 \textcolor{W1}{Method 1:} set $x$ to convenient values.\\[1em]
 \sonslide<2->{When $x=2$ (chosen to eliminate $A$ from the right hand side), we have $14+13=B\cdot9$, so $\textcolor{C3}{B=3}$.\\
If $x=-\frac{5}{2}$ (chosen to eliminate $B$ from the right hand side), then $-\frac{35}{2}+13=A\left( -\frac52-2\right)$, so $\textcolor{C3}{A=1}$.\\}
\vfill
 \textcolor{W1}{Method 2:} match coefficients of powers of $x$.\\[1em]
 \sonslide<3->{$\textcolor{M5}{7}x+\textcolor{C4}{13}=\textcolor{M5}{(A+2B)}x+\textcolor{C4}{(-2A+5B)}$, so $\textcolor{M5}{7=A+2B}$ and $\textcolor{C4}{13=-2A+5B}$.\\
 Then $\textcolor{M5}{A=7-2B}$, so $\textcolor{C4}{13=-2(\textcolor{M5}{7-2B})+5B}$.\\
 Then $\textcolor{C4}{B=3}$ and $\textcolor{M5}{A=1}$.}
 \AnswerBar{1}{2}
 \end{frame}
%----------------------------------------------------------------------------------------
%----------------------------------------------------------------------------------------
\begin{frame}[t]{Distinct Linear Factors}
All together:
\begin{align*}
\color{black}\frac{7x+13}{2x^2+x-10}&\color{black}=\frac{A}{2x+5}+\frac{B}{x-2}\\
\color{black}A&\color{black}=1,\qquad B=3\\
\sonslide<2->{\frac{7x+13}{2x^2+x-10}&=\frac{1}{2x+5}+\frac{3}{x-2}\\
\int\frac{7x+13}{2x^2+x-10}~\dee x&=\int\left(\frac{1}{2x+5}+\frac{3}{x-2}\right)~\dee x\\
&=\frac12\log|2x+5|+3\log|x-2|+C}
\end{align*}
 \AnswerBar{1}{2}
 \end{frame}
%----------------------------------------------------------------------------------------
\CheckFrame{ \AnswerBar{1}{2}
We check that $\ds\int\frac{7x+13}{2x^2+x-10}=\onslide<beamer>{\frac12\log|2x+5|+3\log|x-2|+C}$ by differentiating.}{
\begin{align*}
\diff{}{x}&\left[ \frac12\log|2x+5|+3\log|x-2|+C \right]=\frac12\cdot\frac{1}{2x+5}\cdot2 +3\cdot\frac{1}{x-2}\\
&=\frac{1}{2x+5}\left( \frac{x-2}{x-2}\right)+\frac{3}{x-2}\left(\frac{2x+5}{2x+5}\right)\\
&=\frac{(x-2)+(6x+15)}{(x-2)(2x+5)}=\frac{7x+13}{2x^2+x-10}
\end{align*}
So, our work checks out.

}

%----------------------------------------------------------------------------------------
\begin{frame}{Distinct Linear Factors}
\StatusBar{1}{3}
$\dfrac{x^2+5}{2x(3x+1)(x+5)}$ is hard to antidifferentiate, but it can be written as\\\pause
$\ds\frac{A}{2x}+\frac{B}{3x+1}+\frac{C}{x+5}$ for some constants $A$, $B$, and $C$.\vfill\pause

Once we find $A$, $B$, and $C$, integration is easy:
\begin{align*}\int&\frac{x^2-24x+5}{2x(3x+1)(x+5)}~\dee x\\
&=\int\left(\frac{A}{2x}+\frac{B}{3x+1}+\frac{C}{x+5} \right)~\dee x\\
&=\frac{A}2\log|x|+\frac{B}3\log|3x+1|+C\log|x+5|+D \end{align*}
\QuestionBar{2}{2}
\end{frame}
%----------------------------------------------------------------------------------------
%----------------------------------------------------------------------------------------
\begin{frame}[t]{Distinct Linear Factors}
 \AnswerBar{2}{2}
 \sMoreSpace<2>
\nsMoreSpace<1>
\color{spoilercolor}\begin{align*}
&\color{black}\frac{x^2+5}{2x(3x+1)(x+5)}=\frac{A}{2x}+\frac{B}{3x+1}+\frac{C}{x+5}
\intertext{\color{black}Find constants $A$, $B$, and $C$.\linebreak
 Start: make a common denominator}
\sonslide<2->{
&=\frac{A(3x+1)(x+5)}{2x(3x+1)(x+5)}+
\frac{B(2x)(x+5)}{2x(3x+1)(x+5)}+
\frac{C(2x)(3x+1)}{2x(3x+1)(x+5)}\\
&=\frac{A(3x+1)(x+5)+B(2x)(x+5)+C(2x)(3x+1)}{2x(3x+1)(x+5)}
\intertext{Cancel off denominator}
&x^2+5=A(3x+1)(x+5)+B(2x)(x+5)+C(2x)(3x+1)
}
\end{align*}
\end{frame}
%----------------------------------------------------------------------------------------%----------------------------------------------------------------------------------------
\begin{frame}<beamer>[t]{Distinct Linear Factors}
\AnswerSpace\only<1>{\AnswerYes}
\[
x^2+5=A(3x+1)(x+5)+B(2x)(x+5)+C(2x)(3x+1)\]
\sonly<2>{\textcolor{W1}{Method 1: }use convenient values for $x$. (Setting $x=0$ eliminates $B$ and $C$ from the r.h.s. Setting $x=-5$ eliminates $A$ and $B$ from the r.h.s. Setting $x=-1/3$ eliminates $A$ and $C$ from the r.h.s.)  

\begin{align*}x&=0:& 5&=A(1)(5)+0+0 &\implies A&=1\\
x&=-5:& 30&=0+0+C(-10)(-14)&\implies C&=\frac{3}{14}\\
x&=-\frac13:& \frac19+5&=0+B\left(-\frac23\right)\left(-\frac13+5\right)+0 &\implies B&=-\frac{23}{14}
\end{align*}
So:
\[
\frac{x^2+5}{2x(3x+1)(x+5)}=\frac{1}{2x}+\frac{-23/14}{3x+1}+\frac{3/14}{x+5}\]}
\sonly<3>{\textcolor{W1}{Method 2: }Simplify numerator: gather the $x^2$ terms, the $x^1$ terms and the $x^0$ terms 
\begin{align*}
x^2+5&={x^2(3A+2B+6C)+x(16A+10B+2C)+(5A)}
\intertext{Equate coefficients of matching powers of $x$}
\text{Constant term: } 5&=5A \implies A=1\\
\text{Linear term: } 0&=16A+10B+2C=16+10B+2C\\
\implies C&=-8-5B\\
\text{Squared term: } 1&=3A+2B+6C=3+2B+6(-8-5B)\\
\implies B&=-23/14\\
\implies C&=-8-5B=3/14
\end{align*}}
 \AnswerBar{2}{2}
\end{frame}
%----------------------------------------------------------------------------------------

%----------------------------------------------------------------------------------------
\CheckFrame{ \AnswerBar{2}{2}
Let's check that \[\ds\frac{x^2+5}{2x(3x+1)(x+5)}=\onslide<beamer>{\frac{1}{2x}-\frac{23/14}{3x+1}+\frac{3/14}{x+5}.}\]
}{
\begin{align*}&\textcolor{C4}{\frac{1}{2x}}-\textcolor{C3}{\frac{23/14}{3x+1}}+\textcolor{C2}{\frac{3/14}{x+5}}\\&=
\textcolor{C4}{\frac{1(3x+1)(x+5)}{2x(3x+1)(x+5)}} -\textcolor{C3}{\frac{23/14(2x)(x+5)}{(2x)(3x+1)(x+5)}}+\textcolor{C2}{\frac{3/14(2x)(3x+1)}{(2x)(3x+1)(x+5)}}\\
&=\frac{\textcolor{C4}{(3x^2+16x+5)}-\textcolor{C3}{(\frac{23}7x^2+\frac{115}{7}x)}+\textcolor{C2}{(\frac{9}{7}x^2+\frac{3}{7}x)}}{2x(3x+1)(x+3)}\\
&=\frac{x^2+5}{2x(3x+1)(x+3)}
\end{align*}

So, our algebra is good.
}
%----------------------------------------------------------------------------------------
%----------------------------------------------------------------------------------------
\begin{frame}[t]{Distinct Linear Factors}
 \AnswerBar{2}{2}
\AnswerYes<1>
All together:
\begin{align*}
\frac{x^2+5}{2x(3x+1)(x+5)}&=\frac{1}{2x}-\frac{23/14}{3x+1}+\frac{3/14}{x+5}\\
\sonslide<2->{
\int\frac{x^2-24x+5}{2x(3x+1)(x+5)}~\dee x&\color{spoilercolor}=\int\left(\frac{1}{2x}-\frac{23/14}{3x+1}+\frac{3/14}{x+5} \right)~\dee x\\
&\color{spoilercolor}=\frac{1}2\log|x|-\frac{23}{42}\log|3x+1|+\frac{3}{14}\log|x+5|+C
}
\end{align*}

\end{frame}
%----------------------------------------------------------------------------------------
%\section{Repeated Linear Factors}
%----------------------------------------------------------------------------------------
\begin{frame}[t]
\QuestionBar<2>{1}{3}
\AnswerBar<3>{1}{3}

\begin{block}{Repeated Linear Factors}
A rational function $\displaystyle\frac{P(x)}{(x-1)^4}$, where $P(x)$ is a polynomial of degree strictly less than 4, can be written as
\[\frac{A}{x-1}+\frac{B}{(x-1)^2}+\frac{C}{(x-1)^3}+\frac{D}{(x-1)^4} \]
for some constants $A$, $B$, $C$, and $D$.
\end{block}\pause
\[\frac{5x-11}{(x-1)^2} =\sonslide<3->{ \frac{A}{x-1}+\frac{B}{(x-1)^2} }\]
\unote{Equation~\eref{text}{eq:PFdecompb}}
\end{frame}
%----------------------------------------------------------------------------------------
%----------------------------------------------------------------------------------------
\begin{frame}[t]Set up the form of the partial fractions decomposition. (You do not have to solve for the parameters.)
\[\frac{3x+16}{(x+5)^3}=\sonslide<2->{\frac{A}{x+5}+\frac{B}{(x+5)^2}+\frac{C}{(x+5)^3}}\]
\QuestionBar<1>{2}{3}\QuestionBar<1>{3}{3}\AnswerYes<1>
\AnswerBar<2>{2}{3}
\AnswerBar<2>{3}{3}
\vfill
\[\frac{-2x-10}{(x+1)^2(x-1)}=\sonslide<2->{\frac{A}{x+1}+\frac{B}{(x+1)^2}+\frac{C}{x-1}}\]
\vfill
\end{frame}
%----------------------------------------------------------------------------------------
%----------------------------------------------------------------------------------------
%----------------------------------------------------------------------------------------

%----------------------------------------------------------------------------------------
\begin{frame}[t]{Irreducible Quadratic Factors}
Sometimes it's not possible to factor our denominator into linear factors with real terms.\vfill

\begin{multicols}{2}
\begin{tikzpicture}
\myaxis{x}{1.5}{1.5}{y}{1}{1}

\onslide<1|handout:0>{\draw[C2,thick]plot[domain=-1:1.5](\x,{(\x+.5)*(\x-1.25))});}
\onslide<2->{\draw[C2,thick]plot[domain=-1:1.5](\x,{(\x+.5)*(\x-1.25))})node[above,xshift=1cm]{$c(x-a)(x-b)$};
\xcoord{-.5}{a} \xcoord{1.25}{b}}
\end{tikzpicture}\\
\onslide<2->{If a quadratic function has real roots $a$ and $b$ (possibly $a=b$, possibly $a \neq b$), then we can write it as $c(x-a)(x-b)$ for some constant $c$.}
\columnbreak
\begin{center}
\begin{tikzpicture}
\myaxis{x}{1.5}{1.5}{y}{1}{1}
\draw[C2,thick]plot[domain=-1:1](\x,{\x*\x+.25});
\end{tikzpicture}\end{center}

\onslide<2->{ If a quadratic function has no real roots, then it can't be factored into (real) linear factors. It is \alert{irreducible}.}
\end{multicols}
\end{frame}
%----------------------------------------------------------------------------------------
\begin{frame}[t]{Irreducible Quadratic Factors}
\QuestionBar<1>{1}{2}
\AnswerYes<1,3>
\sNoSpace<1,3>
\nsNoSpace<1-2>
\AnswerBar<2>{1}{2}
\sQuestionBar<3>{2}{2}
\nsQuestionBar<2>{2}{2}
\AnswerBar<4>{2}{2}
When the denominator has an irreducible quadratic factor $x^2+bx+c$, we add a term $\dfrac{Ax+B}{x^2+bx+c}$ to our composition. (The degree of the numerator must still be smaller than the degree of the denominator.)

Write out the form of the partial fraction decomposition (but do not solve for the parameters):
\vfill
\begin{itemize}[<+->]
\item
$\ds\frac{1}{(x+1)(x^2+1)}=\sonslide<+->{\frac{A}{x+1}+\frac{Bx+C}{x^2+1}}$\vfill
\item
$\ds\frac{3x^2-x+5}{(x^2+1)(x^2+2)}=\sonslide<+>{\frac{Ax+B}{x^2+1}+\frac{Cx+D}{x^2+2}}$\vfill
\end{itemize}
\unote{Equation~\eref{text}{eq:PFdecompc}}
\end{frame}
%----------------------------------------------------------------------------------------
\begin{frame}[t]
\sStatusBar{1}{4}
\AnswerSpace
\AnswerYes<3>
\QuestionBar<3>{1}{3}
\AnswerBar<4>{1}{3}

The purpose of the partial fraction decomposition is to end up with functions \alert{that we can integrate}. 

\pause
\begin{itemize}[<+->]
\item Recall: $\ds\int\frac{1}{x^2+1}\dee x = \arctan x+C$. \vfill
\item Evaluate: $\ds\int\frac{1}{(x+1)^2+1}\dee x$
\vfill
\sonslide<+->{
$u=x+1$, $\dee u = \dee x$:\\ $\ds\int\frac{1}{u^2+1}\dee u = \arctan u + C = \arctan(x+1)+C$
}\vfill\vfill

\end{itemize}
\end{frame}
%----------------------------------------------------------------------------------------
%----------------------------------------------------------------------------------------
\begin{frame}[t]
\QuestionBar<1>{2}{3}
\AnswerYes<1>
\AnswerBar<2>{2}{3}
Evaluate $\ds\int\frac{4}{(3x+8)^2+9}\dee x$

\sonslide<2->{
\begin{align*}
&=\int\frac{4}{9\left(\frac{(3x+8)^2}{9}+1\right)}\dee x\\
&=\frac49\int\frac{1}{\left(\frac{3x+8}{3}\right)^2+1}\ \dee x\\
&=\frac49\int\frac{1}{\left(x+\frac83\right)^2+1}\ \dee x\\
\text{$u=x+\frac83$, \quad $\dee u = \dee x$}\hspace{1cm}
&=\frac49\int\frac{1}{u^2+1}\ \dee u\\
&=\frac49\arctan u + C\\
&=\frac49\arctan\left(x+\frac83\right)+C
\end{align*}}
\end{frame}

%----------------------------------------------------------------------------------------
\CheckFrame{
\AnswerBar{2}{3}
We found $\ds\int\frac{4}{(3x+8)^2+9}\dee x =\onslide<beamer>{ \frac49\arctan\left(x+\frac83\right)+C
.}$
}{
\begin{align*}
\diff{}{x}&\left\{ \frac49\arctan\left(x+\frac83\right)+C\right\}=\frac49\cdot\frac{1}{\left(x+\frac83\right)^2+1}\\
&=\frac{4}{9\left(\left(x+\frac83\right)^2+1\right)}\\
&=\frac{4}{3^2\left(x+\frac83\right)^2+9} \\
&=\frac{4}{\left(3x+8\right)^2+9}
\end{align*}
So, our answer works.}
%%----------------------------------------------------------------------------------------
%----------------------------------------------------------------------------------------
\begin{frame}[t]
\QuestionBar<1>{3}{3}\AnswerYes<1>
\AnswerBar<2>{3}{3}
Evaluate $\ds\int\frac{x+1}{x^2+2x+2}\dee x$. \\[1em] (Hint: start by completing the square.)

\sonslide<2->{
\begin{align*}
&&&=
\int\frac{x+1}{(x+1)^2+1}\dee x\\
&\text{Let $y=x+1$, $\dee y = \dee x$:}&
&=\int\frac{y}{y^2+1}\dee y\\
&\text{Let $u=y^2+1$, $\dee u=2y\ \dee y$:}&
&=\frac12\int\frac{1}{u}\ \dee u\\
&&&=\frac12\log\left| u\right| +C\\
&&&=\frac12\log\left| y^2+1\right| +C\\
&&&=\frac12\log\left| (x+1)^2+1\right| +C
\end{align*}}
\end{frame}

%----------------------------------------------------------------------------------------
\CheckFrame{\AnswerBar{3}{3}
We found $\ds\int\frac{x+1}{x^2+2x+2}\dee x =\onslide<beamer>{\frac12\log\left| (x+1)^2+1\right| +C.}
$
}{
\begin{align*}
\diff{}{x}\left\{  \frac12\log\left| (x+1)^2+1\right| +C\right\}&=\frac12\cdot\frac{2(x+1)}{(x+1)^2+1}\\
&=\frac{x+1}{(x+1)^2+1}\\
&=\frac{x+1}{x^2+2x+2}
\end{align*}
So, our answer works.}
%%----------------------------------------------------------------------------------------

%----------------------------------------------------------------------------------------
%----------------------------------------------------------------------------------------
\section{Long Division}
%----------------------------------------------------------------------------------------
%----------------------------------------------------------------------------------------
\begin{frame}[t]
\StatusBar{1}{5}

These rules work \alert{only} when the degree of the numerator is \alert{less than} the degree of the denominator.\pause\vfill

$\ds\int\frac{x^3}{(x-2)^2(x-3)(x-4)^2}~\dee x \quad\onslide<3-|handout:0>{\textcolor{M3}{\checkmark}}$\hfill $\ds\int\frac{x^5}{(x-2)^2(x-3)(x-4)^2} ~\dee x\quad\onslide<4-|handout:0>{\color{M4}\text{X}}$
\vfill
\onslide<5->{If the degree of the numerator is too large, we use polynomial long division.}
\end{frame}
%----------------------------------------------------------------------------------------
%----------------------------------------------------------------------------------------
%----------------------------------------------------------------------------------------
\begin{frame}[t]
\QuestionBar<1>{1}{3}\AnswerYes<1>
\AnswerBar<2>{1}{3}

Evaluate $\ds\int\frac{8x^2+22x+23}{2x+3} \ \dee x$.
\sonslide<2->{\[\polylongdiv{8x^2+22x+23}{2x+3}\]
\vfill So,
\begin{align*}
\frac{8x^2+22x+23}{2x+3}&=4x+5+\frac{8}{2x+3}\\[10pt]
\int \frac{8x^2+22x+23}{2x+3} ~\dee x&=
2x^2+5x+4\log|2x+3|+C
\end{align*}}
\end{frame}
%----------------------------------------------------------------------------------------
%----------------------------------------------------------------------------------------
\CheckFrame{\AnswerBar{1}{3}
We computed \[\int \frac{8x^2+22x+23}{2x+3} ~\dee x=
\onslide<beamer>{2x^2+5x+4\log|2x+3|+C.}\]
}{
\begin{align*}
\diff{}{x}&\left\{2x^2+5x+4\log|2x+3|+C\right\}\\
&=4x+5+\frac{8}{2x+3}\\
&=\frac{4x(2x+3)+5(2x+3)+8}{2x+3}\\
&=\frac{8x^2+12x+10x+15+8}{2x+3}\\
&=\frac{8x^2+22x+23}{2x+3}
\end{align*}
So, our solution works.
}
%----------------------------------------------------------------------------------------
\begin{frame}[t]
\AnswerYes<1>\QuestionBar<1>{2}{3}
\AnswerBar<2>{2}{3}
Evaluate $\ds\int\frac{3x^3+x+3}{x-2} \ \dee x$.
\sonslide<2->{\[\polylongdiv{3x^3+x+3}{x-2}\]
\vfill
So,
\begin{align*}
\int\frac{3x^3+x+3}{x-2}~\dee x&=
\int
\left(3x^2+6x+13+\frac{29}{x-2}
\right)~\dee x\\
&=x^3+3x^2+13x+29\log|x-2|+C
\end{align*}
}
\end{frame}
%----------------------------------------------------------------------------------------
%----------------------------------------------------------------------------------------
\CheckFrame{\AnswerBar{2}{3}
We found 
\[
\int\frac{3x^3+x+3}{x-2}~\dee x=\onslide<beamer>{x^3+3x^2+13x+29\log|x-2|+C.}
\]
}{
\begin{align*}
\diff{}{x}&\left\{x^3+3x^2+13x+29\log|x-2|+C
\right\}\\&=3x^2+6x+13+\frac{29}{x-2}\\
&=\frac{3x^2(x-2)+6x(x-2)+13(x-2)+29}{x-2}\\
&=\frac{3x^3-6x^2+6x^2-12x+13x-26+29}{x-2}
\\&=\frac{3x^3+x+3}{x-2}
\end{align*}
}
%----------------------------------------------------------------------------------------
%----------------------------------------------------------------------------------------

\begin{frame}[t]
\QuestionBar<1>{3}{3}\AnswerYes<1>
\AnswerBar<2>{3}{3}

Evaluate $\ds\int\frac{3x^2+1}{x^2+5x}\ \dee x$.

\sonslide<2->{
\parbox{4cm}{\polylongdiv{3x^2+1}{x^2+5x}}\hfill
\parbox{5cm}{So, $\ds\frac{3x^2+1}{x^2+5x}=3+\frac{-15x+1}{x^2+5x}$}
\vfill
Now, we can use partial fraction decomposition.
\begin{align*}
\frac{-15x+1}{x(x+5)}&=\frac{A}{x}+\frac{B}{x+5}=\frac{(A+B)x+5A}{x(x+5)}\\
A=\frac15,\quad B&=-15-\frac15=-\frac{76}{5}\\
\int \frac{3x^2+1}{x^2+5x}~\dee x&=\int\left(
3+\frac{1/5}{x}-\frac{76/5}{x+5}
\right)~\dee x\\
&=3x+\frac15\log|x|-\frac{76}5\log|x+5|+C
\end{align*}
}
\end{frame}
%----------------------------------------------------------------------------------------
%----------------------------------------------------------------------------------------
\CheckFrame{\AnswerBar{3}{3}
We found $\ds\int\frac{3x^2+1}{x^2+5x}\ \dee x =\onslide<beamer>{ 3x+\frac15\log|x|-\frac{76}5\log|x+5|+C.}$
}{
\begin{align*}
&\diff{}{x}\left\{3x+\frac15\log|x|-\frac{76}5\log|x+5|+C\right\}\\
&=3+\frac{1}{5x}-\frac{76}{5(x+5)}
\\&=3\left(\frac{5x(x+5)}{5x(x+5)}\right)+\frac{1}{5x}\left(\frac{x+5}{x+5}\right)-\frac{76}{5(x+5)}\left(\frac{x}{x}\right)
\\&=\frac{(15x^2+75x)+(x+5)-(76x)}{5x(x+5)}\\
&=\frac{15x^2+5}{5x(x+5)}=\frac{3x^2+1}{x^2+5x}
\end{align*}
So, our solution works.
}
%----------------------------------------------------------------------------------------
%----------------------------------------------------------------------------------------
\begin{frame}[t]{Factoring}
\[P(x)=x^3+2x^2-5x-6\]

\sonslide<2->{\begin{itemize}\color{spoilercolor}
\item To start, let's guess a root.
\begin{itemize}\color{spoilercolor}
\item Since $P(x)$ has integer coefficients, any integer root must divide 6 exactly.
\item So the only possible integer roots are $\pm 1$, $\pm 2$, $\pm 3$, and $\pm 6$. We'll try each until one works.
\begin{itemize}\color{spoilercolor}
\item $P(1) = -8 \neq 0 \implies$ 1 is not a root
\item $P(-1) = 0 \implies$ \alert{-1 is a root}. Therefore, $(x+1)$ is a factor.
\end{itemize}
\end{itemize}
\item Long division gives the rest:\\

\parbox{.35\textwidth}{
\tiny
	\polylongdiv{x^3+2x^2-5x-6}{x+1}}
\parbox{.55\textwidth}{\raggedright	$P(x)={(x+1)(x^2+x-6)}={(x+1)(x-2)(x+3)}$}


\end{itemize}}
\end{frame}

%----------------------------------------------------------------------------------------
%----------------------------------------------------------------------------------------
\begin{frame}[t]{Factoring}
\[P(x)=2x^3-3x^2+4x-6\]

\sonslide<2->{
Notice that the first two terms and the last two terms have the same ratios: $\frac{2x^3}{-3x^2}=\alert{\frac{2x}{-3}}=\frac{4x}{-6}$.  So,
we can factor $2x-3$ out of both pairs.
\begin{alignat*}{3}
P(x)&=2x^3-3x^2&+&4x-6\\
&=\alert{(2x-3)}(x^2)&+&\alert{(2x-3)}(2)\\
&=\alert{(2x-3)}(x^2+2)
\end{alignat*}}
\end{frame}

%----------------------------------------------------------------------------------------%----------------------------------------------------------------------------------------
