% Copyright 2021 Joel Feldman, Andrew Rechnitzer and Elyse Yeager, except where noted.
% This work is licensed under a Creative Commons Attribution-NonCommercial-ShareAlike 4.0 International License.
% https://creativecommons.org/licenses/by-nc-sa/4.0/


 \begin{frame}{Table of Contents }
\mapofcontentsC{\cc}
 \end{frame}
%----------------------------------------------------------------------------------------
%----------------------------------------------------------------------------------------
%----------------------------------------------------------------------------------------
%----------------------------------------------------------------------------------------

\section{3.3.4 Alternating Series Test}

%----------------------------------------------------------------------------------------
%----------------------------------------------------------------------------------------
\begin{frame}[t]{Review}
\sStatusBar{1}{3}
\AnswerSpace
\AnswerYes<2>
Let $\ds S_N=\sum_{n=1}^N a_n$.\\[1em]

Simplify:  $S_{N}-S_{N-1}$.
\hfill
(This will come in handy soon.)

\onslide<2-|handout:0>{\color{spoilercolor}\begin{align*}
S_N&=a_1+a_2+a_3+\cdots+a_{N-1}+a_N\\
S_{N-1}&=a_1+a_2+a_3+\cdots+a_{N-1}\\
\sonslide<3->{S_N-S_{N-1}&=a_N}
\end{align*}}

\vfill

\end{frame}
%----------------------------------------------------------------------------------------
%----------------------------------------------------------------------------------------
\begin{frame}[t]{Alternating Series}

\begin{block}{Alternating Series}
The series
\begin{equation*}
A_1-A_2+A_3-A_4+\cdots =\sum_{n=1}^\infty (-1)^{n-1} A_n
\end{equation*}
is alternating if every $A_n\ge 0$.
\end{block}

\begin{multicols}{2}
Alternating series:\\[2em]
\begin{itemize}
\item $1-2+3-4+5-6+7-8+\cdots$\\[1em]
\item $\ds 1-\frac12+\frac13-\frac14+\frac15-\cdots$
\end{itemize}
\columnbreak

Not alternating:\\[2em]
\begin{itemize}
\item $\cos(1)+\cos(2)+\cos(3)+\cdots$
\item $1-\left(-\dfrac12\right)+\dfrac13-\left(-\dfrac14\right)+\cdots$
\end{itemize}
\end{multicols}
\end{frame}
%----------------------------------------------------------------------------------------

%----------------------------------------------------------------------------------------
\begin{frame}[t]
\begin{tikzpicture}
\only<1-|handout:1>{\weights{6,-5,4,-3,2,-1}{6,-5,4,-3,2,-1}{}}
\sonslide<1>{\draw(5.25,1.5)node{\parbox{\textwidth}{Note: these terms alternate signs, \alert{and} their magnitudes are\\ decreasing: $|6|>|-5|>|4|>|-3|>|2|>|-1|$}};}
\onslide<8-|handout:2>{\begin{scope}[yshift=4cm]
\myaxis{n}{0}{7}{}{0}{2.2}
\draw(1,2)node[vertex,label=above: $S_1$](S1){};
\ycoord{2}{6}
\onslide<9->{\draw(2,1/3)node[vertex,label=below:$S_2$](S2){};
\ycoord{1/3}{1}}
\onslide<14->{\draw[W1,->] (S1)--(S2)node[midway,left]{$a_2$};}
\onslide<10->{\draw(3,5/3)node[vertex,label=above: $S_3$](S3){};
\ycoord{5/3}{5}}
\onslide<15->{\draw[C1,->] (S2)--(S3)node[midway,left]{$a_3$};}
\onslide<11->{\draw(4,2/3)node[vertex,label=below:$S_4$](S4){};
\ycoord{2/3}{2}}
\onslide<16->{\draw[W1,->] (S3)--(S4)node[midway,left]{$a_4$};}
\onslide<12->{\draw(5,4/3)node[vertex,label=above: $S_5$](S5){};
\ycoord{4/3}{4}}
\onslide<17->{\draw[C1,->] (S4)--(S5)node[midway,above]{$a_5$};}
\onslide<13->{\draw(6,1)node[vertex,label=below:$S_6$](S6){};
\ycoord{1}{3}}
\onslide<18->{\draw[W1,->] (S5)--(S6)node[midway,below]{$a_6$};}
\end{scope}}
\sonslide<7-9>{\draw (0.7,3.3)node{ 6};
\draw (2.4,3.3)node{ $-5$};
\draw (3.8,3.3)node{ $4$};
\draw (4.9,3.3)node{ $-3$};
\draw (5.9,3.3)node{ $2$};
\draw (6.5,3.3)node{ $-1$};}
\end{tikzpicture}
\end{frame}
%----------------------------------------------------------------------------------------
%----------------------------------------------------------------------------------------


\begin{frame}[t]
\StatusBar{1}{6}
Consider an alternating series $a_1-a_2+a_3-a_4+\cdots$, where $\{a_n\}$ is a sequence with positive, \alert{decreasing} terms and with $\lim\limits_{n\rightarrow\infty }a_n = 0$.
\begin{tikzpicture}
%partial sums: \pm 4e^{-\sqrt x}+2
\myaxis{n}{0}{11}{}{0}{4}
\foreach \n in {1,3,...,19}{
	\SQRT{\n}{\e}
	\EXP{-\e}{\a}
	\MULTIPLY{\a}{4}{\b}
	\ADD{\b}{2}{\y}
	\draw(\n/2,\y)node[C1,vertex,label=above:{$S_{\n}$}](S\n){};
	}
\foreach \n in {2,4,...,20}{
	\SQRT{\n}{\e}
	\EXP{-\e}{\a}
	\MULTIPLY{\a}{-4}{\b}
	\ADD{\b}{2}{\y}
	\draw(\n/2,\y)node[W1,vertex,label=below:{$S_{\n}$}](S\n){};
	}
	
%S1 - S3
\onslide<2-4|handout:0>{\draw[->,W1](S1)--(S2)node[midway,left]{$a_2$};}
\onslide<3-6|handout:0>{\draw[->,C1](S2)--(S3)node[midway,right]{$a_3$};}
\onslide<4-|handout:0>{\draw[C1] plot[domain=1:21,smooth]({\x/2},{4*exp(-sqrt(\x))+2});}
\onslide<5-|handout:0>{\draw[->,W1](S3)--(S4)node[midway,right]{$a_4$};}
\onslide<6-|handout:0>{\draw[W1] plot[domain=1:20,smooth]({\x/2},{-4*exp(-sqrt(\x))+2});}

\end{tikzpicture}
\vfill
Since $a_2>a_3$, we have $a_1-(a_2-a_3)<a_1$, so $S_3<S_1$.

\color{C1}
\onslide<4-|handout:0>{Odd-indexed partial sums are decreasing.}
\vfill
\onslide<6-|handout:0>{
Since $a_3>a_4$, we have $a_1-a_2+(a_3-a_4)>a_1-a_2$, so $S_4>S_2$.\\
\textcolor{W1}{Even-indexed partial sums are increasing.}}
\vfill
\end{frame}
%----------------------------------------------------------------------------------------
%----------------------------------------------------------------------------------------%----------------------------------------------------------------------------------------

\begin{frame}[t]
\label<6>{note3.3.4a}
\StatusBar{1}{8}
\begin{tikzpicture}
%partial sums: \pm 4e^{-\sqrt x}+2
\myaxis{n}{0}{11}{}{0}{4}
\foreach \n in {1,3,...,19}{
	\SQRT{\n}{\e}
	\EXP{-\e}{\a}
	\MULTIPLY{\a}{4}{\b}
	\ADD{\b}{2}{\y}
	\draw(\n/2,\y)node[C1,vertex,label=above:{$S_{\n}$}](S\n){};
	}
\foreach \n in {2,4,...,20}{
	\SQRT{\n}{\e}
	\EXP{-\e}{\a}
	\MULTIPLY{\a}{-4}{\b}
	\ADD{\b}{2}{\y}
	\draw(\n/2,\y)node[W1,vertex,label=below:{$S_{\n}$}](S\n){};
	}
	

\foreach \s in {2,...,5}{
\only<\s|handout:0>{
	\SUBTRACT{\s}{1}{\t}
	\SUBTRACT{22}{\t}{\a}
	\draw[C1,dashed] (S\t)--+(\a/2,0);
	\draw[W1,dashed] (S\s)--+(\a/2-.5,0);}
}

\onslide<8-|handout:0>{\draw[dashed] (0,2)--(10.25,2)node[right]{$\lim\limits_{n \to \infty}S_n$};}
\end{tikzpicture}
\vfill

\onslide<2->{
\begin{itemize}
\item For all $n \ge 2$, $S_n$ lies between $S_1$ and $S_2$.
\foreach \s in {3,...,5}{
\SUBTRACT{\s}{1}{\t}
\onslide<\s->{\item For all $n \ge \s$, $S_n$ lies between $S_{\t}$ and $S_{\s}$.}
}
\end{itemize}
}
\vfill
\onslide<6->{The difference between consecutive sums $S_{n}$ and $S_{n-1}$ is:\\}
 \onslide<7-|handout:0>{$|a_n|$, which approaches 0.}
\end{frame}
%----------------------------------------------------------------------------------------

%----------------------------------------------------------------------------------------
\begin{frame}[t]
\unote{Theorem~\eref{text}{thm:SRalternating}}
\begin{block}{Alternating Series Test}
Let $\big\{a_n\big\}_{n=1}^\infty$
be a sequence of real numbers that obeys
\begin{enumerate}[(i)]
\item $a_n\ge 0$ for all $n\ge 1$;
\item $a_{n+1}\le a_n$  for all $n\ge 1$ (i.e. the
sequence is monotone decreasing);
\item  and $\lim\limits_{n\rightarrow\infty}a_n=0$.
\end{enumerate}
Then
\begin{equation*}
a_1-a_2+a_3-a_4+\cdots=\sum\limits_{n=1}^\infty (-1)^{n-1} a_n =S
\end{equation*}
converges and, for each natural number $N$,  
$S-S_N$ is between $0$ and (the
first dropped term) $(-1)^N a_{N+1}$. Here $S_N$ is, as previously,
the $N^{\rm th}$ partial sum $\sum\limits_{n=1}^N (-1)^{n-1} a_n$.
\end{block}
\end{frame}
%----------------------------------------------------------------------------------------
%----------------------------------------------------------------------------------------
\begin{frame}
\begin{block}{Alternating Series Test (abridged)}
Let $\big\{a_n\big\}_{n=1}^\infty$
be a sequence of real numbers that obeys
\begin{enumerate}[(i)]
\item $a_n\ge 0$ for all $n\ge 1$;
\item $a_{n+1}\le a_n$  for all $n\ge 1$ (i.e. the
sequence is monotone decreasing);
\item  and $\lim\limits_{n\rightarrow\infty}a_n=0$.
\end{enumerate}
Then
\begin{equation*}
a_1-a_2+a_3-a_4+\cdots=\sum\limits_{n=1}^\infty (-1)^{n-1} a_n
\end{equation*}
converges.
\end{block}
\begin{itemize}
\item True or false: the harmonic series $\ds\sum_{n=1}^\infty \frac{1}{n}$ converges.
\item True or false: the alternating harmonic series 
$\ds\sum_{n=1}^\infty \frac{(-1)^n}{n}$ 
converges.
\end{itemize}
\MoreSpace
\end{frame}

%----------------------------------------------------------------------------------------

\begin{frame}<beamer>[t]
\sStatusBar{1}{3}
\nsStatusBar{1}{2}
\AnswerYes<1-2>
\color{spoilercolor}
Let $a_n=\frac1n$.\pause
\begin{enumerate}[(i)]\color{spoilercolor}
\item $a_n \ge 0 $
\item $a_{n+1} \le a_n$
\item $\lim\limits_{n\to\infty}a_n = 0$
\end{enumerate}
\sonslide<3->{\begin{itemize}\color{spoilercolor}
\item We've already seen that the harmonic series $\ds\sum_{n=1}^\infty \frac{1}{n}$ diverges.
\item By the Alternating Series Test, $\sum_{n=1}^\infty (-1)^{n-1}a_n$ converges. That is,
\[\ds\sum_{n=1}^\infty \frac{(-1)^n}{n}\]
converges.
\end{itemize}}
\end{frame}
%----------------------------------------------------------------------------------------
%----------------------------------------------------------------------------------------
%----------------------------------------------------------------------------------------
\begin{frame}
{Divergence Test $+$ Alternating Series Test}
\StatusBar{1}{5}

\centering
\unote{Warning~\eref{text}{wrn:SRdivTest}}
\begin{tikzpicture}[node distance = 3.25cm, auto,inner sep=3mm,->] 
    \node [ellipse,draw] (lim) {$\ds\lim_{n\to \infty}a_n=$ ?}; 
    \onslide<2->{\node [rectangle,draw,below left of=lim] (nonzero) {$\ds\sum a_n$ diverges};
    \node[below of=nonzero, node distance=8mm]{(divergence test)}; 
    \draw [ultra thick] (lim)--(nonzero)node[midway,left]{$\neq 0$};}
    \onslide<3->{\node [ellipse,draw, below right of=lim] (zero) {\parbox{2.75cm}{Alternating and\\ $|a_{n+1}| \leq |a_{n}| $?}}; 
    \draw [ultra thick] (lim) -- (zero) node[midway]{$=0$}; }
    \onslide<4->{\node [rectangle,draw, below right of=zero,yshift=-2mm] (no) {\parbox{4cm}{$\ds\sum a_n$ may converge or diverge; use another test}};
    \draw [ultra thick] (zero)--(no) node[midway,left]{no};}
        \onslide<5->{\node [W1,rectangle,draw, below left of=zero] (ast) {$\sum a_n$ converges}; 
    \node[W1,below of=ast,node distance=8mm]{(alternating series test)};
    \draw [W1,ultra thick] (zero)--(ast) node[midway,left]{yes};}

\end{tikzpicture}
\end{frame}
%----------------------------------------------------------------------------------------

%----------------------------------------------------------------------------------------
\begin{frame}[t]
\begin{block}{Alternating Series Test}
Let $\big\{a_n\big\}_{n=1}^\infty$
be a sequence of real numbers that obeys
$a_n\ge 0$ for all $n\ge 1$;
$a_{n+1}\le a_n$  for all $n\ge 1$;
 and $\lim\limits_{n\rightarrow\infty}a_n=0$. Then
$\sum\limits_{n=1}^\infty (-1)^{n-1} a_n =S$
converges and  $S-S_N$ is between $0$ and  $(-1)^N a_{N+1}$. 
\end{block}

Using a computer, you find
$\displaystyle\sum_{n=1}^{99} \frac{(-1)^{n-1}}{n} \approx 0.698. $\\
How close is that to the value $\ds\sum_{n=1}^{\infty}  \frac{(-1)^{n-1}}{n} $?
\AnswerYes<1>\NoSpace<1>

\sonslide<2->{\vfill
$\ds\frac{-1}{100} = \frac{(-1)^{100-1}}{100} \leq \sum_{n=1}^{\infty} \frac{(-1)^n}{n}-\sum_{n=1}^{99} \frac{(-1)^n}{n}\leq 0$.\\
That is, the actual series has a sum in the interval $[0.688,0.698]$.
}
\end{frame}
%----------------------------------------------------------------------------------------
\begin{frame}[t]
\NoSpace<1>\AnswerYes<1>
\begin{block}{Alternating Series Test}
Let $\big\{a_n\big\}_{n=1}^\infty$
be a sequence of real numbers that obeys
$a_n\ge 0$ for all $n\ge 1$;
$a_{n+1}\le a_n$  for all $n\ge 1$;
 and $\lim\limits_{n\rightarrow\infty}a_n=0$. Then
$\sum\limits_{n=1}^\infty (-1)^{n-1} a_n =S$
converges and  $S-S_N$ is between $0$ and  $(-1)^N a_{N+1}$. 
\end{block}

Using a computer, you find
$\ds\sum_{n=1}^{19}(-1)^{n-1}\frac{n^2}{n^2+1} \approx 0.6347. $\\
How close is that to the value $\ds\sum_{n=1}^{\infty} (-1)^{n-1}\frac{n^2}{n^2+1} $? 
\vfill

\sonslide<2->{Not close at all: the series is divergent (which we can see by the divergence test).}
\end{frame}
%----------------------------------------------------------------------------------------
\begin{frame}<beamer>[t]{$\sum_{n=1}^{\infty}(-1)^{n-1} \frac{n^2}{n^2+1} $ diverges}
\StatusBar{1}{11}


\begin{tikzpicture}[scale=1]
	\weights{.5,-.8,.9,-0.9411764,0.9615384,-0.972972,0.98,-0.9846153,0.987804
}{
	\frac12,-\frac{4}{5},\frac{9}{10},-\frac{16}{17},\frac{25}{26},-\frac{36}{37},\frac{49}{50},-\frac{64}{65},\frac{81}{82}}{}
\onslide<11->{
	\begin{scope}[yshift=3.5cm,xshift=5mm]
	\myaxis{n}{0}{6.2}{S_n}{1}{1.5}
	\ycoord{1.2}{0.6}
	\ycoord{-0.6}{-0.3}
	\foreach \n in {1,...,12}{
		\xcoord{\n/2}{}
		}
	\foreach \y[count=\x] in {0.5,-0.3,0.6,-0.34,0.62,-0.35,0.63,-0.36,0.63,-0.36,0.63,-0.36
}{
		\draw (\x/2,2*\y)node[vertex]{};}
	\end{scope}
	}
\end{tikzpicture}

\end{frame}
%----------------------------------------------------------------------------------------
%----------------------------------------------------------------------------------------


%----------------------------------------------------------------------------------------
%----------------------------------------------------------------------------------------
\section{3.3.5 Ratio Test}
%----------------------------------------------------------------------------------------
\begin{frame}[t]
\StatusBar{1}{8}
Recall for a geometric series, the \alert{ratios of consecutive terms} is constant.
\vfill

\begin{center}\begin{tikzpicture}
\foreach \x in {1,2,3,4,5}{
	\POWER{2}{\x}{\n}
	\draw(\x,0)node{$\dfrac{1}{\n}$};
	}
\foreach \x in {1,2,3,4}{
	\draw(\x+.5,0)node{$+$};
	\onslide<\x->{
		\draw[->](\x,1) to[bend left](\x+.9,1);
		\draw(\x+.45,1.1)node[above]{\footnotesize$\times \frac12$};
		}
	\ADD{\x}{4}{\s}
	\onslide<\s->{
		\POWER{2}{\x}{\den}
		\MULTIPLY{\den}{2}{\num}
		\draw[decorate,decoration={brace,mirror,amplitude=5pt}](\x+0.05,-1)--(\x+0.95,-1)node[midway,below,yshift=-2mm]{$\frac{1/\num}{1/\den}=$};
		}
	}
\draw(5.5,0)node{$\cdots$};
\onslide<8->{\draw(5.25,-1.6)node{\alert{$\frac12$}};}
\end{tikzpicture}\end{center}

\vfill
If that ratio has magnitude \alert{less then one}, then the series converges.\\
If the ratio has magnitude \alert{greater than one}, the series diverges.
\end{frame}


%----------------------------------------------------------------------------------------
%----------------------------------------------------------------------------------------
%----------------------------------------------------------------------------------------
\begin{frame}[t]
\StatusBar{1}{6}

For series convergence, we are concerned with what happens to terms $a_n$ when $n$ is sufficiently large.

Suppose for a sequence $a_n$, $\alert{\lim\limits_{n \to \infty} \frac{a_{n+1}}{a_n}=L}$ for some constant $L$.\vfill

\begin{center}\begin{tikzpicture}
\draw(0.,0)node(a0){$a_n$};
\foreach \x in {1,2,3,4}{
	\POWER{2}{\x}{\n}
	\draw(\x*1.5,0)node(a\x){$a_{n+\x}$};
	}

\foreach \x in {0,1,2,3,4}{
	\draw(\x*1.5+.75,0)node{$+$};
	\ADD{\x}{1}{\nn}
	\onslide<\nn->{
		\MULTIPLY{\x}{1.5}{\a}
		\ADD{\a}{1.5}{\b}
		\ifnum \x > 0
		\draw[decorate,decoration={brace,mirror,amplitude=5pt}](\a+0.05,-.25)--(\b-0.05,-.25)node[midway,below,yshift=-2mm]{$\dfrac{a_{n+\nn}}{a_{n+\x}}~\approx$};
		\else
		\draw[decorate,decoration={brace,mirror,amplitude=5pt}](\a+0.05,-.25)--(\b-0.05,-.25)node[midway,below,yshift=-2mm]{$\dfrac{a_{n+1}}{a_{n}}~\approx$};
		\fi
		}
	}
\draw(7.25,0)node{$\cdots$};
\onslide<6->{\draw(7.75,-.9)node{\alert{$L$}};}
\end{tikzpicture}\end{center}
\vfill
Like in a geometric series:\\[1em]
If  $L$ has magnitude \alert{less then one}, then the series converges.\\
If  $L$ has magnitude \alert{greater than one}, the series diverges.

\end{frame}

%----------------------------------------------------------------------------------------
\begin{frame}[t]
\unote{Theorem~\eref{text}{thm:SRratio}}
\begin{block}{Ratio Test}
Let $N$ be any positive integer and assume that $a_n\ne 0$ for all $n\ge N$.
\begin{enumerate}[(a)]
\item If $\lim\limits_{n\rightarrow\infty}\Big|\frac{a_{n+1}}{a_n}\Big| = L<1$,
then $\sum\limits_{n=1}^\infty a_n$ converges.
\item If $\lim\limits_{n\rightarrow\infty}\Big|\frac{a_{n+1}}{a_n}\Big| = L>1$,
or $\lim\limits_{n\rightarrow\infty}\Big|\frac{a_{n+1}}{a_n}\Big| = \infty$,
then $\sum\limits_{n=1}^\infty a_n$ diverges.
\end{enumerate}
\end{block}
\end{frame}

%----------------------------------------------------------------------------------------
\begin{frame}[t]
\only<1>{\begin{block}{Ratio Test}
Let $N$ be any positive integer and assume that $a_n\ne 0$ for all $n\ge N$.
\begin{enumerate}[(a)]
\item If $\lim\limits_{n\rightarrow\infty}\Big|\frac{a_{n+1}}{a_n}\Big| = L<1$,
then $\sum\limits_{n=1}^\infty a_n$ converges.
\item If $\lim\limits_{n\rightarrow\infty}\Big|\frac{a_{n+1}}{a_n}\Big| = L>1$,
or $\lim\limits_{n\rightarrow\infty}\Big|\frac{a_{n+1}}{a_n}\Big| = \infty$,
then $\sum\limits_{n=1}^\infty a_n$ diverges.
\end{enumerate}
\end{block}
}
Use the ratio test to determine whether the series \[\sum_{n=1}^\infty \frac{n}{3^n}\] converges or diverges. 
\MoreSpace<1>
\AnswerYes<2>
\QuestionBar{1}{3}<1-2>
\AnswerBar{1}{3}<3->
\vfill
\sonslide<3->{
\begin{align*}
\left|\frac{a_{n+1}}{a_n}\right|&=\frac{\frac{n+1}{3^{n+1}}}{\frac{n}{3^n}}=\frac{n+1}{n}\cdot\frac{3^n}{3^{n+1}}=\left(1+\frac{1}{n}\right)\cdot\frac13\\
\lim_{n \to \infty}\left|\frac{a_{n+1}}{a_n}\right|&=\frac13
\end{align*}
Since $\frac13<1$, by the ratio test, $\ds\sum_{n=1}^\infty \frac{n}{3^n}$ coverges.
}
\end{frame}
%----------------------------------------------------------------------------------------
\begin{frame}[t]{Remark}
\StatusBar{1}{7}
The series we just considered, $\ds\sum_{n=1}^\infty \frac{n}{3^n}$, looks similar to a geometric series, but it is not exactly a geometric series. That's a good indicator that the ratio test will be helpful! 
\vfill\pause
We could have used other tests, but ratio was probably the easiest.
\begin{itemize}[<+->]
\item Integral test: \onslide<+->{$\ds\int \frac{x}{3^x} \ \dee x$ can be evaluated using integration by parts.}
\item Comparison test: 
\begin{itemize}
\item $\sum\frac{1}{3^n}$ is not a valid comparison series, nor is $\sum n$.
\item Because $n<2^n$ for all $n\ge 1$, the series $\sum\left(\frac23\right)^n$ will work.
\end{itemize}
\item  The divergence test is inconclusive, and the alternating series test does not apply. Our series is not geometric, and not obviously telescoping.
\end{itemize}\vfill

\end{frame}
%----------------------------------------------------------------------------------------
\begin{frame}<beamer>[t]
{$\sum_{n=1}^\infty \frac{n}{3^n}$ converges}
\begin{tikzpicture}

\weights{0.3333333333,
0.2222222222,
0.1111111111,
0.049382716,
0.0205761317,
0.0082304527,
0.0032007316,
0.0012193263,
}%sizes
{\frac{1}{3},\frac{2}{3^2},\frac{3}{3^3},\frac{4}{3^4},\frac{5}{3^5},\frac{6}{3^6}
,\frac{7}{3^7}
,\frac{8}{3^8}}%labels
{}%partial sums

\end{tikzpicture}

\end{frame}
%----------------------------------------------------------------------------------------
%----------------------------------------------------------------------------------------

%----------------------------------------------------------------------------------------

\begin{frame}[t]
\begin{block}{Ratio Test}
Let $N$ be any positive integer and assume that $a_n\ne 0$ for all $n\ge N$.
\begin{enumerate}[(a)]
\item If $\lim\limits_{n\rightarrow\infty}\Big|\frac{a_{n+1}}{a_n}\Big| = L<1$,
then $\sum\limits_{n=1}^\infty a_n$ converges.
\item If $\lim\limits_{n\rightarrow\infty}\Big|\frac{a_{n+1}}{a_n}\Big| = L>1$,
or $\lim\limits_{n\rightarrow\infty}\Big|\frac{a_{n+1}}{a_n}\Big| = \infty$,
then $\sum\limits_{n=1}^\infty a_n$ diverges.
\end{enumerate}
\end{block}
Let $a$ and $x$ be nonzero constants.
Use the ratio test to determine whether
\[\sum_{n=1}^\infty anx^{n-1}\]
converges or diverges. (This may depend on the values of $a$ and $x$.)
\MoreSpace
\QuestionBar{2}{3}
\end{frame}
%----------------------------------------------------------------------------------------
\begin{frame}<beamer>[t]
\AnswerYes<1>\QuestionBar<1>{2}{3}
\AnswerBar<2>{2}{3}
\[\sum_{n=1}^\infty anx^{n-1}\]

\sonslide<2->{
\begin{align*}
\left|\frac{a_{n+1}}{a_n}\right|&=\left|\frac{a(n+1)x^n}{anx^{n-1}}\right|=\left|\left(\frac{n+1}{n}\right)x\right|=\left(1+\frac1n\right)|x|\\
\lim_{n \to \infty}\left|\frac{a_{n+1}}{a_n}\right|&=|x|
\end{align*}
So the series converges when $|x|<1$ and diverges when $|x|>1$.
For the cases $x=\pm1$, the ratio test is inconclusive, so we'll need another test. Fortunately, the divergence test makes things quick.

\begin{align*}
\text{For }x&=1: & \lim_{n \to \infty}an(1)^{n-1}&=\lim_{n \to \infty} an  \neq 0\\
\text{For }x&=-1: & \lim_{n \to \infty}an(-1)^{n-1}&  \neq 0
\end{align*}

All together, for any nonzero $a$, the series diverges when $|x| \ge 1$ and converges when $|x|<1$.}
\end{frame}

%----------------------------------------------------------------------------------------

\begin{frame}[t]
\unote{Example~\eref{text}{eg:ratioC}}
\AnswerYes<1>
\QuestionBar{3}{3}<1>
\AnswerBar{3}{3}<2->
Let $x$ be a constant. Use the ratio test to determine whether
\[\sum_{n=1}^\infty \frac{(-3)^n\sqrt{n+1}}{2n+3}x^{n}\]
converges or diverges. (This may depend on the value of $x$.)\color{C1}

\sonly<2>{
\begin{align*}
\left|\frac{a_{n+1}}{a_n}\right|&=\left|
\frac{ \frac{(-3)^{n+1}\sqrt{n+2}}{2(n+1)+3}x^{n+1}}{ \frac{(-3)^n\sqrt{n+1}}{2n+3}x^{n}}\right|
=
\left|\frac{(-3)^{n+1}}{(-3)^n}\cdot\frac{\sqrt{n+2}}{\sqrt{n+1}}\cdot\frac{2n+3}{2n+5}\cdot\frac{x^{n+1}}{x^n}\right|\\
&=3\cdot\sqrt{\frac{n+2}{n+1}}\cdot\left(\frac{2n+3}{2n+5}\right)\cdot |x|
=
3\sqrt{\frac{1+2/n}{1+1/n}}\cdot\left(\frac{2+3/n}{2+5/n}\right)\cdot |x|
\\
\lim_{n \to \infty}\left|\frac{a_{n+1}}{a_n}\right|&=3\sqrt{\frac11}\left(\frac{2}{2}\right)|x|=3|x|
\end{align*}

So the series converges when $3|x|<1$ and diverges when $3|x|>1$. 

So for $|x|<\frac13$, the series converges, and for $|x|>\frac13$, it diverges.
}
\end{frame}
%------------------------------------
\iftoggle{spoiler}{\begin{frame}<beamer>[t]
\AnswerYes\color{spoilercolor}
Consider $x=\frac13$.
\begin{align*}
\sum_{n=1}^\infty \frac{(-3)^n\sqrt{n+1}}{2n+3}x^{n}&=
\sum_{n=1}^\infty \frac{\sqrt{n+1}}{2n+3}\frac{(-3)^n}{3^{n}}=
\sum_{n=1}^\infty (-1)^n\frac{\sqrt{n+1}}{2n+3}
\end{align*}
This is an alternating series. Let's use the alternating series test.
\begin{enumerate}[(i)]\color{spoilercolor}
\item $a_n=\frac{\sqrt{n+1}}{2n+3} \ge 0$ for all $n \ge 1$,
\item To show that $a_n$ is monotonically decreasing, consider the derivative of $f(t)=\frac{\sqrt{t+1}}{2t+3}$:
\begin{align*}
f'(t)&=\frac{(2t+3)\frac{1}{2\sqrt{t+1}}-\sqrt{t+1}(2)}{(2t+3)^2}\left(\textcolor{W1}{\frac{\sqrt{t+1}}{\sqrt{t+1}}}\right)\\
&=\frac{\left(t+\frac32\right)-(t+1)(2)}{(2t+3)^2\sqrt{t+1}}=\frac{-t-\frac12}{(2t+3)^2\sqrt{t+1}}
\end{align*}
Since $f'(t)<0$ for all $t >0$, we see it is a decreasing function on that domain, so $a_{n+1} < a_n$ for all $n \ge 1$.
\item $\lim\limits_{n \to \infty}a_n = 0$
\end{enumerate}
So, our series converges by the alternating series test when $x=\frac13$.

\end{frame}
%----------
\begin{frame}<beamer>[t]
\AnswerBar{3}{3}
\color{spoilercolor}

Finally, consider $x=-\frac13$.
\sonslide{\begin{align*}
\sum_{n=1}^\infty \frac{(-3)^n\sqrt{n+1}}{2n+3}x^{n}&=
\sum_{n=1}^\infty \frac{\sqrt{n+1}}{2n+3}\frac{(-3)^n}{(-3)^{n}}=
\sum_{n=1}^\infty \frac{\sqrt{n+1}}{2n+3}
\end{align*}
We will use the limit comparison test, with comparison series
 $\textcolor{W1}{\sum\limits_{n=1}^\infty \frac{1}{\sqrt n}}$.
\begin{align*}
\lim_{n \to \infty}\frac{\frac{\sqrt{n+1}}{2n+3}}{\textcolor{W1}{\frac{1}{\sqrt n}}}&=
\lim_{n \to \infty}\frac{\sqrt{n+1}\sqrt{n}}{2n+3}=
\lim_{n \to \infty}\frac{\sqrt{n^2+n}}{2n+3}\left(\textcolor{C3}{\frac{1/n}{1/n}}\right)\\
&=\lim_{n \to \infty}\frac{\sqrt{1+1/n}}{2+3/n}=\frac{\sqrt{1+0}}{2+0}=\frac12
\end{align*}
Since $\frac12$ is a nonzero constant, and since $\sum\frac{1}{\sqrt n}$ diverges (by the $p$-test), our series diverges as well.\\
\color{C2}
All together, the original series converges when $-\frac13<x \le \frac13$, and diverges otherwise.
}
\end{frame}}{}

%----------------------------------------------------------------------------------------
%----------------------------------------------------------------------------------------
\section{3.3.6 List of Tests}
%----------------------------------------------------------------------------------------

%----------------------------------------------------------------------------------------
\begin{frame}[t]{Fill in in the blanks}
\AnswerYes<1-3>
\begin{block}{Divergence Test}
If the sequence $\left\{ a_n\right\}_{n=c}^\infty$ \only<2-|handout:0>{fails to converge to zero as $n \to \infty$,}\only<1>{\boxed{\phantom{A}\hspace{5cm}}}\\
 then the series $\sum\limits_{n=c}^\infty a_n$ diverges.
\end{block}\vfill

\begin{block}{Ratio Test}
Let $N$ be any positive integer and assume that $a_n\ne 0$ for all $n\ge N$.
\begin{enumerate}[(a)]
\item If $\lim\limits_{n\rightarrow\infty}\Big|\frac{a_{n+1}}{a_n}\Big| ~\only<-2>{\boxed{\vphantom{A}\hspace{1cm}}} \only<3-|handout:0>{=L<1}$,
then $\sum\limits_{n=1}^\infty a_n$ converges.
\item If $\lim\limits_{n\rightarrow\infty}\Big|\frac{a_{n+1}}{a_n}\Big| 
~\only<-3>{\boxed{\vphantom{A}\hspace{1cm}}} \only<4-|handout:0>{=L>1}$,
or $\lim\limits_{n\rightarrow\infty}\Big|\frac{a_{n+1}}{a_n}\Big| = \infty$,
then $\sum\limits_{n=1}^\infty a_n$ diverges.
\end{enumerate}
\end{block}

\end{frame}
%----------------------------------------------------------------------------------------
\begin{frame}[t]
\only<1-2>{\AnswerYes}
\begin{block}{Integral Test}
Let $N_0$ be any natural number. If $f(x)$ is a function which is defined
and continuous for all $x\ge N_0$ and which obeys
\begin{enumerate}[(i)]
\item \only<1>{\boxed{\vphantom{A}\hspace{3cm}}}\only<2-|handout:0>{$f(x)\ge 0$ for all $x\ge N_0$} and
\item  \only<-2>{\boxed{\vphantom{A}\hspace{3cm}}}\only<3-|handout:0>{$f(x)$ decreases as $x$ increases} and
\item $f(n)=a_n$ for all  $n\ge N_0$.
\end{enumerate}

Then
\hfill\smash{\begin{tikzpicture}[yscale=0.75]
\myaxis{x}{0}{3.25}{y}{0}{1.25}
\xcoord{1}{1}
\xcoord{2}{2}
\xcoord{3}{3}
\draw (1,1)node[vertex,label=above:{$a_1$}]{};
\draw (2,.5)node[vertex,label=above:{$a_2$}]{};
\draw (3,.33)node[vertex,label=above:{$a_3$}]{};
\draw[C1] plot[domain=.8:3.5](\x,{1/\x})node[right]{$y=f(x)$};
\end{tikzpicture}}\begin{equation*}
\sum_{n=1}^\infty a_n\text{ converges }\iff
\int_{N_0}^\infty f(x)\ \dee{x}\text{ converges}
\end{equation*}
Furthermore, when the series converges, the truncation error satisfies
\begin{equation*}
0 \le \sum_{n=1}^\infty a_n-\sum_{n=1}^N a_n\le
  \int_N^\infty f(x)\ \dee{x}\qquad\text{for all $N\ge N_0$}
\end{equation*}

\end{block}
\end{frame}
%----------------------------------------------------------------------------------------
\begin{frame}[t]{Fill in in the blanks}
\AnswerSpace
\only<1-2>{\AnswerYes}
\begin{block}{The Comparison Test}
Let $N_0$ be a natural number and let $K>0$.
\begin{enumerate}[(a)]
\item If $|a_n| \only<1>{~\boxed{\phantom V}~}\only<2-|handout:0>{\leq} K c_n$ for all $n \ge N_0$ and $\sum\limits_{n=0}^\infty c_n$ converges, then $\sum\limits_{n=0}^\infty a_n$ converges.
\item If $a_n \only<-2>{~\boxed{\phantom V}~}\only<3-|handout:0>{\ge} Kd_n \ge 0$ for all $n \ge N_0$ and $\sum\limits_{n=0}^\infty d_n$ diverges, then $\sum\limits_{n=0}^\infty a_n$ diverges.
\end{enumerate}
\end{block}
\end{frame}
%----------------------------------------------------------------------------------------
\begin{frame}[t]{Fill in in the blanks}
\AnswerSpace
\only<1>{\AnswerYes}
\begin{block}{Limit Comparison Theorem}
Let $\sum_{n=1}^\infty a_n$ and $\sum_{n=1}^\infty b_n$ be two series with
$b_n>0$ for all $n$. Assume that
\begin{equation*}
\lim_{n\rightarrow\infty}\frac{a_n}{b_n}=L
\end{equation*}
exists.
\begin{enumerate}[(a)]
\item If $\sum_{n=1}^\infty b_n$  converges, then
$\sum_{n=1}^\infty a_n$ converges too.

\item If $L\ne 0$ and $\sum_{n=1}^\infty b_n$  diverges,
then $\sum_{n=1}^\infty a_n$ diverges too.
\end{enumerate}
In particular, if \only<1>{\boxed{\phantom A \hspace{1cm}}}\only<2|handout:0>{$L\ne 0$}, then $\sum_{n=1}^\infty a_n$ converges
if and only if $\sum_{n=1}^\infty b_n$ converges.
\end{block}
\end{frame}
%----------------------------------------------------------------------------------------
\begin{frame}[t]
\AnswerSpace
\only<1-2>{\AnswerYes}
\begin{block}{Alternating Series Test}
Let $\big\{a_n\big\}_{n=1}^\infty$
be a sequence of real numbers that obeys
\begin{enumerate}[(i)]
\item \only<1>{\boxed{\phantom V\hspace{4cm}}}\only<2-|handout:0>{$a_n\ge 0$ for all $n\ge 1$;}
\item $a_{n+1}\le a_n$  for all $n\ge 1$ (i.e. the
sequence is monotone decreasing);
\item  and \only<-2>{\boxed{\phantom V\hspace{4cm}}}\only<3-|handout:0>{$\lim\limits_{n\rightarrow\infty}a_n=0$.}
\end{enumerate}
Then
\begin{equation*}
a_1-a_2+a_3-a_4+\cdots=\sum\limits_{n=1}^\infty (-1)^{n-1} a_n =S
\end{equation*}
converges and, for each natural number $N$,  $S-S_N$ is between $0$ and (the
first dropped term) $(-1)^N a_{N+1}$. Here $S_N$ is, as previously,
the $N^{\rm th}$ partial sum $\sum\limits_{n=1}^N (-1)^{n-1} a_n$.
\end{block}

\end{frame}
%----------------------------------------------------------------------------------------
%----------------------------------------------------------------------------------------
\begin{frame}[t]{List of convergence tests}
\StatusBar{1}{5}
\begin{Ldescription}[<+->]
\item[Divergence Test]$ $\\
When the $n^{\mathrm{th}}$ term in the series \textit{fails}
to converge to zero as $n$ tends to infinity.

This is a good first thing to check: if it works, it's quick, but it doesn't always work.
%%%
\vfill
\item[Alternating Series Test]$ $\\
\begin{itemize}
\item
successive terms in the series alternate
in sign
\item<.->
don't forget to check that successive terms decrease in magnitude
and tend to zero as $n$  tends to infinity
\end{itemize}
%%%
\vfill
\item[Integral Test]$ $\\
\begin{itemize}
\item
works well when, if you substitute $x$ for $n$ in the $n^{\mathrm{th}}$ term
you get a function, $f(x)$, that you can easily integrate
\item<.->
don't forget to check that $f(x)\ge 0$ and that $f(x)$ decreases
as $x$ increases
\end{itemize}
\end{Ldescription}
\end{frame}
%%%----------------------------------------------------------
\begin{frame}[t]{List of convergence tests}
\StatusBar{1}{4}
\begin{Ldescription}[<+->]
\item[Ratio Test]$ $\\
\begin{itemize}
\item
works well when $\frac{a_{n+1}}{a_n}$ simplifies enough that
you can easily compute
$\lim\limits_{n\rightarrow\infty}\big|\frac{a_{n+1}}{a_n}\big|=L$
\item<.->
this often happens when $a_n$ contains powers, like $7^n$,
or factorials, like $n!$
\item<.->
don't forget that $L=1$ tells you nothing about the convergence/divergence
of the series
\end{itemize}

%%%
\vfill
\item[Comparison Test and Limit Comparison Test]$ $\\
\begin{itemize}
\item Comparison test lets you ignore pieces of a function that feel extraneous (like replacing $n^2+1$ with $n^2$) \textit{but} there is a test to make sure the comparison is still valid. Either the limit of a ratio is the right thing, or an inequality goes the right way.
\item<.->
Limit comparison works well when, for very large $n$, the $n^{\mathrm{th}}$ term
$a_n$ is approximately the same as a simpler, nonnegative term $b_n$

\end{itemize}

\end{Ldescription}
\end{frame}

%----------------------------------------------------------------------------------------
\begin{frame}[t]
\StatusBar{1}{3}
\begin{itemize}[<+->]
\item The integral test gave us the $p$-test. When you're looking for comparison series, $p$-series $\ds\sum \frac{1}{n^p}$ are often good choices, because their convergence or divergence is so easy to ascertain.\vfill

\item Geometric series have the form $\ds\sum a \cdot r^n$ for some nonzero constants $a$ and $r$. The magnitude of $r$ is all you need to know to deicide whether they converge or diverge, so these are also common comparison series.
\vfill

\item Telescoping series have partial sums that are easy to find because successive terms cancel out. These are less obvious, and are less common choices for comparison series.
\end{itemize}
\end{frame}
%----------------------------------------------------------------------------------------

\begin{frame}[t]
\QuestionBar{1}{3}<1>
\AnswerYes<1-2>
\AnswerBar<2->{1}{3}
\only<1>{\begin{block}{Test List}
\begin{multicols}{2}
\begin{itemize}
\item divergence
\item integral
\item alternating series
\columnbreak

\item ratio
\item comparison 
\item limit comparison
\end{itemize}
\end{multicols}
\end{block}}
Determine whether the series $\ds\sum\limits_{n=1}^\infty \frac{\cos n}{2^n}$ converges or diverges.\vfill


\sonly<2>{\color{W1}
The \textbf{divergence test} is inconclusive, because $\lim\limits_{n \to \infty}\frac{\cos n}{2^n}=0$ (which you can show with the squeeze theorem).
\vfill

The \textbf{integral test} doesn't apply, because $f(x)=\frac{\cos x}{2^x}$ is not always positive (and not decreasing).\vfill

The \textbf{alternating series test} doesn't apply because the signs of the series do not strictly alternate every term.\vfill

The \textbf{ratio test} does not apply, because $\lim\limits_{n \to \infty}\frac{a_{n+1}}{a_n}$ does not exist.
}

\sonly<3>{\color{C1}
\textbf{Comparison test:} Let $a_n=\frac{\cos n}{2^n}$. Note $|a_n| \leq \frac{1}{2^n}$, and $\sum_{n=1}^\infty \frac1{2^n}$ converges (it is a geometric sum with ratio of consecutive terms $\frac12$). So by the comparison test, $\sum\limits_{n=1}^\infty \frac{\cos n}{2^n}$ converges.
\vfill

\textbf{Limit comparison:} Set $a_n=\frac{\cos n}{2^n}$ and $b_n=\left(\frac{2}{3}\right)^n$. Then
\begin{align*}
\frac{a_n}{b_n}&=\frac{\frac{\cos n}{2^n}}{\frac{2^n}{3^n}}=\left(\frac{3}{4}\right)^n\cos n
\\
-\left(\tfrac{3}{4}\right)^n &\leq \left(\tfrac{3}{4}\right)^n \cos n
 \leq \left(\tfrac{3}{4}\right)^n , \text{ and } \lim_{n \to \infty}-\left(\tfrac{3}{4}\right)^n =\lim_{n \to \infty}\left(\tfrac{3}{4}\right)^n =0
 \intertext{So, by the Squeeze Theorem,}
\lim_{n \to \infty}\frac{a_n}{b_n}&=0
\end{align*}
Since $\sum_{n=1}^\infty b_n$  converges, by the limit comparison theorem, $\sum\limits_{n=1}^\infty \frac{\cos n}{2^n}$ converges as well.
}
\end{frame}
%---------------------------------------------------------------------------------------

%----------------------------------------------------------------------------------------

\begin{frame}[t]
\QuestionBar<1>{2}{3}
\AnswerYes<1-4>
\AnswerBar<2->{2}{3}
\sStatusBar{1}{5}
\only<1>{\begin{block}{Test List}
\begin{multicols}{2}
\begin{itemize}
\item divergence
\item integral
\item alternating series
\columnbreak

\item ratio
\item comparison 
\item limit comparison
\end{itemize}
\end{multicols}
\end{block}}
Determine whether the series $\ds\sum\limits_{n=1}^\infty \frac{2^n\cdot n^2}{(n+5)^5}$ converges or diverges.\vfill

\color{C1}
\sonly<2>{
\vfill
\textcolor{W1}{The \textbf{alternating series test} doesn't apply because the signs of the series do not  alternate.}\vfill
\textcolor{W1}{The \textbf{integral test} doesn't apply $f(x)=\frac{2^x\cdot x^2}{(x+5)^5}$ is not a decreasing function.}\vfill
\textbf{Divergence test:} $\lim\limits_{n \to \infty}\frac{2^n\cdot n^2}{(n+5)^5}=\infty$ (which you can see because the numerator is larger than a power function; the denominator is a polynomial; and power functions grow faster than polynomials), so the series diverges by the divergence test.\\[1em]
\textit{This is the fastest option, but not the only one.}
\vfill
}

\sonly<3>{
\textbf{Ratio test:}
\begin{align*}
\frac{a_n}{b_n}&=\frac{\frac{2^{n+1}\cdot (n+1)^2}{(n+1+5)^5}}{\frac{2^n\cdot n^2}{(n+5)^5}}
=
\frac{2^{n+1}}{2^n}\cdot\frac{(n+1)^2}{n^2}\cdot\frac{(n+5)^5}{(n+6) ^5}\\
&=2\left(1+\frac1n\right)^2\left(1-\frac{1}{n+6}\right)^5\\
\lim_{n \to \infty}\frac{a_n}{b_n}&=2(1)^2(1)^5=2
\end{align*}
So, the limit of the ratio of consecutive terms is greater than 1. Therefore $\sum\limits_{n=1}^\infty \frac{2^n\cdot n^2}{(n+5)^5}$ diverges by the ratio test.
}

\sonly<4>{
\textbf{Comparison test:} 
Since power functions grow faster than polynomials, for large values of $n$, $2^n>(n+5)^5$, so $\frac{2^n}{(n+5)^5}>1$. Then, for large enough $n$, 
\[\frac{2^n\cdot n^2}{(n+5)^5}>n^2~.\]
By the divergence test, $\sum_{n=1}^\infty n^2$ diverges. So by the comparison test,  $\sum\limits_{n=1}^\infty \frac{2^n\cdot n^2}{(n+5)^5}$ diverges as well.\vfill
}

\sonly<5>{
\textbf{Limit comparison:} Set $a_n= \frac{2^n\cdot n^2}{(n+5)^5}$ and $b_n=\frac{2^n}{n^3}$. 


Then
\begin{align*}
\frac{a_n}{b_n}&=\frac{ \frac{2^n\cdot n^2}{(n+5)^5}}{\frac{2^n}{n^3}}=
\frac{n^5}{(n+5)^5}=\left(1-\frac{5}{n+5}\right)^5\\
\text{So, }\lim_{n \to \infty}\frac{a_n}{b_n}&=1^5=1
\end{align*}
Note that $\sum_{n=1}^\infty \frac{2^n}{n^3}$ diverges. (You can show this using the tests we've already used on the original series, so this method isn't really an improvement.) Since 
$\lim\limits_{n \to \infty}\frac{a_n}{b_n}$ exists and is nonzero, by the limit comparison theorem,
$\sum\limits_{n=1}^\infty \frac{2^n\cdot n^2}{(n+5)^5}$ diverges.
}
\end{frame}
%---------------------------------------------------------------------------------------

%----------------------------------------------------------------------------------------

\begin{frame}[t]
\sStatusBar{1}{5}
\QuestionBar<1>{3}{3}
\AnswerYes<1-4>
\AnswerBar<2->{3}{3}
\only<1>{\begin{block}{Test List}
\begin{multicols}{2}
\begin{itemize}
\item divergence
\item integral
\item alternating series
\columnbreak

\item ratio
\item comparison 
\item limit comparison
\end{itemize}
\end{multicols}
\end{block}}
Determine whether the series $\ds\sum\limits_{n=1}^\infty \frac1n\sin\left(\frac1n\right)$ converges or diverges.\\[1em]
\textit{Hint:} If $\theta \ge 0$ then $\sin \theta \leq \theta$.
\vfill

\color{C1}
\sonly<2>{
\color{W1}
\vfill
The \textbf{divergence test} is inconclusive because $\lim\limits_{n \to \infty}\frac{\sin\left(\frac1n\right)}{n}=0$.\vfill
The \textbf{alternating series test} does not apply because we are not considering an alternating series.\vfill
The \textbf{integral test} won't work for us because $\int_1^\infty \frac{1}{x}\sin\left(\frac1x\right)\ \dee x$ cannot be evaluated with techniques we've learned in class so far.
\vfill
}

\sonly<3>{
\color{W1}
The 
\textbf{ratio test} is inconclusive because $\lim\limits_{n \to \infty} \frac{\frac{1}{n+1}\sin\left(\frac1{n+1}\right)}{\frac{1}{n}\sin\left(\frac1{n}\right)}=1$:\small

Set $x=\frac{1}{n+1}$. 
 Then $\frac{1}{n}=\frac{x}{1-x}$:
\begin{align*}
\lim_{n \to \infty}\frac{\sin\left(\frac{1}{n+1}\right)}{\frac1n}&=
\lim_{x \to 0^+}\frac{\sin x}{\frac{x}{1-x}}=
\lim_{x \to 0^+}(1-x)\frac{\sin x}{x}=1\cdot 1=1
\intertext{Set $y=\frac{1}{n}$. Then $\frac{1}{n+1}=\frac{y}{1+y}$:}
\lim_{n \to \infty}\frac{\sin\left(\frac{1}{n}\right)}{\frac1{n+1}}&=
\lim_{y \to 0^+}\frac{\sin y}{\frac{y}{1+y}}=
\lim_{y \to 0^+}(1+y)\frac{\sin y}{y}=1\cdot 1=1
\intertext{Therefore,}
\lim\limits_{n \to \infty} \frac{\frac{1}{n+1}\sin\left(\frac1{n+1}\right)}{\frac{1}{n}\sin\left(\frac1{n}\right)}&=1
\end{align*}
}

\sonly<4>{
\textbf{Comparison test:} 
For $n \ge 1$, $\frac{1}{n} >0$. Then setting  $\theta = \frac1n$ in the hint, $\sin\left(\frac1n\right) \le \frac1n$.  Furthermore, $0<\frac{1}{n}<\pi$, so $\sin\left(\frac1n\right)>0$.
\[0<\frac1n\sin\left(\frac1n\right) \le \frac1n\left(\frac1n\right)=\frac{1}{n^2}\]
The $p$-series $\sum_{n=1}^\infty\frac1{n^2}$ converges, so by the comparison test, $\sum_{n=1}^\infty \frac1n\sin\left(\frac1n\right)$ converges as well.
\vfill}


\sonly<5>{
\textbf{Limit comparison:} Set $a_n=\frac1n\sin\left(\frac1n\right)$ and $b_n=\frac{1}{n^2}$.

\begin{align*}
\lim_{n \to \infty}\frac{a_n}{b_n}&=\lim_{n \to \infty}
\frac{\frac1n\sin\left(\frac1n\right)}{\frac{1}{n^2}}
=
\lim_{n \to \infty}\frac{\sin\left(\frac1n\right)}{\frac{1}{n}}
\intertext{Setting $x=\frac1n$,}
&=\lim_{x \to 0^+}\frac{\sin x}{x}=1
\end{align*}

The $p$-series $\sum_{n=1}^\infty\frac1{n^2}$ converges, so by the limit comparison test, $\sum_{n=1}^\infty \frac1n\sin\left(\frac1n\right)$ converges as well.
}


\end{frame}
%---------------------------------------------------------------------------------------


%----------------------------------------------------------------------------------------
