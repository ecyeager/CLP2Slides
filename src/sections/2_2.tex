% Copyright 2021 Joel Feldman, Andrew Rechnitzer and Elyse Yeager, except where noted.
% This work is licensed under a Creative Commons Attribution-NonCommercial-ShareAlike 4.0 International License.
% https://creativecommons.org/licenses/by-nc-sa/4.0/


 \begin{frame}{Table of Contents }
\mapofcontentsB{\bb}
 \end{frame}
%----------------------------------------------------------------------------------------
%----------------------------------------------------------------------------------------
\section{2.2 Averages}
%----------------------------------------------------------------------------------------
%----------------------------------------------------------------------------------------
\begin{frame}[t]
\StatusBar{1}{6}
\begin{center}
\begin{tikzpicture}[xscale=1.5]
\myaxis{x}{0}{5}{y}{0}{3}
\draw[thick] plot[smooth,domain=0:5](\x,{\x/2+sin(\x r)+1});

\onslide<2|handout:0>{\foreach \n in {1,...,5}{
	\SUBTRACT{\n}{.5}{\x}
	\SIN{\x}{\sx}
	\DIVIDE{\x}{2}{\hx}
	\ADD{\hx}{\sx}{\xa}
	\ADD{\xa}{1}{\y}
	\draw(\x,\y)node[vertex]{};
	\xcoord{\x}{x_{\n}}
	}}
\onslide<3|handout:0>{\foreach \n in {1,...,15}{
	\DIVIDE{\n}{3}{\xx}
	\SUBTRACT{\xx}{.15}{\x}
	\SIN{\x}{\sx}
	\DIVIDE{\x}{2}{\hx}
	\ADD{\hx}{\sx}{\xa}
	\ADD{\xa}{1}{\y}
	\draw(\x,\y)node[vertex]{};
	\xcoord{\x}{x_{\n}}
	}}
\onslide<4->{\foreach \n in {1,...,30}{
	\DIVIDE{\n}{6}{\xx}
	\SUBTRACT{\xx}{.075}{\x}
	\SIN{\x}{\sx}
	\DIVIDE{\x}{2}{\hx}
	\ADD{\hx}{\sx}{\xa}
	\ADD{\xa}{1}{\y}
	\draw(\x,\y)node[vertex]{};
	\xcoord{\x}{}
	}
	\xcoord{0.09}{x_1}
	\xcoord{4.925}{x_n}}
\end{tikzpicture}
\end{center}\color{spoilercolor}
\only<2|handout:0>{\[\text{Average}\approx\frac{f(x_1)+\cdots +f(x_5)}{5}\]}
\only<3|handout:0>{\[\text{Average}\approx\frac{f(x_1)+\cdots +f(x_{15})}{15}\]}
\color{black}
\only<4->{\begin{align*}
	\text{Average}&\approx\frac{f(x_1)+\cdots +f(x_{n})}{n}\\
	\onslide<5->{\text{Average}&=\lim_{n \to \infty}\left[\frac{1}{n}\sum_{i=1}^nf(x_i)\right]}
	\onslide<6->{=\lim_{n \to \infty}\left[\frac{(b-a)}{(b-a)n}\sum_{i=1}^nf(x_i)\right]
	\\&=\lim_{n \to \infty}\left[\frac{1}{b-a}\sum_{i=1}^n f(x_i) \Delta x\right]
	=\frac{1}{b-a}\int_a^bf(x)\dee x}
	\end{align*}
	}
\end{frame}
%----------------------------------------------------------------------------------------
\begin{frame}[t]
\MoreSpace<2>
\QuestionBar<2-3>{1}{2}
\AnswerBar<4->{1}{2}
\AnswerYes<3>
\only<-2>{\begin{block}{Average}
Let $f(x)$ be an integrable function defined on the interval
$a\le x\le b$. The average value of $f$ on that interval is
\begin{align*}
f_{\text{ave}}
=\frac{1}{b-a}\int_a^b f(x)\ \dee{x}
\end{align*}
\end{block}}\pause
The temperature in a certain city at time $t$ (measured in hours past midnight) is given by
\[T(t)=t-\frac{t^2}{30}\]
What was the average temperature of one day (from $t=0$ to $t=24$)? 

\sonslide<4->{
	\begin{align*}
	\text{Average}&=\frac{1}{24}\int_0^{24}\left[t-\frac{t^2}{30}\right] \dee t\\
	&=
	\frac1{24}\left[ \frac{t^2}{2}-\frac{t^3}{90} \right]_0^{24}\\
	&=\frac{1}{24}\left[ \frac{24^2}{2}-\frac{24^3}{90} \right]\\
	&=\frac{28}{5}=5.6
	\end{align*}
	}
\unote{Definition~\eref{text}{def:AVaverage}}
\end{frame}
%----------------------------------------------------------------------------------------
\begin{frame}[t]
\AnswerBar{1}{2}
Let's check that our answer makes some intuitive sense.
\begin{center}
\begin{tikzpicture}
\myaxis{t}{0}{8.2}{y}{0}{4.2}
\draw[dotted] (0,0) grid (8,4);
\draw[thick] plot[domain=0:24,xscale=1/3,smooth,yscale=1/2](\x,{\x-\x*\x/30});
\ycoord{4}{8}
\ycoord{3}{6}
\ycoord{2}{4}
\ycoord{1}{2}
\foreach\x in {1,...,8}{\MULTIPLY{\x}{3}{\t} \xcoord{\x}{\t}}
\draw[thick,C3](0,2.8)--(8,2.8)node[right]{\only<beamer>{5.6}};
\end{tikzpicture}
\end{center}
\pause
Since the temperature is always between 0 and 8, we expect the average to be between 0 and 8
\end{frame}
%----------------------------------------------------------------------------------------
\begin{frame}[t]
\AnswerBar{1}{2}
\StatusBar{1}{3}
Let's also recall the motivation for our definition
\begin{center}
\begin{tikzpicture}
\myaxis{t}{0}{8.2}{y}{0}{4.2}
\draw[dotted] (0,0) grid (8,4);
\draw[thick] plot[domain=0:24,xscale=1/3,smooth,yscale=1/2](\x,{\x-\x*\x/30});
\ycoord{4}{8}
\ycoord{3}{6}
\ycoord{2}{4}
\ycoord{1}{2}
\foreach\x[evaluate=\t using int(3*\x)] in {1,...,8}{ \xcoord{\x}{\t}}

\foreach \n[evaluate=\average using 1] in {8,24,1440}{
	\onslide<+|handout:0>{
		\ifnum \n<100
		\foreach \i[	
			evaluate = \t using 24/\n*\i,
			evaluate = \y using \t-\t*\t/30
			] in {1,...,\n}{
			\draw (\t/3,\y/2)node[vertex]{};}%draw vertices
		\else
			\draw[ultra thick,densely dotted] plot[domain=0:24,xscale=1/3,smooth,yscale=1/2](\x,{\x-\x*\x/30});
		\fi
		%show average
		\ifnum \n=8\COPY{5.85}{\average}\fi
		\ifnum \n=24\COPY{5.694}{\average}\fi
		\ifnum \n=1440\COPY{5.602}{\average}\fi
		\draw[thick,C3](0,\average/2)--(8,\average/2)node[right]{\average};
		}
	}


\end{tikzpicture}
\end{center}
\only<1|handout:0>{Average temperature, measured every three hours }
\only<2|handout:0>{Average temperature, measured every hour}
\only<3|handout:0>{Average temperature, measured every minute}
\end{frame}
%----------------------------------------------------------------------------------------
\begin{frame}[t]
\QuestionBar{2}{2}<1>
\AnswerYes<1>
\AnswerBar<2>{2}{2}
Find the average value of the function $f(x) = \dfrac{e^x}{e^{2x}+1}$ over the interval $\left[0,\frac{\log 3}{2}\right]$.

\begin{multicols}{2}
\begin{tikzpicture}[scale=2]
\myaxis{x}{0}{1.2}{y}{0}{.75}
\draw(.55,0.1)--(.55,-.1)node[below]{$\frac{\log3}{2}$};
\draw[thick,C1] plot[domain=0:1](\x,{exp(\x)/(exp(2*\x)+1)});
\end{tikzpicture}
\columnbreak

\sonslide<2->{
Let $u(x)=e^x$. Then $u(0)=1$ and $u\left(\frac{\log 3}{2}\right)=e^{\frac{\log 3}{2}}=3^{1/2}=\sqrt3$.
\begin{align*}
&\frac{1}{\frac{\log 3}{2}-0}\cdot \int_0^{\frac{\log 3}{2}}\frac{e^x}{e^{2x}+1}\ \dee x\\ &=\frac{2}{\log 3}\int_{1}^{\sqrt 3}\frac{1}{u^2+1}\dee u\\
&=\frac{2}{\log 3}\left[\arctan(\sqrt 3)-\arctan(1)\right]\\
&=\frac{2}{\log 3}\left[\frac{\pi}{3}-\frac{\pi}{4}\right]=\frac{\pi}{6\log 3}\approx 0.477
\end{align*}
}
\end{multicols}
\end{frame}
%----------------------------------------------------------------------------------------
\begin{frame}[t]{Average Velocity}
\AnswerYes<1>

Let $x(t)$ be the position at time $t$ of a car moving along the $x$-axis.
The velocity of the car at time $t$ is the derivative $v(t)=x'(t)$.
The average velocity of the car over the time interval $a\le t\le b$ is:
\sonslide<2->{
\[v_{\text{ave}} = \frac{1}{b-a}\int_a^b v(t)\,\dee{t}=\frac{1}{b-a}\int_a^b x'(t)\,\dee{t}=\frac{x(b)-x(a)}{b-a}\]
\vfill
That is: $\dfrac{\text{change in distance}}{\text{change in time}}$

\vfill
Notice that this is exactly the formula we used way back at the start of your
\alert{differential} calculus class to help introduce the idea of the derivative. Of
course this is a very circuitous way to get to this formula --- but it is reassuring that
we get the same answer.
}
\unote{Example~\eref{text}{eg:AVvelocity}}
\end{frame}
%----------------------------------------------------------------------------------------

