% Copyright 2021 Joel Feldman, Andrew Rechnitzer and Elyse Yeager, except where noted.
% This work is licensed under a Creative Commons Attribution-NonCommercial-ShareAlike 4.0 International License.
% https://creativecommons.org/licenses/by-nc-sa/4.0/


 \begin{frame}{Table of Contents }
\mapofcontentsC{\cf}
 \end{frame}
%%----------------------------------------------------------------------------------------
%----------------------------------------------------------------------------------------
\section{3.6.1 Extending Taylor Polynomials}

%---------------------------------------------------------------------------------------
\newcommand{\taylorblock}{\begin{block}{Taylor series}
The Taylor series for the function $f(x)$ expanded around $a$
is the power series
\begin{align*}
\sum_{n=0}^\infty \frac{1}{n!}\,f^{(n)}(a)\, (x-a)^n.
\end{align*}
When $a=0$ it is also called the Maclaurin series of $f(x)$. 
\end{block}}
%----------------------------------------------------------------------------------------

%----------------------------------------------------------------------------------------
\begin{frame}[t]
\unote{CLP--1 Definition~\eref{text1}{def_3_4_2} and CLP--2 Definition~\eref{text}{defn:taylorSeries}, first part}

\begin{block}{Taylor polynomial}
  Let $a$ be a constant and let $n$ be a non-negative integer. The $n^\mathrm{th}$
order Taylor polynomial for $f(x)$ about $x=a$ is
\begin{align*}
  T_n(x) &= \sum_{k=0}^n \frac{1}{k!}\, f^{(k)}(a) \cdot (x-a)^k.
\end{align*}
\end{block}\pause

\taylorblock

\end{frame}

%----------------------------------------------------------------------------------------
\begin{frame}
Let's compute some Taylor series, using the definition.
\vfill
The method is nearly identical to finding Taylor \textit{polynomials}, which is covered in CLP--1.
\end{frame}

%----------------------------------------------------------------------------------------

\begin{frame}[t]
Find the Maclaurin series for $f(x)=\sin x$.
\vspace{1em}
\sQuestionBar<1-2>{1}{3}
\nsQuestionBar<1>{1}{3}
\AnswerBar<3->{1}{3}
\sonly<1>{\taylorblock\MoreSpace}
\AnswerYes<2>
\sonslide<3->{
\[\begin{array}{ccrcccr}
f(x)&=&\sin x &~\hspace{1cm}~&f(0)&=&0\\
f'(x)&=&\cos x &&f'(0)&=&1\\
f''(x)&=&-\sin x &&f''(0)&=&0\\
f'''(x)&=&-\cos x &&f'''(0)&=&-1\\
\end{array}\]
The derivatives then repeat. Notice we only have non-zero derivatives for odd orders, and these alternate in sign.

We can write the Maclaurin series as follows:
\begin{align*}
\sin x & \approx \frac{x^1}{1!}-\frac{x^3}{3!}+\frac{x^5}{5!}-\frac{x^7}{7!}+\cdots\\
&=\sum_{n=0}^\infty (-1)^{n}\frac{x^{2n+1}}{(2n+1)!}
\end{align*}
}
\end{frame}
%----------------------------------------------------------------------------------------
\begin{frame}[t]
Find the Maclaurin series for $f(x)=\cos x$.
\vspace{1em}
\QuestionBar<1-2>{2}{3}
\AnswerBar<3->{2}{3}
\only<1|handout:0>{\taylorblock\MoreSpace}
\AnswerYes<2>
\sonslide<3->{
\[\begin{array}{ccrcccr}
f(x)&=&\cos x &&f(0)&=&1\\
f'(x)&=&-\sin x &&f'(0)&=&0\\
f''(x)&=&-\cos x &&f''(0)&=&-1\\
f'''(x)&=&\sin x &&f'''(0)&=&0\\
\end{array}\]
The derivatives then repeat. Notice we only have non-zero derivatives for even orders, and these alternate in sign.

We can write the Maclaurin series as follows:
\begin{align*}
\cos x & \approx1- \frac{x^2}{2!}+\frac{x^4}{4!}-\frac{x^6}{6!}+\frac{x^8}{8!}-\cdots\\
&=\sum_{n=0}^\infty (-1)^{n}\frac{x^{2n}}{(2n)!}
\end{align*}
}
\end{frame}
%----------------------------------------------------------------------------------------
%----------------------------------------------------------------------------------------
%----------------------------------------------------------------------------------------
%----------------------------------------------------------------------------------------
\begin{frame}[t]
The Maclaurin series for $f(x)=e^x$\quad is\only<-2>{:}\quad \sonslide<3->{$\ds\sum\limits_{n=0}^\infty \frac{x^n}{n!}$.}
\vspace{1em}
\AnswerYes<2>
\QuestionBar<1-2>{3}{3}
\AnswerBar<3->{3}{3}
\only<1|handout:0>{\taylorblock\MoreSpace}


\sonslide<3->{
Every derivative of $e^x$ is $ e^x$, so all coefficients $f^{(n)}(0)$ are $e^0$, i.e. 1.

\begin{align*}
e^x&\approx 1+x+\frac{x^2}{2}+\frac{x^3}{3!}+\cdots \\
&= \sum_{n=0}^\infty \frac{x^n}{n!}
\end{align*}
}
\end{frame}

%----------------------------------------------------------------------------------------
\begin{frame}
Let $T_n(x)$ be the $n$-th order Taylor polynomial of the function $f(x)$, centred at $a$. 
\vfill
When we introduced Taylor polynomials in CLP--1, we framed $T_n(x)$ as an approximation of $f(x)$. 
\vfill
Let's see how those approximations look in two cases:
\end{frame}
%----------------------------------------------------------------------------------------
\begin{frame} {Taylor Polynomials for $e^x$}
\StatusBar{1}{10}
\begin{tikzpicture}
\foreach \y [count=\n] in {0.12,
0.37,
0.7866666667,
1.3075,
1.8283333333,
2.2623611111,
2.5723809524,
2.7661433532,
2.8737891314,
2.9276120205
}{
	\draw<\n|handout:0> (7.25,\y+1)node[C1,right]{$T_{\n}$};}
\begin{axis}[
%	xlabel=\(x\),
	axis lines = center,
	ymin=-50,
	ymax=190,
	clip=false]
\addplot[domain=-6:5, color=red,smooth] {exp(x)}node[above]{$e^x$};
\foreach \n in {1,...,10}{
	\MODULO{\n}{5}{\m}
\onslide<\n>{
	\xdef\taylor{0.0}
	\foreach \X in {1,...,\n}
		{\xdef\taylor{
			\taylor+x^\X/(\X!)
			}}
	\only<\n|handout:0>{
		\addplot[domain=-6:5,smooth,thick,color=C1] {\taylor};}
	}%onslide n
}%foreach n
\end{axis}
\end{tikzpicture}

\only<1|handout:0>{\[e^x \approx 1+x\qquad \text{ for } x \approx 0 \qquad \text{ (linear approximation)}\]}
\only<2|handout:0>{\[e^x \approx 1+x+\frac{x^2}{2} \qquad\text{ for } x \approx 0\qquad \text{ (quadratic approximation)}\]}
\foreach \n in {3,...,9}{
	\only<\n|handout:0>{\[e^x \approx 1+x+\frac{x^2}{2}\foreach \x in{3,...,\n}{+\frac{x^{\x}}{\x!}} \qquad\text{ for } x \approx 0\]}
	}
\only<10>{
\vspace{4mm}It seems like high-order Taylor polynomials do a pretty good job of approximating the function $e^x$, at least when $x$ is near enough to 0.}
\end{frame}
%----------------------------------------------------------------------------------------
%----------------------------------------------------------------------------------------

\begin{frame}{Taylor Polynomials for  a different function}

But that is not the case for all functions. Define
$$
f(x) = \begin{cases}
e^{-\frac1x} & x>0\\
0 & x \le 0
\end{cases}
$$
Using the definition of the derivative and l'H\^opital's rule, one can show that $f^{(n)}(0)=0$ for all natural numbers $n$.
\end{frame}

%----------------------------------------------------------------------------------------
%----------------------------------------------------------------------------------------

\begin{frame}[t]{Taylor Polynomials for  a different function}
\StatusBar{2}{6}
\begin{tikzpicture}
\myaxis{x}{3}{6}{y}{0}{3.5}
\draw[thick,W1](-2.9,0)-- plot[domain=0.1:1,smooth](\x,{4*exp(-1/\x)})-- plot[domain=1:4,smooth](\x,{4*exp(-1/\x)})node[right]{$f(x) = \begin{cases}
e^{-\frac1x} & x>0\\
0 & x \le 0
\end{cases}$};
\foreach \n in {1,...,5}{
	\ADD{\n}{1}{\s}\draw<\s|handout:0>[thick,C1] (-2.9,0)--(5.9,0)node[above]{$T_{\n}(x)=0$};}
\draw<handout>[thick,C1] (-2.9,0)--(5.9,0)node[above]{$T_{n}(x)=0$};
\end{tikzpicture}
\only<2|handout:0>{\[f(x) \approx 0+0x=0\qquad \text{ for } x \approx 0 \qquad \text{ (linear approximation)}\]}
\only<3|handout:0>{\[f(x) \approx 0+0x+0\frac{x^2}{2}=0 \qquad\text{ for } x \approx 0\qquad \text{ (quadratic approximation)}\]}
\only<4|handout:0>{\[f(x) \approx 0+0x+0\frac{x^2}{2}+0\frac{x^3}{3!}=0 \qquad\text{ for } x \approx 0\qquad \text{ (cubic approximation)}\]}
\only<5|handout:0>{\[f(x) \approx 0+0x+0\frac{x^2}{2}+0\frac{x^3}{3!}+0\frac{x^4}{4!}=0 \ \ \text{ for } x \approx 0\ \  \text{ (quartic approximation)}\]}
\only<6|handout:0>{\[f(x) \approx 0+0x+0\frac{x^2}{2}+0\frac{x^3}{3!}+0\frac{x^4}{4!}+0\frac{x^5}{5!}=0 \ \ \text{ for } x \approx 0\ \  \text{ (quintic approximation)}\]}

\vspace{1em}

\onslide<6->{Taylor polynomial approximations don't \alert{always} get better as their orders increase -- it depends on the function being approximated.}
\end{frame}
%----------------------------------------------------------------------------------------
%\begin{frame}[t]
%\StatusBar{1}{3}
%\unote{Equations~\eref{text}{eq:TaylorPolyPlusError}, \eref{text}{eq:TaylorPolyPlusError_a}, 
%\eref{text}{eq:TaylorPolyPlusError_b}; Definition~\eref{text}{defn:taylorSeries}, second part}
%\begin{align*}
%T_n(x)&=f(a)+f'(a)\,(x-a)+\cdots+\tfrac{1}{n!}f^{(n)}(a)\, (x-a)^n\\
%f(x) &=T_n(x) +\textcolor{W1}{E_n(x)}
%\onslide<2->{\intertext{The error satisfies the equation}
%\textcolor{W1}{E_n(x)}&=\tfrac{1}{(n+1)!}\,f^{(n+1)}(c)\, (x-a)^{n+1}}
%\end{align*}
%\onslide<2->{for some $c$ strictly between $a$ and $x$.}
%
%\onslide<3->{
%\begin{block}{Taylor series}
%If 
%$\lim\limits_{n\rightarrow\infty}\textcolor{W1}{E_n(x)}=0$, then
%\[
%f(x)=\sum_{n=0}^\infty \frac{1}{n!}\,f^{(n)}(a)\, (x-a)^n
%\]
%\end{block}}
%\end{frame}
%%----------------------------------------------------------------------------------------
%\begin{frame}
%Before we get into computing errors,\\ we'll warm up by finding some Taylor series.
%\end{frame}
%----------------------------------------------------------------------------------------
%----------------------------------------------------------------------------------------
%----------------------------------------------------------------------------------------
\section{Do the Taylor series match their functions?}
%----------------------------------------------------------------------------------------
%----------------------------------------------------------------------------------------
%----------------------------------------------------------------------------------------
%----------------------------------------------------------------------------------------
%----------------------------------------------------------------------------------------
\begin{frame}{Investigation}
\StatusBar{1}{3}
\begin{itemize}[<+->]
\item We found the Maclaurin series for $f(x)=e^x$ is $\ds\sum_{n=0}^\infty\frac{x^n}{n!}$.\vfill

\item But, it's not immediately clear whether  $e^x\stackrel{?}{=}\ds\sum_{n=0}^\infty\frac{x^n}{n!}$.\vfill
\item We're going to demonstrate that $e^x$ is in fact equal to $\ds\sum_{n=0}^\infty\frac{x^n}{n!}$. The proof involves a particular limit: $\lim\limits_{n \to \infty}\textcolor{C2}{\frac{|x|^n}{n!}}$. We'll talk about that limit first, so that it doesn't distract us later.
\end{itemize}
\end{frame}
%----------------------------------------------------------------------------------------
%----------------------------------------------------------------------------------------
\begin{frame}[t]
\StatusBar{2}{10}
\MoreSpace<11>
\AnswerYes<12>
\only<-4>{\COPY{3}{\xy}}
\only<5-6,11->{\COPY{4}{\xy}}
\only<7-8>{\COPY{5}{\xy}}
\only<9-10>{\COPY{4}{\xy}}
\ADD{\xy}{1}{\xz}
\SUBTRACT{\xy}{1}{\xx}
\newcommand{\xval}{
	\only<2,11-|handout:1>{\textcolor{M1}{|x|}}
	\only<3-4|handout:0>{\textcolor{M4}{2}}
	\only<5-6|handout:0>{\textcolor{M3}{3}}
	\only<7-8|handout:0>{\textcolor{M5}{4}}
	\only<9-10|handout:0>{\textcolor{M2}{\pi}}
	}
Intermediate result: $\lim\limits_{n \to \infty}\textcolor{C2}{\frac{|x|^n}{n!}}$, when $x$ is some fixed number.

\only<2-11|handout:1>{For large $n$, we can think of $\textcolor{C2}{\frac{|x|^n}{n!}}$ as a long multiplication, with decreasing terms. At some point, those terms are all decreasing \textit{and less than 1}.
\only<2-11|handout:1>{\begin{align*}
\only<2,11->{\frac{|x|^n}{n!}}%awkwardness around display of absolute values
\only<3-10|handout:0>{\frac{|\only<7-8|handout:0>{\textcolor{M5}{-}}\xval|^n}{n!}}&=\frac{
	\foreach \n in {1,...,6}{\xval\cdot}\ldots\cdot\xval
	}{\foreach \n in {1,...,6}{\n\cdot}\ldots\cdot n}\\
	&=\foreach \n[evaluate=\s using int(2*\n+2),evaluate=\m using \n-1, evaluate=\ss using int(\s-1)] in {1,...,4}{
		\only<\ss|handout:0>{\foreach \k in {1,...,6}{
				\left(\frac{\xval}{\k}\right)}
				\cdots
				 \left(\frac{\xval}{n}\right)}
		\only<\s|handout:0>{
			\foreach \k in {1,...,\xx}{
				\left(\frac{\xval}{\k}\right)}
			~|~
			\underbrace{\left(\frac{\xval}{\xy}\right)}_{<1}
			\foreach \k in {\xz,...,6}{
			\underbrace{\left(\frac{\xval}{\k}\right)}_{<\frac{\xval}{\xy}}}
			\cdots \underbrace{\left(\frac{\xval}{n}\right)}_{<\frac{\xval}{\xy}}
		}}
\only<2,11|handout:0>{\foreach \n in {1,...,6}{\left(\frac{\xval}{\n}\right)}\cdots\left(\frac{\xval}{n}\right)}\end{align*}}}
\only<11|handout:2>{We're multiplying terms that are closer and closer to 0, so it seems quite reasonable that this sequence should converge to 0.\\[1em]
 For a more formal proof, we can use the squeeze theorem to compare this sequence to a geometric sequence.}

\only<12>{\unote{Convenient notation: $\lceil x \rceil$ is the number you get when you round $x$ up to the nearest whole number.}}
\only<12-|handout:0>{
Let $\frac{|x|}{k}$ be the first factor that's less than 1. Then when $n>k$:
\sonly<13->{
\begin{align*}
\frac{|x|^n}{n!}&=\left(\frac{|x|}{1}\frac{|x|}{2}\cdots\frac{|x|}{k-1}\right)\left(\frac{|x|}{k}\cdots\frac{|x|}{n} \right)\\
&<\left(\frac{|x|}{1}\frac{|x|}{2}\cdots\frac{|x|}{k-1}\right)\left(\frac{|x|}{k}\right)^{n-(k-1)}\\
&=\underbrace{\frac{\left(\frac{|x|}{1}\frac{|x|}{2}\cdots\frac{|x|}{k-1}\right)}{
	\left(\frac{|x|}{k}\right)^{k-1}}}_{a}\underbrace{\left(\frac{|x|}{k}\right)^n}_{r^n}
\end{align*}
Since $|r|<1$, the sequence $ar^n$ (as defined above) converges to 0. Since $0 \le \frac{|x^n|}{n!} < ar^n$ for large $n$,  we conclude by the squeeze theorem that
\[\lim_{n \to \infty}\textcolor{C2}{\frac{|x|^n}{n!}}=0.\]
}}
\end{frame}
%----------------------------------------------------------------------------------------

%----------------------------------------------------------------------------------------
\begin{frame}{Investigation}

\label<1>{note3.6a}

\StatusBar{1}{2}
\begin{itemize}
\item We found the Maclaurin series for $f(x)=e^x$ is $\ds\sum_{n=0}^\infty\frac{x^n}{n!}$.
\item But, it's not immediately clear whether  $e^x\stackrel{?}{=}\ds\sum_{n=0}^\infty\frac{x^n}{n!}$.\\[-2mm]
 How could we determine this?
\item<2->{ \begin{align*}
e^x&=\sum_{n=0}^\infty\frac{x^n}{n!}\\
\iff
0&=e^x - \sum_{n=0}^\infty \frac{x^n}{n}=e^x-\lim_{n \to \infty}\underbrace{\sum_{k=0}^n\frac{x^k}{k!}}_{T_n(x)}=\lim_{n \to \infty}\underbrace{\left[e^x-T_n(x)\right]}_{E_n(x)}\\
\iff 0&=\lim_{n \to \infty}E_n(x)\quad\text{ (for all $x$)}
\end{align*}}
\end{itemize}
\end{frame}
%----------------------------------------------------------------------------------------

%----------------------------------------------------------------------------------------
\begin{frame} {Taylor Polynomial Error for $f(x)=e^x$}
\COPY{7}{\lastpoly}% last taylor polynomial to display
\StatusBar{1}{\lastpoly}
If $\lim\limits_{n \to \infty}E_n(x)=0$ for all $x$, then $e^x=\sum_{n=0}^\infty\frac{x^n}{n!}$ for all $x$.
 
 \onslide<\lastpoly>{It \textit{looks} plausible, especially when $x$ is close to 0. Let's try to prove it.}
 \vfill
\begin{tikzpicture}
\begin{axis}[
	axis lines = center,
	ymin=-50,
	ymax=190,
	clip=false]
\addplot[domain=-6:5, color=red,smooth] {exp(x)}node[above]{$e^x$};
%plot Taylor polynomials
\foreach \n in {1,...,\lastpoly}{
\onslide<\n>{
	\xdef\taylor{0.0}%recursively define the taylor polynomials
	\foreach \X in {1,...,\n}
		{\xdef\taylor{
			\taylor+x^\X/(\X!)
			}}
		%draw in several error lines; for loops aren't working in this environment
		\addplot[domain=-5:-5.01,opacity=0]{\taylor}coordinate(T-5){};
		\addplot[domain=-5:-5.01,opacity=0=0]{exp(\x)}coordinate(f-5){};
		\addplot[domain=2:2.01,opacity=0]{\taylor}coordinate(T2){};
		\addplot[domain=2:2.01,opacity=0]{exp(\x)}coordinate(f2){};
		\addplot[domain=4:4.01,opacity=0]{\taylor}coordinate(T4){};
		\addplot[domain=4:4.01,opacity=0]{exp(\x)}coordinate(f4){};
		\only<\n|handout:0>{\addplot[domain=-6:5,smooth,thick,color=C1] {\taylor};
		\draw[|-|](T-5)--(f-5);
		\draw[|-|](T2)--(f2);
		\draw[|-|](T4)--(f4);
		}%\only<\n>
	}%onslide n
}%foreach n
\end{axis}

\begin{scope}[W1,xshift=3.75 cm,yshift=1.22cm,xscale=1.18/2,yscale=1.2/50]%scale to match axis
%label errors; can't access \n inside the axis environment, so doing it separately here
\foreach \s in {1,...,\lastpoly}{\foreach \x[evaluate=\y using exp(\x)] in {-5,2,4}{
	\onslide<\s|handout:0>{\draw(\x,\y)node[above,yshift=2mm]{$E_{\s}(\x)$};}
	}}
\end{scope}
\end{tikzpicture}
\end{frame}
%----------------------------------------------------------------------------------------
%----------------------------------------------------------------------------------------
\begin{frame}[t]
\AnswerSpace
\only<2>{\MoreSpace}
\begin{block}{Equation~\eref{text}{eq:TaylorPolyPlusError_b}}
Let $T_n(x)$ be the $n$-th order Taylor approximation of a function $f(x)$, centred at $a$. Then $E_n(x)=f(x)-T_n(x)$ is the error in the $n$-th order Taylor approximation.

For some $c$ strictly between $x$ and $a$,
\[E_n(x)=\frac{1}{(n+1)!}\ f^{(n+1)}(c)\cdot(x-a)^{n+1}\]
\end{block}
\unote{CLP--1 Equation~\eref{text1}{eq:taylorErrorN}, CLP--2 Equation~\eref{text}{eq:TaylorPolyPlusError_b}}
\pause\vfill
When $f(x)=e^x$,
\[E_n(x)=e^c\frac{x^{n+1}}{(n+1)!}\]
for some $c$ between 0 and $x$.
\end{frame}
%----------------------------------------------------------------------------------------
\begin{frame}[t]
\AnswerYes<1>
\color{spoilercolor}
\begin{align*}
\color{black}E_n(x)&\color{black}=e^x-T_n(x)\\
&\color{black}=e^c\frac{x^{n+1}}{(n+1)!}&\color{black}\mbox{for some $c$ between 0 and $x$}\\
\sonslide<2->{0\le |E_n(x)|&< \left| e^c\frac{x^{n+1}}{(n+1)!}\right|\\
&\le e^{|x|}\color{C2}\frac{|x|^{n+1}}{(n+1)!}\\
0&=\lim_{n\to\infty}\color{C2}\frac{|x|^{n+1}}{(n+1)!}&\mbox{ by our previous result}\\
\implies 0&=\lim_{n \to \infty}|E_n(x)|&\mbox{by the squeeze theorem}
}
\end{align*}
\end{frame}
%----------------------------------------------------------------------------------------
%----------------------------------------------------------------------------------------
\begin{frame}
\COPY{7}{\lastpoly}% last taylor polynomial to display
\StatusBar{1}{\lastpoly}
\only<1>{\AnswerYes}
We found $0 \le |E_n(x)|<e^{|x|}\frac{|x|^{n+1}}{(n+1)!}$ for large $n$, hence $\lim\limits_{n \to \infty}|E_n(x)|=0$.
 
 \vfill
 \parbox{5.75cm}{
\begin{tikzpicture}[xscale=0.8]
\begin{axis}[
	axis lines = center,
	ymin=-50,
	ymax=190,
	clip=false]
\addplot[domain=-6:5, color=red,smooth] {exp(x)}node[above]{$e^x$};
%plot Taylor polynomials
\foreach \n in {1,...,\lastpoly}{
\onslide<\n>{
	\xdef\taylor{0.0}%recursively define the taylor polynomials
	\foreach \X in {1,...,\n}
		{\xdef\taylor{
			\taylor+x^\X/(\X!)
			}}
		%draw in several error lines; for loops aren't working in this environment
		\addplot[domain=-5:-5.01,opacity=0]{\taylor}coordinate(T-5){};
		\addplot[domain=-5:-5.01,opacity=0=0]{exp(\x)}coordinate(f-5){};
		\addplot[domain=4:4.01,opacity=0]{\taylor}coordinate(T4){};
		\addplot[domain=4:4.01,opacity=0]{exp(\x)}coordinate(f4){};
		\only<\n|handout:0>{\addplot[domain=-6:5,smooth,thick,color=C1] {\taylor};
		\draw[|-|](T-5)--(f-5);
		\draw[|-|](T4)--(f4);
		}%\only<\n>
	}%onslide n
}%foreach n
\draw(axis cs: 0,0) coordinate;
\pgfgetlastxy{\Xorigin}{\Yorigin} 
\end{axis}

\begin{scope}[W1,xshift=3.75 cm,yshift=1.22cm,xscale=1.18/2,yscale=1.2/50]%scale to match axis
%label errors; can't access \n inside the axis environment, so doing it separately here
\foreach \s[evaluate=\t using int(\s+1)] in {1,...,\lastpoly}{\foreach \x[evaluate=\y using exp(\x),evaluate=\X using int(abs(\x))] in {-5,4}{
	\onslide<\s|handout:0>{\draw(\x*.7,\y)node[above,yshift=2mm,fill=white,inner sep=0]{$|E_{\s}(\text{\x})|<e^{\text{\X}}\left(\textcolor{C2}{\frac{\X^{\t}}{\t!}}\right)$};
	}
	}}
\end{scope}
\end{tikzpicture}}\hfill
\parbox{4cm}{
For a particular value of $x$:
\begin{align*}
\text{We saw }\quad0&=\lim_{n \to \infty}\textcolor{C2}{\frac{|x|^{n+1}}{(n+1)!}}\\
\onslide<2->{\text{so }\quad 0&=\lim_{n \to \infty}E_n(x)\\
\text{That is, }~ 0&=\lim_{n \to \infty}\left[e^x-T_n(x)\right]\\
\text{So, }e^x&=\lim_{n \to \infty}T_n(x)\\
&=\sum_{n=0}^{\infty} \frac{x^n}{n!}}
\end{align*}}
\end{frame}
%----------------------------------------------------------------------------------------%----------------------------------------------------------------------------------------
%----------------------------------------------------------------------------------------%----------------------------------------------------------------------------------------
%----------------------------------------------------------------------------------------
%----------------------------------------------------------------------------------------
\begin{frame}[t]{Taylor polynomial error for sine and cosine}
\MoreSpace<1>
\AnswerYes<2>
\only<1>{\begin{block}{Equation~\eref{text}{eq:TaylorPolyPlusError_b}}
Let $T_n(x)$ be the $n$-th order Taylor approximation of a function $f(x)$, centred at $a$. Then $E_n(x)=f(x)-T_n(x)$ is the error in the $n$-th order Taylor approximation.

For some $c$ strictly between $x$ and $a$,
\[E_n(x)=\frac{1}{(n+1)!}\ f^{(n+1)}(c)\cdot(x-a)^{n+1}\]
\end{block}
\unote{CLP--1 Equation~\eref{text1}{eq:taylorErrorN}, CLP--2 Equation~\eref{text}{eq:TaylorPolyPlusError_b}}
\vfill}

Suppose $f(x)$ is either $\sin x$ or $\cos x$. \only<1>{Is $f(x)$ equal to its Maclaurin series? }\sonslide<3->{In either case, $|f^{(n+1)}(c)|$ is either $|\sin c|$ or $|\cos c|$, so it's between 0 and 1.}

\begin{align*}
\onslide<2->{\color{black}\left|\textcolor{W1}{E_n(x)}\right|&
\color{black}=\frac{1}{(n+1)!}\left|f^{(n+1)}(c)\right||x|^{n+1}}
\sonly<3->{\leq\textcolor{C2}{ \frac{|x|^{n+1}}{(n+1)!}}\\
\implies 0  \le |E_n(x)|  &\le\textcolor{C2}{ \frac{|x|^{n+1}}{(n+1)!}}
\intertext{We saw before that $\lim\limits_{n \to \infty} \textcolor{C2}{\frac{|x|^{n+1}}{(n+1)!}}=0$. So, by the squeeze theorem,}\quad
\lim_{n \to \infty}|\textcolor{W1}{E_n(x)}|&=0\\
}
\end{align*}

\sonly<3->{So sine and cosine are equal to their Taylor series everywhere.}
\end{frame}
%----------------------------------------------------------------------------------------
%----------------------------------------------------------------------------------------
%----------------------------------------------------------------------------------------


\begin{frame} {Taylor Polynomials for $\sin(x)$}
\StatusBar{1}{8}
\centerline{\begin{tikzpicture}
\begin{axis}[
	axis lines = center,
	ymin=-5,
	ymax=5,
	clip=false]
\addplot[domain=-12:12, color=red,smooth,samples=100] {sin(x*180/pi)}node[right]{$\sin(x)$};
\foreach \n in {1,...,8}{
	\MODULO{\n}{5}{\m}
\onslide<\n>{
	\xdef\taylor{x}
	\foreach \X in {1,...,\n}
		{\xdef\taylor{
			\taylor+(-1)^\X*x^(2*\X+1)/((2*\X+1)!)
			}}
	\only<\n|handout:0>{
	\addplot[domain=-13:13,smooth,thick,color=C1, restrict y to domain =-5:5,samples=100] {\taylor};}
	}%onslide n
}%foreach n
\addplot[opacity=0,domain=-10:10]{x*.13};%keeping space free for Taylor polynomials
\end{axis}
\end{tikzpicture}}


\only<1|handout:0>{\[T_3(x)= x-\frac{x^3}{3!} \]}
\foreach \n[evaluate=\Nindex using int(2*\n+1)] in {2,...,8}{
	\only<\n|handout:0>{\[T_{\Nindex}(x)= x-\frac{x^3}{3!}\foreach \x[evaluate=\N using int(2*\x+1)] in{2,...,\n}{\ifodd \x -\frac{x^{\N}}{\N!} \else +\frac{x^{\N}}{\N!} \fi}\]}
	}
\end{frame}
%----------------------------------------------------------------------------------------

%----------------------------------------------------------------------------------------
\begin{frame} {Taylor Polynomials for $\cos(x)$}
\StatusBar{1}{8}
\centerline{\begin{tikzpicture}
\begin{axis}[
	axis lines = center,
	ymin=-5,
	ymax=5,
	clip=false]
\addplot[domain=-12:12, color=red,smooth,samples=100] {cos(x*180/pi)}node[right]{$\cos(x)$};
\foreach \n in {1,...,8}{
	\MODULO{\n}{5}{\m}
\onslide<\n>{
	\xdef\taylor{1}
	\foreach \X in {1,...,\n}
		{\xdef\taylor{
			\taylor+(-1)^\X*x^(2*\X)/((2*\X)!)
			}}
	\only<\n|handout:0>{
	\addplot[domain=-13:13,smooth,thick,color=C1, restrict y to domain =-5:5,samples=100] {\taylor};}
	}%onslide n
}%foreach n
\addplot[opacity=0,domain=-10:10]{x*.13};%keeping space free for Taylor polynomials
\end{axis}
\end{tikzpicture}}


\only<1|handout:0>{\[T_2(x)=1- x^2 \]}
\foreach \n[evaluate=\Nindex using int(2*\n)] in {2,...,8}{
	\only<\n|handout:0>{\[T_{\Nindex}(x)= 1-x^2\foreach \x[evaluate=\N using int(2*\x)] in{2,...,\n}{\ifodd \x -\frac{x^{\N}}{\N!} \else +\frac{x^{\N}}{\N!} \fi}\]}
	}
\end{frame}
%----------------------------------------------------------------------------------------

%----------------------------------------------------------------------------------------
\begin{frame}[t]

\unote{\eref{text}{thm:SRimportantTaylorSeries}}
\begin{block}{Selected Taylor series that equal their functions}
\begin{alignat*}{5}
e^x &= \textstyle\sum\limits_{n=0}^\infty\dfrac{x^n}{n!}
    &&\qquad\text{for all $-\infty<x<\infty$} \\[1mm]
\sin(x) &= \textstyle\sum\limits_{n=0}^\infty(-1)^n\dfrac{1}{(2n+1)!}x^{2n+1}
    &&\qquad\text{for all $-\infty<x<\infty$} \\[1mm]
\cos(x) &= \textstyle\sum\limits_{n=0}^\infty(-1)^n\dfrac{1}{(2n)!}x^{2n}
    &&\qquad\text{for all $-\infty<x<\infty$} \\[1mm]
\frac{1}{1-x} &= \textstyle\sum\limits_{n=0}^\infty x^n
    &&\qquad\text{for all $-1<x<1$} \\[1mm]
\log(1+x) &= \textstyle\sum\limits_{n=0}^\infty (-1)^n\dfrac{x^{n+1}}{n+1}
    &&\qquad\text{for all $-1<x\le 1$} \\[1mm]
\arctan x &=  \textstyle\sum\limits_{n=0}^\infty (-1)^n\dfrac{x^{2n+1}}{2n+1}
    &&\qquad\text{for all $-1\le x\le 1$}
\end{alignat*}
\end{block}
\end{frame}
%----------------------------------------------------------------------------------------
%----------------------------------------------------------------------------------------
\section{3.6.2 Computing with Taylor Series}
%---------------------------------------------------------------------------------------%----------------------------------------------------------------------------------------

%----------------------------------------------------------------------------------------
\begin{frame}[t]{Computing $\pi$}
\QuestionBar<1>{1}{4}
\AnswerBar<2>{1}{4}
\unote{Example~\eref{text}{eg:pi}}
Use the fact that $\arctan 1 = \frac{\pi}{4}$ to find a series converging to $\pi$ whose terms are rational numbers.

\sonslide<2->{
For all $-1 \le x\le 1$:
\begin{align*}
\arctan x & =\sum_{n=0}^\infty(-1)^n\frac{x^{2n+1}}{2n+1}\\
4\arctan x & =4\sum_{n=0}^\infty(-1)^n\frac{x^{2n+1}}{2n+1}\\
\pi=4\arctan 1 & =4\sum_{n=0}^\infty(-1)^n\frac{1^{2n+1}}{2n+1}\\
&=\sum_{n=0}^\infty(-1)^n\frac{4}{2n+1}\\
&=4-\frac43+\frac45-\frac47+\frac49-\cdots 
\end{align*}
}
\end{frame}
%----------------------------------------------------------------------------------------
%----------------------------------------------------------------------------------------
\begin{frame}{Error function}
%\QuestionBar{2}{4}
\unote{Example~\eref{text}{eg:erf}}
The \textit{error function}
\[\text{erf}(x) =\frac{2}{\sqrt{\pi}}\int_0^x e^{-t^2}\ \dee{t}\]
is used in computing ``bell curve'' probabilities. 

\end{frame}
%----------------------------------------------------------------------------------------
\begin{frame}
\StatusBar{1}{12}
\begin{tikzpicture}[xscale=1.5]
\myaxis{x}{0}{2}{y}{0}{7}
\draw[very thick] plot[domain=0:1.39](\x,{exp(\x*\x)})node[right]{$y=e^{x^2}$};
\foreach \x in {0,0.1,0.2,0.3,0.4,0.5,0.6,0.7,0.8,0.9,1,1.1}{\onslide<+|handout:0>{
	\xcoord{\x}{\x}
	\filldraw[fill=C3,fill opacity=0.5] plot[domain=0:\x](\x,{exp(\x*\x)})|-(0,0)--(0,1);
	}}
\end{tikzpicture}
\hfill
\begin{tikzpicture}[xscale=1.5]
\myaxis{x}{0}{2}{y}{0}{7}
\foreach \x[count=\n,evaluate=\l using \x+\x^3/3+\x^5/10+\x^7/42+\x^9/216] in {0,0.1,0.2,0.3,0.4,0.5,0.6,0.7,0.8,0.9,1,1.1}{
	\onslide<\n|handout:0>{
	\xcoord{\x}{\x}
	\ifnum \n>1
	\ROUND[2]{\l}{\L}
	\draw[very thick] plot[domain=0:\x](\x,{\x+\x^3/3+\x^5/10+\x^7/42+\x^9/216})node[above right,fill=white,inner sep=1mm]{$\ds\int_0^{\x}e^{x^2}\,\dee x \approx \L$};
	\ycoord{\L}{\L}
	\else
	\draw (0,0)node[above right]{$\ds\int_0^0e^{x^2}\,\dee x =0$};
	\fi
	}}
\onslide<13->{\draw[very thick] plot[domain=0:1.745](\x,{\x+\x^3/3+\x^5/10+\x^7/42+\x^9/216})node[right]{$\ds\int_0^xe^{t^2}\,\dee t$};}

\onslide<14->{\draw[very thick,C3] plot[domain=0:1.694](\x,{1.128*(\x+\x^3/3+\x^5/10+\x^7/42+\x^9/216)})node[left,xshift=-2mm,fill=white,inner sep=1mm]{$\ds\frac{2}{\sqrt\pi}\int_0^xe^{t^2}\,\dee t$};}
\end{tikzpicture}
\end{frame}
%----------------------------------------------------------------------------------------
%----------------------------------------------------------------------------------------
\begin{frame}[t]{Error function}
\MoreSpace<1>
\AnswerYes<2>
\QuestionBar<1-2>{2}{4}
\AnswerBar<3->{2}{4}
\unote{Example~\eref{text}{eg:erf}}
\only<-2>{The \textit{error function}
\[\text{erf}(x) =\frac{2}{\sqrt{\pi}}\int_0^x e^{-t^2}\ \dee{t}\]
is used in computing ``bell curve'' probabilities. }

\only<1>{\vspace{1em}The indefinite integral
of the integrand $e^{-t^2}$ cannot be expressed in terms of standard
functions. But we can still evaluate the integral to within any desired degree
of accuracy by using the Taylor expansion of the exponential.

For example, evaluate $\text{erf}\left(\frac{1}{\sqrt 2}\right)$.}



\sonslide<2->{\begin{align*}
\text{erf}\left(\frac{1}{\sqrt 2}\right)&=\onslide<3->{
\frac{2}{\sqrt{\pi}}\int_0^{\frac{1}{\sqrt 2}} e^{-t^2}\ \dee{t}
=\frac{2}{\sqrt\pi}\int_0^{\frac{1}{\sqrt 2}} 
\left.\left(\sum_{n=0}^\infty \frac{x^n}{n!}\right)\right|_{x=-t^2}
\ \dee{t}\\
&=\frac{2}{\sqrt\pi}\int_0^{\frac{1}{\sqrt 2}} 
\left(\sum_{n=0}^\infty\frac{(-1)^n t^{2n}}{n!}\right)
\ \dee{t}
=\frac{2}{\sqrt\pi}
\left[
\sum_{n=0}^\infty \frac{(-1)^nt^{2n+1}}{(2n+1)n!}
\right]_0^{\frac{1}{\sqrt 2}} \\
&=\frac{2}{\sqrt\pi}
\left[
\sum_{n=0}^\infty \frac{(-1)^n}{(2n+1)n!{(\sqrt 2)}^{2n+1}}-
\sum_{n=0}^\infty \frac{(-1)^n\cdot 0^{2n+1}}{n!\cdot(2n+1)}
\right]\\
&=\frac{2}{\sqrt \pi}\sum_{n=0}^\infty\frac{(-1)^n}{ \sqrt 2\cdot 2^n(2n+1)n!}
=
\sqrt{\frac{2}{\pi}}\sum_{n=0}^\infty\frac{ (-1)^n}{2^n(2n+1)n!}\\
&=\sqrt{\frac{2}{\pi}}\left(1-\frac{1}{6}+\frac{1}{40}-\frac{1}{336}+\cdots\right)
}%end slide 3-
\end{align*}
}%end slide 2-
\end{frame}
%----------------------------------------------------------------------------------------
%----------------------------------------------------------------------------------------
\begin{frame}[t]{Evaluating a convergent series}
\AnswerSpace
\sQuestionBar<1-2>{3}{4}
\nsQuestionBar<1>{3}{4}
\sMoreSpace<1>
\AnswerYes<2>
\AnswerBar<3->{3}{4}
Evaluate $\ds\sum_{n=1}^\infty\frac{1}{n\cdot 3^n}$
\unote{Example~\eref{text}{eg:lntwo}}

\sonly<1>{\scriptsize\vspace{1em}
\fbox{\parbox{.8\textwidth}{\vspace{-2mm}
\begin{alignat*}{5}
e^x &= \textstyle\sum\limits_{n=0}^\infty\dfrac{x^n}{n!}
    &&\qquad\text{for all $-\infty<x<\infty$} \\[1mm]
\sin(x) &= \textstyle\sum\limits_{n=0}^\infty(-1)^n\dfrac{1}{(2n+1)!}x^{2n+1}
    &&\qquad\text{for all $-\infty<x<\infty$} \\[1mm]
\cos(x) &= \textstyle\sum\limits_{n=0}^\infty(-1)^n\dfrac{1}{(2n)!}x^{2n}
    &&\qquad\text{for all $-\infty<x<\infty$} \\[1mm]
\frac{1}{1-x} &= \textstyle\sum\limits_{n=0}^\infty x^n
    &&\qquad\text{for all $-1<x<1$} \\[1mm]
\log(1+x) &= \textstyle\sum\limits_{n=0}^\infty (-1)^n\dfrac{x^{n+1}}{n+1}
    &&\qquad\text{for all $-1<x\le 1$} \\[1mm]
\arctan x &=  \textstyle\sum\limits_{n=0}^\infty (-1)^n\dfrac{x^{2n+1}}{2n+1}
    &&\qquad\text{for all $-1\le x\le 1$}
\end{alignat*}\vspace{-2mm}
}}}
\sonslide<3->{
The series most closely resembles the Taylor series $\log(1+x)=\sum\limits_{m=0}^\infty (-1)^m\dfrac{x^{m+1}}{m+1}$. To make that relation clearer, set $m=n-1$:
\begin{align*}
\sum_{n=1}^\infty\frac{1}{n\cdot 3^n}&=\sum_{m=0}^\infty \frac{1}{(m+1)\cdot 3^{m+1}}\\
&=\sum_{m=0}^\infty(-1)^{m+1} \frac{(-1)^{m+1}}{(m+1)\cdot 3^{m+1}}\\
&=-\sum_{m=0}^\infty(-1)^{m} \frac{\left(-\frac13\right)^{m+1}}{(m+1)}\\
&=-\log\left(1-\frac13\right)=-\log\left(\frac23\right)=\log\left(\frac32\right)
\end{align*}
}
\end{frame}
%----------------------------------------------------------------------------------------%----------------------------------------------------------------------------------------
\begin{frame}[t]{Finding a high-order derivative}
\QuestionBar<1>{4}{4}
\AnswerYes<1-2>
\AnswerBar<2->{4}{4}
Let $f(x) = \sin(2x^3)$. Find $f^{(15)}(0)$, the fifteenth derivative
of $f$ at $x=0$.\vspace{1em}


\sonly<2>{
Differentiating directly gets messy quickly. Instead, let's find the Taylor series. Let $y=2x^3$:
\begin{align*}
\sin(y)&=\sum_{n=0}^\infty(-1)^n\dfrac{1}{(2n+1)!}y^{2n+1}\\
\implies \quad f(x)=\sin(2x^3)&=\sum_{n=0}^\infty(-1)^n\dfrac{1}{(2n+1)!}{(2x^3)}^{2n+1}\\
\implies \quad f(x)=\sum_{m=0}^\infty \frac{f^{(m)}(0)}{m!}x^m &=
\sum_{n=0}^\infty(-1)^n\dfrac{2^{2n+1}}{(2n+1)!}x^{6n+3}
\end{align*}
}
\sonly<3>{
The coefficients of $x^{15}$ on the left and right series must match for the series to be equal.

When $m=15$ on the left-hand side, we get the term $\frac{f^{(15)}(0)}{15!}x^{15}$. The right-hand side term corresponding to $x^{15}$ occurs when $6n+3=15$, i.e. when $n=2$.

\begin{align*}
\underbrace{\frac{f^{(15)}(0)}{15!}}_{m=15}&=\underbrace{(-1)^2\frac{2^5}{5!}}_{n=2}\\
f^{(15)}(0)&=\frac{15!}{5!}\cdot 2^5
\end{align*}
}
\unote{Example~\eref{text}{eg:SRfindDeriv}.}
\end{frame}
%----------------------------------------------------------------------------------------
%----------------------------------------------------------------------------------------
\section{3.6.4 Evaluating limits}
%---------------------------------------------------------------------------------------%----------------------------------------------------------------------------------------

%----------------------------------------------------------------------------------------
\begin{frame}[t]
\AnswerYes<1>
\QuestionBar<1>{1}{2}
\AnswerBar<2->{1}{2}
\unote{Example~\eref{text}{eg:TaylorlimitA}}
Given that $\sin x = x-\frac{x^3}{3!}+\frac{x^5}{5!}-\cdots$, we have a new way of evaluating the familiar limit
\[\lim_{x \to 0}~\frac{\sin x}{x}\colon\]

\vfill
\sonly<2>{
\begin{align*}
\lim_{x \to 0}~\frac{\sin x}{x}&=
\lim_{x \to 0}~\frac{ x-\frac{x^3}{3!}+\frac{x^5}{5!}-\cdots}{x}\\[2mm]
&=\lim_{x \to 0}\left[1-\frac{x^2}{3!}+\frac{x^4}{5!}-\cdots\right]\\[2mm]
&=0
\end{align*}
\vfill
This technique is sometimes faster than l'H\^opital's rule.}
\end{frame}
%----------------------------------------------------------------------------------------%----------------------------------------------------------------------------------------

%----------------------------------------------------------------------------------------
\begin{frame}[t]
\AnswerYes<1>\QuestionBar<1>{2}{2}
\AnswerBar<2->{2}{2}
\unote{Example~\eref{text}{eg:TaylorlimitB}}
Evaluate $\ds\lim_{x\rightarrow 0}~\frac{\arctan x -x}{\sin x-x}$.



\sonly<2>{
\begin{align*}
\arctan x - x &= \left(x-\frac{x^3}{3}+\frac{x^5}{5}-\cdots\right)-x\\
&=-\frac{x^3}{3}+\frac{x^5}{5}-\cdots\\[1em]
\sin x - x & = \left(x-\frac{x^3}{3!}+\frac{x^5}{5!}-\cdots\right)-x\\
&=-\frac{x^3}{3!}+\frac{x^5}{5!}-\cdots\\[1em]
\lim_{x \to 0}~\frac{\arctan x -x}{\sin x-x}&=
\lim_{x \to 0}~\frac{-\frac{x^3}{3}+\frac{x^5}{5}-\cdots}{-\frac{x^3}{3!}+\frac{x^5}{5!}-\cdots}\left(\frac{\frac{1}{x^3}}{\frac{1}{x^3}}\right)\\
&=\lim_{x \to 0}\frac{-\frac13+\frac{x^2}{5}-\cdots}{-\frac1{3!}+\frac{x^2}{5!}-\cdots}=\frac{-\frac13}{-\frac16}=2
\end{align*}
}

\end{frame}
%----------------------------------------------------------------------------------------%----------------------------------------------------------------------------------------
%----------------------------------------------------------------------------------------
