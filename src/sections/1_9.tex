% Copyright 2021 Joel Feldman, Andrew Rechnitzer and Elyse Yeager, except where noted.
% This work is licensed under a Creative Commons Attribution-NonCommercial-ShareAlike 4.0 International License.
% https://creativecommons.org/licenses/by-nc-sa/4.0/


 \begin{frame}{Table of Contents }
\mapofcontentsA{\ai,\atech}
 \end{frame}
%----------------------------------------------------------------------------------------

\section{1.9 Trigonometric Substitution}
%----------------------------------------------------------------------------------------

%----------------------------------------------------------------------------------------
\begin{frame}[t]{Warmup}
\only<1>{\AnswerYes}\AnswerSpace
Evaluate $\displaystyle\int_3^7\frac{1}{\sqrt{x^2+2x+1}}~\dee x$.\vfill

\sonslide<2->{
\begin{align*}
\int_3^7\frac{1}{\sqrt{x^2+2x+1}}~\dee x&=\int_3^7\frac{1}{\sqrt{(x+1)^2}}~\dee x\\
&=\int_3^7\frac{1}{|x+1|}~\dee x
\intertext{When $3 \le x \le 7$, we have $|x+1|=x+1$.}
&=\int_3^7\frac{1}{x+1}~\dee x\\
&=\left[\log|x+1|\right]_3^7\\
&=\log8-\log4=\log2
\end{align*}
\pause\color{W1}\vfill
Idea: $\sqrt{(\text{something})^2}=|\text{something}|$. 
We cancelled off the square root.
}
\end{frame}
%----------------------------------------------------------------------------------------
\begin{frame}[t]
Evaluate $\displaystyle\int\frac{1}{\sqrt{x^2+1}}~\dee x$.

\sonslide<2->{\vfill
We still want to cancel off the square root, but $x^2+1$ is not obviously of the form $(\text{something})^2$.

Let $\textcolor{M3}{x=\tan \theta}$, $\textcolor{C3}{\dee x=\sec^2\theta\, \dee \theta}$.
\begin{align*}
\int&\frac{1}{\sqrt{\textcolor{M3}x^2+1}}~\textcolor{C3}{\dee x}
=\int\frac{1}{\sqrt{\textcolor{M3}\tan^2\textcolor{M3}\theta+1}}\,\textcolor{C3}{\sec^2\theta \dee \theta}
=\int\frac{\sec^2\theta}{\sqrt{\sec^2\theta}}\dee \theta\\
&=\int\frac{\sec^2\theta}{\sec\theta}\dee \theta=\int\sec \theta\, \dee \theta=\log|\sec \theta + \tan \theta|+C
\intertext{We need to get these back in terms of $x$. From our substitution, we know $\tan\theta=x$. From simplifying our denominator, we also know $\sec \theta = \sqrt{x^2+1}$.}
&=\log\left| \sqrt{x^2+1}+x\right|+C
\end{align*}

\color{W1}
\vfill
Same idea: $\sqrt{(\text{something})^2}=|\text{something}|$; cancel off the square root.
}
\end{frame}
%----------------------------------------------------------------------------------------
\CheckFrame{
Let's verify that $\displaystyle\int\frac{1}{\sqrt{x^2+1}}=\onslide<beamer>{\log\left| \sqrt{x^2+1}+x\right|+C}$.\\
Seems improbable, right?}{
\begin{align*}
\diff{}{x}&\left[ \log\left| \sqrt{x^2+1}+x\right|+C\right]=\frac{1}{\sqrt{x^2+1}+x}\cdot\left( \frac{2x}{2\sqrt{x^2+1}}+1\right)\\
&=\frac{x+\sqrt{x^2+1}}{(\sqrt{x^2+1}+x)\sqrt{x^2+1}}=\frac{1}{\sqrt{x^2+1}}
\end{align*}
So, our answer works! }

%----------------------------------------------------------------------------------------
\begin{frame}{Method (one standard case)}
\StatusBar{1}{9}
\begin{itemize}[<+->]
\item An integrand has the form: \textcolor{W1}{$\sqrt{\text{quadratic}}$}, and we'd like to cancel off the square root.
\vfill
\item So, we need to write our quadratic expression as a perfect square. Choose a helpful substitution:
\vfill
\begin{itemize}
\item \textcolor{M3}{$x=\sin\theta$}, $1-\sin^2\theta=\cos^2 \theta$ changes \textcolor{W1}{$\sqrt{1-x^2}$} into \onslide<+-|handout:0>{ \textcolor{W1}{$\sqrt{\cos^2\theta}=|\cos \theta|$}}
\vfill
\item \textcolor{M3}{$x=\tan\theta$}, $1+\tan^2\theta=\sec^2 \theta$ changes \textcolor{W1}{$\sqrt{1+x^2}$} into  \onslide<+-|handout:0>{\textcolor{W1}{$\sqrt{\sec^2\theta}=|\sec \theta|$}}
\vfill
\item \textcolor{M3}{$x=\sec\theta$}, $\sec^2\theta-1=\tan^2 \theta$ changes \textcolor{W1}{$\sqrt{x^2-1}$} into \onslide<+-|handout:0>{\textcolor{W1}{$\sqrt{\tan^2\theta}=|\tan\theta|$}}
%\vfill
%\item Fine print: you may assume for this method that $\sqrt{\cos^2\theta}=\cos\theta$, 
%$\sqrt{\sec^2\theta}=\sec\theta$, and $\sqrt{\tan^2\theta}=\tan\theta$.
\end{itemize}
\vfill
\item After integrating, convert back to the original variable\\ (possibly using a triangle--more details later)
\end{itemize}
\end{frame}
%----------------------------------------------------------------------------------------
%----------------------------------------------------------------------------------------
\begin{frame}[t]{Focus on the Algebra}
\AnswerYes<1-3>
\nsNoSpace<1>
\textcolor{W2}{$1-\sin^2\theta=\cos^2\theta $ \hfill $1+\tan^2\theta=\sec^2\theta $ \hfill $
\sec^2\theta-1=\tan^2\theta $}\vfill


\onslide<+->{Choose a trigonometric substitution that will allow the square root to cancel out of the following expressions:}
\begin{itemize}
	\item $\sqrt{x^2-1}$\\
	\sonslide<+->{Let $\textcolor{M3}{x=\sec \theta}$, so $\sqrt{\textcolor{M3}x^2-1}$ becomes $\sqrt {\textcolor{M3}\sec^2\textcolor{M3} \theta - 1} = \sqrt{\tan^2 \theta} = |\tan\theta|$}
	\vfill 
	
	\item $\sqrt{x^2+1}$\\
	\sonslide<+->{Let $\textcolor{M3}{x=\tan\theta}$, so $\sqrt{\textcolor{M3}x^2+1}$ becomes 
	$\sqrt{\textcolor{M3}\tan^2 \textcolor{M3}\theta + 1}=\sqrt{\sec^2 \theta} = |\sec\theta|$}
	\vfill
	
	\item $\sqrt{1-x^2}$\\
	\sonslide<+->{Let $\textcolor{M3}{x=\sin \theta}$ so $\sqrt{1-\textcolor{M3}x^2}$ becomes 
	$\sqrt{1-\textcolor{M3}\sin^2\textcolor{M3} \theta} = \sqrt{\cos^2 \theta} = |\cos \theta|$\\
	(Alternately, $x=\cos \theta$ works as well)}
	\vfill
\end{itemize}

\end{frame}
%----------------------------------------------------------------------------------------
%----------------------------------------------------------------------------------------
\begin{frame}[t]{Focus on the Algebra}
\AnswerYes<1-2>
\sNoSpace<1-2>
\nsNoSpace<1>
\textcolor{W2}{$1-\sin^2\theta=\cos^2\theta $ \hfill $1+\tan^2\theta=\sec^2\theta $ \hfill $
\sec^2\theta-1=\tan^2\theta $}\vfill


\onslide<+->{Choose a trigonometric substitution that will allow the square root to cancel out of the following expressions:}
\begin{itemize}
\item $\sqrt{x^2+7}$\\
	\sonslide<+->{Adjust a given identity by multiplying both sides by 7: $7\tan^2 \theta + 7 = 7\sec^2 \theta$. Now we see we want $\textcolor{M3}{x^2 = 7\tan^2 \theta}$. That is, $x=\sqrt{7}\tan \theta$:\\
	$\sqrt{\textcolor{M3}{x^2}+7}=\sqrt{\textcolor{M3}{7\tan^2\theta}+7}=\sqrt{7(\sec^2\theta)} = \sqrt{7}\ |\sec\theta|$}
	\vfill 
	
\item $\sqrt{3-2x^2}$\\
	\sonslide<+->{
	Adjust a given identity by multiplying both sides by 3: $3-3\sin^2\theta=3\cos^2\theta$. Now we see we want 
	$\textcolor{M3}{2x^2=3\sin^2\theta}$, so $x=\sqrt{\frac32}\sin\theta$:\\
	 $\sqrt{3-\textcolor{M3}{2x^2}} = \sqrt{3-\textcolor{M3}{2\left(\frac32\sin^2\theta\right)}}=\sqrt{3-3\sin^2\theta}=\sqrt{3\cos^2\theta}=\sqrt{3}\ |\cos\theta|$
	}
	\vfill
\end{itemize}
\end{frame}
%----------------------------------------------------------------------------------------


%----------------------------------------------------------------------------------------
\begin{frame}[t]{Closer look at absolute values \hfill \hyperlink{EndOfAbsValueDiscussion}{\beamerskipbutton{skip closer look}}}
\only<1>{\AnswerYes\NoSpace\label{note1.9a}}\AnswerSpace
Consider the substitution $\textcolor{M3}{x=\sin \theta}$, $\textcolor{C3}{\dee x = \cos \theta \ \dee \theta}$ for the integral:
\[\int_0^1 \sqrt{1-x^2}\ \dee x\]

When $x=0$ (lower limit of integration), what is $\theta$?\\
When $x=1$ (upper limit of integration), what is $\theta$? 

\sonly<2->{\vfill\small
	If $x=0$, then $\sin \theta = 0$, but there are infinitely many values of $\theta$ that could make this true. To use the substitution $x=\sin\theta$, we need the function $x=\sin \theta$ to be invertible. That way, we can unambiguously convert between $x$ and $\theta$. With that in mind, we'll actually set $\theta = \arcsin x$. Now $\theta$ is restricted to the interval $-\frac{\pi}{2} \le \theta \le \frac{\pi}{2}$.
	
\begin{align*}\int_{\textcolor{M3}0}^{\textcolor{M3}1} \sqrt{1-\textcolor{M3}x^2}\ \textcolor{C3}{\dee x} &= \int_{\textcolor{M3}{\arcsin 0}}^{\textcolor{M3}{\arcsin 1}} \sqrt{1-\textcolor{M3}{\sin}^2\textcolor{M3}{\theta}}\,\textcolor{C3}{\cos \theta\ \dee \theta} = \int_0^\frac{\pi}{2}\sqrt{\cos^2\theta}\cdot\cos \theta \ \dee \theta\\
&=\int_0^{\frac{\pi}{2}}|\cos \theta| \cdot \cos \theta\ \dee \theta
\end{align*}
For $0 \le \theta \le \frac{\pi}{2}$, we have $\cos \theta \ge 0$, so $|\cos \theta|=\cos \theta$.
	}
\end{frame}	
%----------------------------------------------------------------------------------------
\begin{frame}[t]{Closer look at absolute values\hfill \hyperlink{EndOfAbsValueDiscussion}{\beamerskipbutton{skip closer look}}}
\StatusBar{1}{5}
More generally, suppose $a$ is a positive constant and we use the substitution $x= a \sin \theta$ for the term $\sqrt{a^2-x^2}$.\pause
\begin{itemize}[<+-|handout:0>]\color{spoilercolor}
\item $\theta = \arcsin\left(\frac{x}{a}\right)$, so $-\frac{\pi}{2}\le \theta \le \frac{\pi}{2}$
\item $\sqrt{a^2-x^2}=\sqrt{a^2-a^2\sin^2\theta}=\sqrt{a^2\cos^2\theta}=a|\cos\theta|$
\item On the interval $-\frac{\pi}{2} \le \theta \le \frac{\pi}{2}$, $\cos\theta \ge 0$, so $|\cos \theta|=\cos\theta$
\begin{center}
\begin{tikzpicture}
\myaxis{\theta}{3.5}{3.5}{y}{1}{1.25}
\draw[C1,thick] plot[smooth,domain=-3.25:3.25](\x,{cos(\x r)});
\xcoord{1.57}{\frac{\pi}{2}}
\xcoord{-1.57}{-\frac{\pi}{2}}
\end{tikzpicture}
\end{center}
\item So, in general, when we use the substitution $x=\sin\theta$ with trigonometric substitution, we can expect $|\cos \theta| = \cos \theta$.
\end{itemize}
\end{frame}	
%%----------------------------------------------------------------------------------------
\begin{frame}[t]{Closer look at absolute values\hfill \hyperlink{EndOfAbsValueDiscussion}{\beamerskipbutton{skip closer look}}}
\StatusBar{1}{5}
Now, consider the substitution $x=a\tan \theta$ for $\sqrt{a^2+x^2}$, where $a$ is a positive constant.\pause
\begin{itemize}[<+-|handout:0>]\color{spoilercolor}
\item $\theta = \arctan\left(\frac{x}{a}\right)$, so $-\frac{\pi}{2}\le \theta \le \frac{\pi}{2}$
\item $\sqrt{a^2+x^2}=\sqrt{a^2+a^2\tan^2\theta}=\sqrt{a^2\sec^2\theta}=\frac{a}{|\cos \theta|}$
\item On the interval $-\frac{\pi}{2} \le \theta \le \frac{\pi}{2}$, $\cos\theta \ge 0$, so $|\cos \theta|=\cos\theta$ and $|\sec \theta| = \sec \theta$.
\begin{center}
\begin{tikzpicture}
\myaxis{\theta}{3.5}{3.5}{y}{1}{1.25}
\draw[C1,thick] plot[smooth,domain=-3.25:3.25](\x,{cos(\x r)});
\xcoord{1.57}{\frac{\pi}{2}}
\xcoord{-1.57}{-\frac{\pi}{2}}
\end{tikzpicture}
\end{center}
\item So, in general, when we use the substitution $x=\tan\theta$ with trigonometric substitution, we can expect $|\sec \theta| = \sec \theta$.
\end{itemize}
\end{frame}
%%----------------------------------------------------------------------------------------%%----------------------------------------------------------------------------------------
\begin{frame}[t]{Closer look at absolute values\hfill \hyperlink{EndOfAbsValueDiscussion}{\beamerskipbutton{skip closer look}}}
\StatusBar{1}{6}
Finally, consider the substitution $x=a\sec \theta$ for $\sqrt{x^2-a^2}$, where $a$ is a positive constant.\pause
\begin{itemize}[<+-|handout:0>]\color{spoilercolor}
\item $\sec \theta = \frac{x}{a}$, so $\cos \theta=\frac{a}{x}$, so $\theta= \arccos\left(\frac{a}{x}\right)$. Then $0 \le \theta \le \pi$
\item $\sqrt{x^2-a^2}=\sqrt{a^2\sec^2\theta-a^2}=\sqrt{a^2\tan^2\theta}=a|\tan\theta|$
\item Now this case gets slightly more complicated than the others:
\begin{itemize}
\item For $\sqrt{x^2-a^2}$ to be defined, we need $x^2\ge a^2$. I.e. $x\ge a$ or $x\le -a$.
\item When $x\ge a$, we have $0 \le \theta <\frac{\pi}{2}$, $\tan\theta \ge 0$, so $|\tan \theta|=\tan\theta$.
\item When $x\le -a$, we have $\frac{\pi}{2}< \theta \le \pi$, $\tan\theta<0$, so $|\tan \theta|=-\tan\theta$.
\begin{tikzpicture}[scale=0.7]
\myaxis{\theta}{0}{3.3}{y}{1.5}{1.25}
\draw[C1,very thick] plot[smooth,domain=0:1.05](\x,{tan(\x r)});
\draw[W1,very thick] plot[smooth,domain=2.1:3.14](\x,{tan(\x r)});
\draw(1.5,-1.75)node[below,C1]{$y=\tan\theta$};
\xcoord{1.57}{\frac{\pi}{2}}
\xcoord{3.14}{\pi}
%
\begin{scope}[xshift=5cm]
\myaxis{\theta}{0}{3.5}{y}{0}{1.25}
\draw[C3,very thick] plot[smooth,domain=0:1.05](\x,{abs(tan(\x r))});
\draw[C3,very thick] plot[smooth,domain=2.1:3.14](\x,{abs(tan(\x r))});
\draw (1.57,-1.75)node[C3,below]{$y=\sqrt{\tan^2\theta}=|\tan\theta|$};
\xcoord{1.57}{\frac{\pi}{2}}
\xcoord{3.14}{\pi}
\end{scope}
\end{tikzpicture}
\end{itemize}
\end{itemize}
\end{frame}
%%----------------------------------------------------------------------------------------

\begin{frame}[t,label=EndOfAbsValueDiscussion]{Absolute Values}
From now on, we will assume:
\begin{itemize}
\item With the substitution $x=a \sin \theta$ for $\sqrt{a^2-x^2}$,~ $|\cos \theta|=\cos \theta$
\item With the substitution $x=a \tan \theta$ for $\sqrt{a^2+x^2}$,~ $|\sec \theta|=\sec \theta$
\end{itemize}
\end{frame}
%%----------------------------------------------------------------------------------------
%%----------------------------------------------------------------------------------------
\begin{frame}[t]
\only<1>{
\MoreSpace
\begin{block}{Identities}
\begin{align*}
1-\sin^2 \theta & = \cos^2 \theta & \sin(2\theta)&=2\sin\theta\cos\theta\\
1+\tan^2\theta & = \sec^2 \theta & \sin^2 \theta &= \frac{1-\cos(2\theta)}{2}\\
\sec^2\theta-1&=\tan^2\theta & \cos^2\theta &= \frac{1+\cos(2\theta)}{2}
\end{align*}
\end{block}}
\only<2>{\AnswerYes}
Evaluate $\ds\int_0^1(1+x^2)^{-3/2}~\dee x$

\sonslide<3->{Let $\textcolor{M3}{x=\tan\theta}$, $\textcolor{C3}{\dee x = \sec^2 \theta\ \dee \theta}$.
When $x=0$, then $\theta = \arctan{0}=0$; when $x=1$, then $\theta = \arctan 1 = \frac{\pi}{4}$.
	\begin{align*}
	\int_{\textcolor{M3}0}^{\textcolor{M3}1}(1+\textcolor{M3}x^2)^{-3/2}~\textcolor{C3}{\dee x}
	&=\int_{{\color{M3}\theta=0}}^{\color{M3}\theta=\pi/4} \frac{1}{\sqrt{1+\textcolor{M3}\tan^2\textcolor{M3}\theta}^3}\color{C3}\sec^2\theta \ \dee \theta\\
	&=\int_0^{\pi/4}\frac{\sec^2 \theta}{\sqrt{\sec^2\theta}^3}\dee\theta
	=\int_0^{\pi/4} \frac{\sec^2\theta}{|\sec \theta|^3}\dee\theta\\
	&=\int_0^{\pi/4} \frac{1}{|\sec \theta|}\ \dee \theta
	=\int_0^{\pi/4} |\cos \theta|\ \dee \theta
	\intertext{Given our previous investigation,}
	&=\int_0^{\pi/4} \cos \theta\ \dee\theta=\big[\sin \theta\big]_0^{\pi/4}\\
	&=\sin\frac{\pi}{4}-\sin 0 = \frac{1}{\sqrt 2}
	\end{align*}
	}
\end{frame}
%----------------------------------------------------------------------------------------


\begin{frame}[t]
\only<1>{
\MoreSpace
\begin{block}{Identities}
\begin{align*}
1-\sin^2 \theta & = \cos^2 \theta & \sin(2\theta)&=\cos\theta\\
1+\tan^2\theta & = \sec^2 \theta & \sin^2 \theta &= \frac{1-\cos(2\theta)}{2}\\
\sec^2\theta-1&=\tan^2\theta & \cos^2\theta &= \frac{1+\cos(2\theta)}{2}
\end{align*}
\end{block}}
\only<2>{\AnswerYes}
Evaluate $\ds\int \sqrt{1-4x^2}~\dee x$

\color{C1} 
\sonly<3>{Under the square root, we have ``one minus a term with a variable," which matches the identity $1-\sin^2\theta$. So, we want $4x^2$ to become $\sin^2 \theta$. That is, $\textcolor{M3}{x=\frac12\sin\theta}$. Then $\textcolor{C3}{\dee x = \frac12\cos \theta \ \dee \theta}$.
\begin{align*}
\int \sqrt{1-4\textcolor{M3}x^2}~\color{C3}\dee x
&=\int \sqrt{1-4\left(\textcolor{M3}{\frac12\sin\theta}\right)^2}\cdot\textcolor{C3}{\frac12 \cos \theta \ \dee \theta}\\
&=\frac12\int\sqrt{1-\sin^2\theta}\cdot\cos\theta \dee \theta
=\frac12\int\sqrt{\cos^2\theta}\cdot\cos\theta \dee \theta\\
&=\frac12\int|\cos \theta|\cdot\cos\theta \ \dee \theta
=\frac12\int\cos^2\theta \ \dee \theta\\
&=\frac12\int\left(\frac{1+\cos(2\theta)}{2}\right)\dee\theta=\frac14\int\big(1+\cos(2\theta)\big)\dee\theta\\
&=\frac14\left[\theta+\frac12\sin(2\theta)\right]+C=\frac14\left[\theta+\sin\theta\cos\theta\right]+C
\end{align*}
It remains to convert $\theta$ back into $x$.
	}
\sonly<4>{\small
The substitution $x=\frac12\sin\theta$ tells us $\sin\theta=2x$. This in turn gives us $\theta = \arcsin(2x)$. We should still convert $\cos \theta$ back into terms of $x$. You might notice in the calculation we did that $\sqrt{1-4x^2}$ turned into $\cos\theta$, so $\cos\theta=\sqrt{1-4x^2}$. \vfill

Alternately, to find $\cos\theta$ in terms of $x$, we can use a triangle. From $\sin\theta = 2x$, and the understanding that $\sin\theta$ is the ratio $\frac{\text{opposite}}{\text{hypotenuse}}$ for a right triangle with angle $\theta$, we can set up a triangle whose opposite side has length $2x$, and hypotenuse has length $1$.
\begin{multicols}{2}
\TrigTri{\theta}{\sqrt{1-4x^2}}{2x}{1}

The Pythagorean theorem tells us the side adjacent to $\theta$ has length $\sqrt{1-4x^2}$. So $\cos\theta = \frac{\text{adjacent}}{\text{hypotenuse}}=\sqrt{1-4x^2}$.
\end{multicols}
 $\int\sqrt{1-4x^2}\ \dee x =
  \frac14\Big(\underbrace{\arcsin(2x)}_{\theta}+\underbrace{2x\sqrt{1-4x^2}}_{\sin\theta \cos \theta}\Big)+C$
}
\end{frame}
%%----------------------------------------------------------------------------------------

\CheckFrame{
In the last example, we computed
\[\int\sqrt{1-4x^2}\ \dee x = \onslide<beamer>{\frac14\big(\arcsin(2x)+2x\sqrt{1-4x^2}\big)+C.}\]
To check, we differentiate.
}{
\begin{align*}
\diff{}{x}&\left\{\frac14\big(\arcsin(2x)+2x\sqrt{1-4x^2}\big)+C \right\}\\
=&\frac14\left( \frac{2}{\sqrt{1-(2x)^2}}+2x\frac{-8x}{2\sqrt{1-4x^2}}+2\sqrt{1-4x^2} \right)\\
=&\frac14\left(\frac{2}{\sqrt{1-4x^2}}-\frac{8x^2}{\sqrt{1-4x^2}}+\frac{2(1-4x^2)}{\sqrt{1-4x^2}}\right)\\
&=\frac14\left(\frac{2-8x^2+2-8x^2}{\sqrt{1-4x^2}}\right)=\frac{1-4x^2}{\sqrt{1-4x^2}}=\sqrt{1-4x^2}\qquad \checkmark
\end{align*}}
%----------------------------------------------------------------------------------------
%----------------------------------------------------------------------------------------


\begin{frame}[t]
\only<1>{
\MoreSpace
\begin{block}{Identities}
\begin{align*}
1-\sin^2 \theta & = \cos^2 \theta & \sin(2\theta)&=\cos\theta\\
1+\tan^2\theta & = \sec^2 \theta & \sin^2 \theta &= \frac{1-\cos(2\theta)}{2}\\
\sec^2\theta-1&=\tan^2\theta & \cos^2\theta &= \frac{1+\cos(2\theta)}{2}
\end{align*}
\end{block}}
\only<2>{\AnswerYes}
\only<1-2>{Evaluate $\ds\int \frac{1}{\sqrt{x^2-1}}~\dee x$}


\sonly<3>{We use the substitution $\textcolor{M3}{x=\sec \theta}$, $\textcolor{C3}{\dee x = \sec \theta \tan \theta \ \dee \theta}$.

To make the substitution work, we're actually taking $\theta = \arccos\left(\frac1x\right)$, and so $0 \le \theta \le \pi$.\\
Note that the integrand exists on the intervals $x<-1$ and $x>1$. 
\begin{itemize}\color{C1}
\item When $x>1$,  then $0 < \frac{1}{x} <1$, so $0<\arccos{\left(\frac1x\right)}<\frac{\pi}{2}$.\\
That is, $0 < \theta < \frac{\pi}{2}$, so $|\tan\theta|=\tan\theta$.
\item When $x<-1$,  then $-1 < \frac{1}{x} <0$, so $\frac{\pi}{2}<\arccos{\left(\frac1x\right)}<\pi$.\\
That is, $\frac{\pi}{2}<\theta<\pi$, so $|\tan \theta|=-\tan\theta$.
\end{itemize}

\begin{align*}
\int\frac{1}{\sqrt{\textcolor{M3}x^2-1}}\textcolor{C3}{\dee x}
 & = \int\frac{1}{\sqrt{\textcolor{M3}\sec^2\textcolor{M3}\theta-1}}\cdot\textcolor{C3}{\sec\theta\tan\theta \ \dee \theta}=\int\frac{\sec \theta \tan \theta}{\sqrt{\tan^2\theta}}\dee\theta\\
&=\int\sec\theta\left(\frac{\tan\theta}{|\tan\theta|}\right)\dee\theta=\begin{cases}
\int\sec \theta \ \dee \theta & 0<\theta<\frac{\pi}{2}\\
-\int\sec \theta \ \dee \theta & \frac{\pi}{2}<\theta<\pi
\end{cases}\\
&=\begin{cases}
\log|\sec \theta + \tan \theta|+C & 0<\theta<\frac{\pi}{2}\\
-\log|\sec \theta + \tan \theta|+C & \frac{\pi}{2}<\theta<\pi
\end{cases}
\end{align*}
	}
\sonly<4>{
Our substitution tells us $\sec\theta = x$. We saw from the denominator of our integrand that $\sqrt{x^2-1}=|\tan \theta|$.
\begin{itemize}\color{spoilercolor}
\item When $0 < \theta < \frac{\pi}{2}$, $\tan\theta=|\tan\theta|=\sqrt{x^2-1}$
\item When $\frac{\pi}{2} < \theta < \pi$, $\tan\theta=-|\tan\theta|=-\sqrt{x^2-1}$
\end{itemize}
\[
\int\frac{1}{\sqrt{x^2-1}}\dee x = \begin{cases}
\log|x+ \sqrt{x^2-1}|+C & x>1\\
-\log|x -\sqrt{x^2-1}|+C & x<-1\\
\end{cases}
\]
Although the two branches look different, they are actually equivalent. Remember $-\log(A) =\log\left(A^{-1}\right)  =\log\left(\frac1A\right) $:
\begin{align*}
-\log|x-\sqrt{x^2-1}|&=\log\left| \frac{1}{x-\sqrt{x^2-1}}\right|=\log\left| \frac{1}{x-\sqrt{x^2-1}}\cdot\frac{x+\sqrt{x^2-1}}{x+\sqrt{x^2-1}}\right|\\
&=\log\left|\frac{x+\sqrt{x^2-1}}{x^2-x^2+1} \right|=\log\left|x+\sqrt{x^2-1}\right|
\end{align*}
So,
\[\int\frac{1}{\sqrt{x^2-1}}\dee x = \log\left|x+\sqrt{x^2-1}\right|+C\]
}
\end{frame}
%%----------------------------------------------------------------------------------------
\CheckFrame{Let's check our result, $\ds\int\frac{1}{\sqrt{x^2-1}}\dee x =\onslide<beamer>{ \log\left|x+\sqrt{x^2-1}\right|+C.}$}{
\begin{align*}
&\diff{}{x}\left\{\log\left|x+\sqrt{x^2-1}\right|+C\right\}=\frac{1+\frac{2x}{2\sqrt{x^2-1}}}{x+\sqrt{x^2-1}}
=\frac{1+\frac{x}{\sqrt{x^2-1}}}{x+\sqrt{x^2-1}}\\
&=\frac{1+\frac{x}{\sqrt{x^2-1}}}{x+\sqrt{x^2-1}}\left(\frac{\sqrt{x^2-1}}{\sqrt{x^2-1}}\right)
=\frac{(\sqrt{x^2-1}+x)}{\left(x+\sqrt{x^2-1}\right)\sqrt{x^2-1}}\\
&=\frac{1}{\sqrt{x^2-1}}
\end{align*}
So, our answer works.}
%%----------------------------------------------------------------------------------------
%----------------------------------------------------------------------------------------

%----------------------------------------------------------------------------------------
\section{Completing the Square}

%----------------------------------------------------------------------------------------
\begin{frame}[t]{Completing the Square}
\label{note1.9b}
\sStatusBar{1}{4}
Choose a trigonometric substitution to simplify $\sqrt{3-x^2+2x}$.\\[1em]

Identities have two ``parts" that turn into one part:
 \begin{itemize}\color{W2}
\item $1-\sin^2\theta=\cos^2\theta$ \hspace{1cm}\sonslide<3->{$4-4\sin^2\theta=4\cos^2\theta$}
\item $1+\tan^2\theta=\sec^2\theta$
\item $\sec^2\theta-1=\tan^2\theta$
\end{itemize}
But our quadratic expression has \textit{three} parts.\\
\pause
Fact: \quad $3-x^2+2x=4-(x-1)^2$\hfill~

\sonslide<4->{
\begin{align*}
\sqrt{3-x^2+2x}&=\sqrt{4-(x-1)^2} \\
\intertext{We want $(x-1)^2=4\sin^2\theta$, \quad so let $\textcolor{M3}{(x-1)=2\sin\theta}$}
&=\sqrt{4-4\sin^2\theta}=\sqrt{4\cos^2\theta}=2\cos\theta
\end{align*}
}
\end{frame}
%----------------------------------------------------------------------------------------
%----------------------------------------------------------------------------------------
\begin{frame}[t]{Completing the Square}
\StatusBar{1}{7}
\AnswerYes<2,4,6>
\nsAnswerYes<2,4,6>
{\color{W1}\begin{align*}
(x+b)^2&=x^2+2bx+b^2\\
c-(x+b)^2&=(c-b^2)-x^2-2bx\\
\end{align*}}
Write $3-x^2+2x$ in the form $c-(x+b)^2$ for constants $b$, $c$.\pause\vfill
\begin{enumerate}[<+->]
	\item Find $b$:  \qquad\onslide<+-|handout:0>{\color{spoilercolor}$-2bx=2x$, so $b=-1$}\vfill
	\item Solve for $c$: \qquad\onslide<+-|handout:0>{\color{spoilercolor} $3=c-b^2=c-1$, so $c=4$}\vfill
	\item All together: \qquad\onslide<+-|handout:0>{\color{spoilercolor}  $3-x^2+2x=4-(x-1)^2$}\vfill
\end{enumerate}

\end{frame}
%----------------------------------------------------------------------------------------
%----------------------------------------------------------------------------------------
\begin{frame}[t]
\AnswerYes<1>
\sMoreSpace<2>
\nsMoreSpace<1>

Evaluate $\displaystyle\int\frac{x^2-6x+9}{\sqrt{6x-x^2}}~\dee x$.\\[1em]

 Identities have two ``parts" that turn into one part:
 \begin{itemize}\color{W2}
\item $1-\sin^2\theta=\cos^2\theta$
\item $1+\tan^2\theta=\sec^2\theta$
\item $\sec^2\theta-1=\tan^2\theta$
\end{itemize}
One of those parts is a constant, and one is squared.

\sonslide<2->{
Write $\textcolor{W1}{6x-x^2}$ as $\textcolor{W1}{c-(x+b)^2}$. \\ 
\begin{align*}
\color{W2}c-(x+b)^2&=\color{W2}(c-b^2)-x^2-2bx\\
6x&=-2bx \implies b=-3\\
0&=c-b^2=c-9 \implies c=9\\
\color{W1}6x-x^2&=\color{W1}9-(x-3)^2
\end{align*}
}
\end{frame}
%----------------------------------------------------------------------------------------
\begin{frame}[t]
\AnswerSpace\only<1>{\AnswerYes}
Evaluate $\displaystyle\int\frac{x^2-6x+9}{\sqrt{6x-x^2}}~\dee x ~=~\int\frac{(x-3)^2}{\sqrt{9-(x-3)^2}}\dee x$.\\[1em]
\sonslide<2->{
We use the identity $9-9\sin^2\theta = 9\cos^2\theta$.\\
 We want $(x-3)^2=9\sin^2\theta$, so take $\textcolor{M3}{(x-3)=3\sin\theta}$, $\textcolor{C3}{\dee x = 3\cos \theta\ \dee\theta}$.
\begin{align*}
\int\frac{\textcolor{M3}{(x-3)^2}}{\sqrt{9-\textcolor{M3}{(x-3)^2}}}\textcolor{C3}{\dee x}
&=\int\frac{\textcolor{M3}{9\sin^2\theta}}{\sqrt{9-\textcolor{M3}{9\sin^2\theta}}}\,\textcolor{C3}{3\cos\theta\ \dee \theta}\\
&=\int\frac{9\sin^2\theta}{\sqrt{9\cos^2\theta}}\,3\cos\theta\ \dee \theta=\int9\sin^2\theta\ \dee \theta\\
&=\frac{9}{2}\int(1-\cos2\theta)\ \dee \theta=\frac92\left(\theta-\frac12\sin 2\theta \right)+C\\
&=\frac{9}{2}\left( \theta-\sin\theta\cos\theta\right)+C\\
\smash{\TrigTri{\theta}{\sqrt{6x-x^2}}{x-3}{3}}\quad
&=\frac92\left( \arcsin\left( \frac{x-3}{3}\right)-\frac{x-3}{3}\cdot\frac{\sqrt{6x-x^2}}{3}\right)+C
\end{align*}
}

\end{frame}
%----------------------------------------------------------------------------------------
\CheckFrame{
Let's verify that $\ds\int\frac{x^2-6x+9}{\sqrt{6x-x^2}}=\onslide<beamer>{\frac92\left( \arcsin\left( \frac{x-3}{3}\right)-\frac{x-3}{3}\cdot\frac{\sqrt{6x-x^2}}{3}\right)+C:$}}{
\begin{align*}
\diff{}{x}&\left\{ \frac92\left( \arcsin\left( \frac{x-3}{3}\right)-\frac{x-3}{3}\cdot\frac{\sqrt{6x-x^2}}{3}\right)+C\right\}\\
&=  \frac92\left( \frac{1/3}{\sqrt{1-\left( \frac{x-3}{3}\right)^2}}-\frac{x-3}{3}\cdot\frac{3-x}{3\sqrt{6x-x^2}}-\frac19\sqrt{6x-x^2}\right)\\
&=  \frac92\left( \frac{9}{9\sqrt{6x-x^2}}-\frac{6x-x^2-9}{9\sqrt{6x-x^2}}-\frac{6x-x^2}{9\sqrt{6x-x^2}}\right)
\\&=   \frac{9-6x+x^2}{\sqrt{6x-x^2}}\end{align*}
So, our answer works.
}
%----------------------------------------------------------------------------------------
%----------------------------------------------------------------------------------------

