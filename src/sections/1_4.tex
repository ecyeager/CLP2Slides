% Copyright 2021 Joel Feldman, Andrew Rechnitzer and Elyse Yeager, except where noted.
% This work is licensed under a Creative Commons Attribution-NonCommercial-ShareAlike 4.0 International License.
% https://creativecommons.org/licenses/by-nc-sa/4.0/


 \begin{frame}{Table of Contents }
\mapofcontentsA{\ad,\atech}
 \end{frame}
%----------------------------------------------------------------------------------------
%----------------------------------------------------------------------------------------
\section{1.4: Substitution}
%----------------------------------------------------------------------------------------
\begin{frame}{Antiderivatives}
\StatusBar{1}{5}
Fact:
\[\diff{}{x}\left\{\sin\left(x^2+x \right) \right\}=\pause \only<beamer>{(2x+1)\cos(x^2+x)}\]\pause\vfill
Related Fact:
\[\int(2x+1)\cos(x^2+x)\,\dee x =\onslide<4-|handout:0>{ \sin\left(x^2+x \right)+C} \]\vfill
\onslide<5-|handout:0>{Hard to guess the antiderivative without seeing the derivative first!}
\end{frame}
%----------------------------------------------------------------------------------------
\begin{frame}{Antiderivatives}
\StatusBar{1}{3}
Chain Rule:
\[\diff{}{x}
\left\{\sin\Big(\underbrace{\sdu{x^2+x}}_{\mathclap{\text{ inside function}}}\Big) ~~ \right\}=\pause\Big(\underbrace{\sdu{2x+1}}_{\mathclap{\text{\parbox{1.75cm}{derivative of \\[-1mm]inside function}}}}\Big)\cos\Big(\underbrace{\sdu{x^2+x}}_{\text{\makebox[0pt]{ inside function}}}\Big)
\]\pause\vfill
Hallmark of the chain rule: an ``inside" function, with that function's derivative multiplied.\\ 
\end{frame}
%----------------------------------------------------------------------------------------
\begin{frame}{Solve by Inspection}
$\ds \int 2xe^{x^2+1}\ \dee x$ \sonslide<2->{$=e^{x^2+1}+C$}\vfill
$\ds\int \frac1x \cos(\log x)\ \dee x $ \sonslide<2->{$=\sin(\log x)+C$}\vfill
$\ds\int 3(\sin x+1)^2 \cos x\ \dee x $ \sonslide<2->{$=(\sin x + 1)^3+C$} \vfill
(Look for an ``inside" function, with its derivative multiplied.)
\AnswerYes<1>
\end{frame}
%----------------------------------------------------------------------------------------
%----------------------------------------------------------------------------------------
\begin{frame}[t]{Undoing the Chain Rule}
Chain Rule:
\[\diff{}{x}\left\{ f(u(x))\right\} = f'(u(x))\cdot u'(x) \]
(Here, $u(x)$ is our ``inside function")\vfill
Antiderivative Fact:
\[\int f'(u(x))\cdot u'(x)\ \dee x=f(u(x))+C \]
\end{frame}
%----------------------------------------------------------------------------------------
\begin{frame}{Undoing the Chain Rule}
Antiderivative Fact:
\[\int f'(u(x))\cdot u'(x)\ \dee x=f(u(x))+C \]\vfill\pause

Shorthand: call $\su{u(x)}$ simply $\su{}$;\\
since $\diff{u}{x}=u'(x)$, call $\sdu{u'(x)\ \dee x}$ simply $\sdu{}$.\pause\vfill
\[\int f'(\su{u(x)})\cdot \sdu{u'(x)\ \dee x} = \int f'(\su{})\ \sdu{}\Big|_{\su{u=u(x)}}=f\big(\su{u(x)}\big)+C\]\vfill
This is the \alert{substitution rule}.
\end{frame}
%%----------------------------------------------------------------------------------------
%----------------------------------------------------------------------------------------
\begin{frame}[t]
We saw these integrals before, and solved them by inspection. Now try using the language of substitution.
\vfill

$\ds \int 2xe^{x^2+1}\, \dee x$

 \sonslide<2->{\vfill Using $\su{}$ as shorthand for $\su{x^2+1}$, and $\sdu{}$ as shorthand for $\sdu{2x\,\dee x}$:
 $\int \sdu{2x}e^{\su{x^2+1}}\, \sdu{\dee x}=\int e^{\su{}}\,\sdu{\dee u }= e^{\su{}}+C
 = e^{\su{x^2+1}}+C$}
 \vfill
 
$\ds\int \frac1x \cos(\log x)\, \dee x $ 

\sonslide<2->{\vfill
Using $\su{}$ as shorthand for $\su{\log x}$, and $\sdu{}$ as shorthand for $\sdu{\frac1x\,\dee x}$:
$\int \sdu{\frac1x} \cos(\su{\log x})\, \sdu{\dee x} = 
\int \cos(\su{})\,\sdu{}=\sin(\su{})+C=\sin(\su{\log x})+C$ 
}
\vfill
$\ds\int 3(\sin x+1)^2 \cos x\, \dee x $

 \sonslide<2->{\vfill Using $\su{}$ as shorthand for $\su{\sin x+1}$, and $\sdu{}$ as shorthand for $\sdu{\cos x\,\dee x}$:

$\int 3(\su{\sin x+1})^2 \sdu{\cos x\, \dee x}
=\int 3\su{}^2\,\sdu{}
=\su{}^3+C
=(\su{\sin x + 1})^3+C$} \vfill

\only<1>{\AnswerYes\NoSpace}
\end{frame}

%----------------------------------------------------------------------------------------
%\subsection{Examples Using the Substitution Rule}
%----------------------------------------------------------------------------------------
\begin{frame}[t]
\only<-6>{\AnswerYes}
\StatusBar{1}{9}
\begin{align*}
\int (\sdu<3,5-|handout:0>{3x^2})\sin(\su<2,5-|handout:0>{x^3+1})\,\sdu<5-|handout:0>{\dee x} &=
\onslide<5-|handout:2->{\left.\int \sin(\su{})\,\sdu{\dee u }\right|_{\su{u=x^3+1}}
\\}
\onslide<6-|handout:3>{&=\left.-\cos(\su{})+C\right|_{\su{u=x^3+1}}\\}
\onslide<7-|handout:3>{&=\cos(\su{x^3+1})+C}
\end{align*}

\onslide<2-|handout:2->{``Inside" function: \su{$x^3+1$}.} \onslide<3-|handout:2->{Its derivative: \sdu{$3x^2$}}

\onslide<4-|handout:2->{Shorthand: $\su{x^3+1 \to u}$,\quad $\sdu{3x^2\,\dee x \to \dee u}$}
\vfill

\color{W1}
\begin{overlayarea}{\textwidth}{6 em}
\only<8|handout:3>{Warning 1: We don't just change $ \dee x$ to $ \dee u$. We need to couple $ \dee x$ with the derivative of our inside function. 

After all, we're undoing the chain rule! We need to have an ``inside derivative."
\vspace{1em}}

\only<9|handout:3>{Warning 2: The final answer is a function of $x$.}
\end{overlayarea}
\end{frame}
%----------------------------------------------------------------------------------------
\begin{frame}[t]
\only<1>{\AnswerYes}\AnswerSpace
We used the substitution rule to conclude
\begin{align*}
\int (3x^2)\sin(x^3+1)\,\dee x &= -\cos(x^3+1)+C
\intertext{We can check that our antiderivative is correct by differentiating.}
\sonslide<2-|handout:0>{\diff{}{x}\left\{-\cos(x^3+1)+C\right\}&\color{spoilercolor}=\sin(x^3+1)(3x^2)}
\end{align*}


\end{frame}
%----------------------------------------------------------------------------------------
%\subsection{Definite Integrals and the Substitution Rule}
%----------------------------------------------------------------------------------------
\begin{frame}[t]
\AnswerYes<-3>
\sStatusBar{1}{4}
We saw:
\[\int 3x^2\sin(x^3+1)\, \dee x=-\cos(x^3+1)+C\]
So, we can evaluate:
\[\int_0^1  3x^2\sin(x^3+1) \,\dee x
 \sonslide<2->{=\left.-\cos(x^3+1)\right|_0^1
 =\cos(1)-\cos(2)}\]

\sonslide<3->{Alternately, we can put in the limits of integration as we substitute. The bounds are originally given as values of $x$; we simply change them to values of $u$.

If $\su{u(x)=x^3+1}$, then $\su{u(0)=1}$ and $\su{u(1)=2}$.}
\sonslide<4->{
\[
\underbrace{\int_{0}^{1}}_{x\text{-values}} 3x^2\sin(x^3+1)\,\dee x =\underbrace{\int_{\su{1}}^{\su{2}}}_{u\text{-values}}\sin(\su{})\,\dee u=-\cos(2)+\cos(1)
\]

}
\end{frame}
%------------------------------------------------------------
%----------------------------------------------------------------------------------------
%\subsection{Notation for Substituting Bounds}
%----------------------------------------------------------------------------------------
%----------------------------------------------------------------------------------------
\begin{frame}[t]{Notation: limits of integration}
\StatusBar{1}{11}

\[\int_{\pi/4}^{\pi/2} \frac{\cos x}{\sin^3 x}\ \dee x\]\pause

Let $\su{u=\sin x}$, $\sdu{\ \dee u=\cos x \ \dee x}$. Note the limits (or bounds) of integration $\pi/4$ and $\pi/2$ are values of $x$, not $u$: they follow the differential, unless otherwise specified.
\pause

\begin{tikzpicture}
%unsubbed
\draw (0,0) node {$\ds\int_{\su{\pi/4}}^{\su{\pi/2}} \frac{\sdu{\cos x}}{\su{\sin}^3 \su{x}}\sdu{\ \dee x}$};
\draw[|->] (1.,-.25) -- (1.,-1)-|(-.8,-.75);
\draw (0,-1.5) node{$x=\frac{\pi}{2}\atop x=\frac{\pi}{4}$};

%
\begin{scope}[xshift=4cm]%totally subbed
\onslide<5-7|handout:0>{\draw (0,0) node {$\ds\int_{\su{\pi/4}}^{\su{\pi/2}} \frac{1}{\su{}^3}\sdu{\ \dee u}$};
\onslide<6-7|handout:0>{\draw[|->] (0.8,-.25) -- (0.8,-1)-|(-.6,-.75);
	\draw (0,-1.75) node{${u=\frac{\pi}{2}}\atop{u=\frac{\pi}{4}}$};}
%cross out
\filldraw[M5, fill opacity=0.1] (-1.25,-2.25) rectangle (1.25,.75);}
\onslide<7|handout:0>{\draw[ultra thick, M5] (-1.25,-2.25)--(1.25,.75) (-1.25,.75)--(1.25,-2.25);
\draw[M5] (0,-2.5) node{different};}
%fixed partial sub
\onslide<8-|handout:0>{\draw (0,0) node {$\ds\int_{\su{x=\pi/4}}^{\su{x=\pi/2}} \frac{1}{\su{}^3}\sdu{\ \dee u}$};}
\onslide<9-|handout:0>{\draw (0,-1.75) node{${x=\frac{\pi}{2}}\atop{x=\frac{\pi}{4}}$};}
%cross out
\onslide<10|handout:0>{\filldraw[W4, fill opacity=0.1] (-1.5,-2.25) rectangle (1.5,.75);
\draw[W4!50!black] (0,-2.5) node{not standard, but OK};
}
\onslide<10-11|handout:0>{
\draw[W4!50!black] (-2,0) node{$=$};
\draw[W4!50!black] (2,0) node{$=$};
}
\end{scope}
%

\onslide<4-|handout:0>{\begin{scope}[xshift=8cm]%partly subbed
\draw (0,0) node {$\ds\int_{\su{1/\sqrt2}}^{\su{1}} \frac{1}{\su{}^3}\sdu{\ \dee u}$};
\draw[|->] (0.8,-.25) -- (0.8,-1)-|(-.6,-.75);
\draw (0,-1.75) node{${u=\sin\left(\frac{\pi}{2}\right)=1}\atop{u=\sin\left(\frac{\pi}{4}\right)=\frac{1}{\sqrt2}}$};
\end{scope}}
\end{tikzpicture}
\end{frame}
%----------------------------------------------------------------------------------------

%----------------------------------------------------------------------------------------




%----------------------------------------------------------------------------------------
\begin{frame}<beamer>[t]
\only<1>{\AnswerYes}\AnswerSpace
\[\int_{\pi/4}^{\pi/2} \frac{\cos x}{\sin^3 x}\ \dee x\]

Let $\su{u=\sin x}$, $\sdu{ \dee u=\cos x \ \dee x}$. 

\color{C1}
\begin{align*}
\color{black}\int_{\su{\pi/4}}^{\su{\pi/2}} \frac{\sdu{\cos x}}{\su{\sin}^3 \su{x}}\sdu{\ \dee x}
&\color{black}=\ds\int_{\su{1/\sqrt2}}^{\su{1}} \frac{1}{\su{}^3}\sdu{\ \dee u}
\sonslide<2>{\\&=\ds\int_{\su{1/\sqrt2}}^{\su{1}} \su{}^{-3} \sdu{\ \dee u}\\
&=\left[\frac{1}{-2\su{}^2}\right]_{\su{1/\sqrt2}}^{\su{1}}\\&=-\frac12-(-1)=\frac12}
\end{align*}

\end{frame}
%----------------------------------------------------------------------------------------

%----------------------------------------------------------------------------------------
\begin{frame}[t]{True or False?}
\only<1>{\AnswerYes}

\begin{enumerate}
\item[1.] Using $u= x^2$, \[\ds\int e^{x^2}\ \dee x=\int e^u\ \dee u\]
\vfill
\sonslide<2->{False: missing $u'(x)$. \\$ \dee u=(2x\ \dee x) \neq \ \dee x$}
\item[2.] Using $u=x^2+1$, \[\ds\int_0^1 x\sin(x^2+1)\ \dee x=\int_0^1 \frac12\sin u \ \ \dee u\]
\sonslide<2->{False: limits of integration didn't translate. \\When $x=0$, $u=0^2+1=1$, and when $x=1$, $u=1^2+1=2$.}
\end{enumerate}
\end{frame}
%----------------------------------------------------------------------------------------
%----------------------------------------------------------------------------------------
\begin{frame}[t]
\only<1>{\AnswerYes} \AnswerSpace
Evaluate $\ds\int_0^1 x^7\left(x^4+1\right)^5 \, \dee x$.
\sonslide<2->{
\begin{align*}
u&=x^4+1,\ \dee u = 4 x^3\, \dee x \\
u(0)&=1,\ u(1)=2 \\
x^4&=u-1,\ x^3\ \dee x= \tfrac14\, \dee u \\
\int_0^1 x^7\left(x^4+1\right)^5\dee x &=\int_0^1 (x^4)\cdot(x^4+1)^5\cdot x^3\, \dee x\\
&=\int_1^2(u-1)\cdot u^5 \cdot \tfrac14\, \dee u\\
&=\tfrac14\int_1^2(u^6-u^5)\,\dee u\\
&=\tfrac14\left[\tfrac17u^7-\tfrac16u^6\right]_1^2\\
&=\tfrac14\left[\tfrac{2^7}{7}-\tfrac{2^6}{6}-\tfrac17+\tfrac16\right]
\end{align*}
}
\end{frame}
%------------------------------------------------------------
%----------------------------------------------------------------------------------------

\section{More Substitution Rule Examples}
\begin{frame}
Time permitting, more examples using the substitution rule
\end{frame}
%----------------------------------------------------------------------------------------
%----------------------------------------------------------------------------------------
\begin{frame}[t]
\[\text{Evaluate }\int \sin x \cos x\, \dee x.\]

\sonslide<2->{
Let \su{$u=\sin x$}, \sdu{$\ \dee u=\cos x \ \dee x$}:
\[\int\su{ \sin x} \,\sdu{ \cos x \ \dee x}=\int\su{}\sdu{\ \dee u}=\frac{1}{2}\su{}^2+C=\frac12\su{\sin}^2\su{x}+C\]
\vfill
Or, let \su{$u=\cos x$}, \sdu{$\dee u=-\sin x \ \dee x$}:
\[\int\su{ \cos x} \,\sdu{ \,\sin x \ \dee x}=\sdu{-}\int\su{}\sdu{\ \dee u}=-\frac{1}{2}\su{}^2+C=-\frac12\su{\cos}^2\su{x}+C\]
\vfil
Recall $\sin^2 x + \cos^2 x = 1$ for all $x$, so $\frac12\sin^2 x = -\frac12\cos^2x+\frac12$. The two answers look different, but they only differ by a constant, which can be absorbed in the arbitrary constant $C$. If we rename the second $C$ to $C'$ so that the second answer is $-\frac12{\cos}^2x+C'$, then $C'=C+\frac{1}{2}$.
}

\QuestionBar<1>{1}{7}
\AnswerBar<2>{1}{7}
\end{frame}
%------------------------------
%------------------------------
\checkframe{\label{note1.4a}
\AnswerBar{1}{7}

We can check that $\ds\int \sin x \cos x \ \dee x = \onslide<beamer>{\frac{1}{2}\sin^2 x +C}$ by differentiating.

{\sonslide<2->{\begin{align*}
\diff{}{x}\left\{\frac12\sin^2x+C\right\}&=\frac{2}{2}\sin x \cdot \cos x = \sin x \cos x
\end{align*}
Our answer works.}}\vfill

We can check that $\ds\int \sin x \cos x \ \dee x = \onslide<beamer>{-\frac{1}{2}\cos^2 x +C}$ by differentiating.
\sonslide<2->{\begin{align*}
\diff{}{x}\left\{-\frac12\cos^2x+C\right\}&=-\frac{2}{2}\cos x \cdot (-\sin x) = \sin x \cos x
\end{align*}
This answer works too.}
}

%----------------------------------------------------------------------------------------
\begin{frame}[t]
\[\text{Evaluate }\int \frac{\log x}{3x} \ \dee x.\]

\sonslide<2->{
Let \su{$u=\log x$}, \sdu{$\ \dee u=\frac1x \ \dee x$}:
\begin{align*}
\int \frac{\su{\log x}}{3}\cdot\sdu{ \frac1x \ \dee x}&=\frac13\int {\su{}}\ \sdu{\ \dee u}
\\&=\frac16{\su{u}^2}+C
\\&=\frac16{\su{\log}^2 \su{x}}+C\end{align*}
}


\AnswerYes<1>\QuestionBar<1>{2}{7}
\AnswerBar<1>{2}{7}
\end{frame}
%----------------------------------------------------------------------------------------
%------------------------------
\CheckFrame{
\AnswerBar{2}{7}

We can check that $\ds\int \frac{\log x}{3x} \ \dee x =\onslide<beamer>{ \frac16\log^2x +C$} by differentiating.}{
\begin{align*}
\diff{}{x}\left\{\frac16\log^2x+C\right\}&=\frac26\log x\cdot\frac1x=\frac{\log x}{3x}
\end{align*}
Our answer works.
}


%----------------------------------------------------------------------------------------
%----------------------------------------------------------------------------------------

%------------------------------
\begin{frame}[t]
\AnswerYes<1>\QuestionBar<1>{3}{7}\QuestionBar<1>{4}{7}\NoSpace<1>
\AnswerBar<2>{3}{7}
\AnswerBar<3>{4}{7}

\[\text{Evaluate }\int\frac{e^x}{e^x+15} \ \dee x.\]\vfill

\sonly<2>{
Let \su{$u=e^x+15$, $\sdu{\ \dee u=e^x\ \dee x}$}
\[\int\frac{\sdu{e^x}}{\su{e^x+15}} \ \sdu{\dee x}=\int\frac{1}{\su{}} \ \sdu{}=\log|\su{}|+C=\log|\su{e^x+15}|+C\]
In this case, since $e^x+15>0$, the absolute values on $|e^x+15|$ are optional.
\vfill}


\[\text{Evaluate }\int x^4(x^5+1)^8 \ \dee x.\]\vspace{1em}
\sonly<3->{
Let \su{$u=x^5+1$}, $\sdu{\ \dee u=5x^4\ \ \dee x}$. Then, \sdu{$x^4\ \dee x=\frac{1}{5}\ \dee u$}.

\begin{align*}
\int \sdu{x^4}(\su{x^5+1})^8 \sdu{\ \dee x}&=\int \sdu{\frac15}(\su{})^8 \sdu{\ \dee u}\\&=\frac{1}{5}\cdot\frac19\su{}^9+C=\frac{1}{45}\su{(x^5+1)}^9+C\end{align*}
}

\end{frame}
%------------------------------
\checkframe{
\AnswerBar{3}{7}\AnswerBar{4}{7}

We can check that $\ds\int \frac{e^x}{e^x+15} \ \dee x =\onslide<beamer>{ \log|e^x+15| +C}$ by differentiating.
{\sonslide<2->{\begin{align*}
\diff{}{x}\left\{\log|e^x+15| +C\right\}&=\frac{1}{e^x+15}\cdot e^x=\frac{e^x}{e^x+15}
\end{align*}
Our answer works.}}
\vfill
We can check that $\ds\int x^4(x^5+1)^8 \ \dee x =\onslide<beamer>{ \frac{1}{45}(x^5+1)^9 +C}$ by differentiating.
\sonslide<2->{\begin{align*}
\diff{}{x}\left\{\frac{1}{45}(x^5+1)^9 +C\right\}&=\frac{9}{45}(x^5+1)^8\cdot5x^4=(x^5+1)^8x^4
\end{align*}
Our answer works.
}
}


%----------------------------------------------------------------------------------------
\begin{frame}[t]
\AnswerYes<1>\QuestionBar<1>{5}{7}
\AnswerBar<2>{5}{7}
Evaluate $\ds\int_4^8 \frac{s}{s-3}\,\dee s$. Be careful to use correct notation.
\vfill

\sonslide<2->{Let $u=s-3$, $\sdu{\dee u=\dee s}$.\\
Then $s=u+3,\ u(4)=1$ and $u(8)=5$.
\begin{align*}
\int_{4}^{8} \frac{s}{s-3}\,\sdu{\dee s}&=
\int_{1}^{5}\frac{u+3}{u}\,\sdu{}\\
&=
\int_{1}^{5}\left(1+\frac3{u} \right)\,\sdu{}\\
&=\left[u+3\log|u| \right]_{1}^{5}
\\&=\left[5+3\log5 \right]-\left[1+3\log1 \right]\\
&=4+3\log5
\end{align*}\vfill}
\end{frame}
%----------------------------------------------------------------------------------------
%----------------------------------------------------------------------------------------
\begin{frame}[t]
\AnswerYes<1>\QuestionBar<1>{6}{7}
\AnswerBar<2>{6}{7}


\[\text{Evaluate }\int x^9(x^5+1)^8 \ \dee x.\]

\sonslide<2>{
Let $u=x^5+1$, \sdu{$\dee u=5x^4\ \dee x$}.\\
Then \sdu{$ x^4\ \dee x=\frac{1}{5}\,\dee u$}, $x^5=u-1$.
\begin{align*}
\int x^9&(x^5+1)^8 \ \dee x=\int\sdu{(x^4)}\cdot(x^5)\cdot (x^5+1)^8\sdu{\ \dee x}\\
&=\int \sdu{ \left(\frac15\right)}\cdot(u-1)\cdot u^8\ \sdu{}=\frac15\int(u^9-u^8)\ \dee u\\
&=\frac{1}{5}\left[ \frac{1}{10}u^{10}-\frac19u^9\right]+C\\&=\frac{1}{5}\left[ \frac{(x^5+1)^{10}}{10}-\frac{(x^5+1)^9}{9}\right]+C
\end{align*}
}


\end{frame}
%%----------------------------------------------------------------------------------------

%----------------------------------------------------------------------------------------
\CheckFrame{\AnswerBar{6}{7}
We can check that $\ds\int x^9(x^5+1)^8 \ \dee x=\onslide<beamer>{\frac{1}{5}\left[ \frac{(x^5+1)^{10}}{10}-\frac{(x^5+1)^9}{9}\right]+C}$\\ by differentiating.}{
\begin{align*}
\diff{}{x}&\left\{ \frac{1}{5}\left[ \frac{(x^5+1)^{10}}{10}-\frac{(x^5+1)^9}{9}\right]+C\right\}
\\&=\frac{1}{5}\left[ (x^5+1)^9\cdot5x^4-(x^5+1)^8\cdot5x^4\right]
\\&= x^4(x^5+1)^9-x^4(x^5+1)^8 \\
&=x^4(x^5+1)^8\big[(x^5+1)-1\big]\\
&=x^4(x^5+1)^8[x^5]\\
&=x^9(x^5+1)^8
\end{align*}
Our answer works.
}
%----------------------------------------------------------------------------------------
\begin{frame}[t]{Particularly Tricky Substitution}
\AnswerYes<1>\QuestionBar<1>{7}{7}
\AnswerBar<2>{7}{7}


\[\text{Evaluate }\int\frac{1}{e^x+e^{-x}}\ \dee x.\]
\sonslide<2>{

Let $\su{u=e^{x}}$, $\ \sdu{\dee u = e^x\,\dee x}$. Then $\sdu{\dee x=\frac{\dee u}{e^x}=\frac{\dee u}{u}}$.
\begin{align*}
\int\frac{1}{\su{e^x}+\su{e^{-x}}}\ \sdu{\dee x}
&=\int\frac{1}{\su{}+\frac{1}{\su{}}}\ \sdu{\frac{\dee u}{u}}
\\&=\int\frac{1}{\su{}^2+1}\ \sdu{}
\\&=\arctan(\su{})+C\\
&=\arctan(\su{e^x})+C
\end{align*}
}


\end{frame}
%----------------------------------------------------------------------------------------
\CheckFrame{\AnswerBar{7}{7}

We can check that $\ds\int\frac{1}{e^x+e^{-x}}\ \dee x=\onslide<beamer>{\arctan(e^x)+C}$ by differentiating.}{
\begin{align*}
\diff{}{x}\left\{\arctan(e^x)+C \right\}&=\frac{1}{(e^x)^2+1}\cdot e^x
\\&=\frac{e^x}{(e^x)^2+1}\\&=\frac{e^x}{(e^x)^2+1}\cdot\frac{e^{-x}}{e^{-x}}\\&=\frac{1}{e^x+e^{-x}}
\end{align*}
Our answer works.
}

%----------------------------------------------------------------------------------------
