% Copyright 2021 Joel Feldman, Andrew Rechnitzer and Elyse Yeager, except where noted.
% This work is licensed under a Creative Commons Attribution-NonCommercial-ShareAlike 4.0 International License.
% https://creativecommons.org/licenses/by-nc-sa/4.0/


 \begin{frame}{Table of Contents }
\mapofcontentsC{\ca}
 \end{frame}
%----------------------------------------------------------------------------------------

%----------------------------------------------------------------------------------------
\section{Introduction to Sequences and Series}
%----------------------------------------------------------------------------------------
%----------------------------------------------------------------------------------------
\begin{frame}
\StatusBar{1}{4}
We can imagine the list of numbers below carrying on forever:
\begin{align*}
\onslide<2>{a_1&=}0.1\\
\onslide<3-|handout:0>{+}\onslide<2>{a_2&=}0.01\\
\onslide<3-|handout:0>{+}\onslide<2>{a_3&=}0.001\\
\onslide<3-|handout:0>{+}\onslide<2>{a_4&=}0.0001\\
\onslide<3-|handout:0>{+}\onslide<2>{a_5&=}0.00001\\
&\qquad\vdots\\
\onslide<3-|handout:0>{&\rule{2cm}{1 pt}}\\
\onslide<4-|handout:0>{&\hphantom{=} 0.11111\cdots}
\end{align*}
A \alert{sequence}  is a list of infinitely many numbers with a specified order. \pause It is denoted
$\{a_1,\ a_2,\ \cdots , a_n,\ \cdots\}$ or $\{a_n\}_{n=1}^\infty$, etc. \pause

Imagine \textit{adding up} this sequence of numbers.\\ \pause A \alert{series} is a sum $a_1+a_2+\cdots+a_n+\cdots$ of infinitely many terms.
\end{frame}
%----------------------------------------------------------------------------------------
\begin{frame}
To handle sequences and series, we should define them more carefully. A good definition should allow us to answer some basic questions, such as:
\begin{itemize}
\item What does it mean to add up infinitely many things?
\item Should infinitely many things add up to an infinitely large number?
\item Does the order in which the numbers are added matter?
\item Can we add up infinitely many functions, instead of just infinitely many numbers?
\end{itemize}
\end{frame}
%----------------------------------------------------------------------------------------

%----------------------------------------------------------------------------------------

%----------------------------------------------------------------------------------------
\section{3.1 Sequences}
%-------------------------------------------------------------
\begin{frame}
\begin{block}{Sequence}
A \alert{sequence} is a list of infinitely many numbers with a specified order. 
\end{block}
Some examples of sequences:
\begin{itemize}
\item $\{1,2,3,4,5,6,7,8,\cdots\}$\qquad (natural numbers)\vfill
\item $\{3,1,4,1,5,9,2,6,\cdots\}$ \qquad (digits of $\pi$)\vfill
\item $\{1,-1,1,-1,1,\cdots\}$ \qquad (powers of $-1:$ $(-1)^0$, $(-1)^1$, $(-1)^2$, etc.)
\end{itemize}
\end{frame}

%%----------------------------------------------------------------------------------------
%----------------------------------------------------------------------------------------
\begin{frame}
\begin{block}{Sequence}
A \alert{sequence} is a list of infinitely many numbers with a specified order. It is denoted
$\{a_1,\ a_2,\ a_3,\ \cdots, a_n\ \cdots\}$ or $\{a_n\}$ or $\{a_n\}_{n=1}^\infty$, etc.
\end{block}

\[\{\alert<4|handout:0>{a_n}\}_{\alert<2|handout:0>{n=1}}^{\alert<3|handout:0>{\infty}} = \left\{ \alert<4|handout:0>{\frac1n}\right\}_{\alert<2|handout:0>{n=1}}^{\alert<3>{\infty}}\]
\pause\vfill
\begin{itemize}[<+->]
\item \alert<2|handout:0>{$n=1$: this is the index of the first term of our sequence. Sometimes it's 0, sometimes something else, for example a year.}
\item \alert<3|handout:0>{$\infty$: there is no end to our sequence.}
\item \alert<4|handout:0>{$\frac1n$: this tells us the value of $a_n$.}
\item Often we omit the limits and even the brackets, writing \alert<beamer>{$a_n=\frac1n$}.
\end{itemize}
\end{frame}


%----------------------------------------------------------------------------------------
\begin{frame}{Sequence Notation}
For convenience, we write $a_1$ for the first term of a sequence, $a_2$ for the second term, etc.
\vfill
In the sequence $1,\frac{1}{2},\frac{1}{3},\frac14,\cdots$, \\
$a_3$ is another name for  \sonslide<2->{$\frac{1}{3}$.}
\vfill
Sometimes we can find a rule for a sequence.\\ 
In the above sequence, $a_n=$\sonslide<3->{$ \frac{1}{n}$ (whenever $n$ is a whole number).}
\vfill
We can write $\{a_n\}_{n=1}^\infty =\sonslide<4->{{ \left\{ \frac1n\right\}}_{n=1}^\infty.}$
\end{frame}

%----------------------------------------------------------------------------------------
\begin{frame}
Our primary concern with sequences will be the behaviour of $a_n$ as $n$
tends to infinity and, in particular, whether or not $a_n$ ``settles down''
to some value as $n$ tends to infinity.
\begin{block}{Convergence}
A sequence $\big\{a_n\big\}_{n=1}^\infty$ is said to \alert{converge} to the limit
$A$ if $a_n$ approaches $A$ as $n$ tends to infinity. If so, we write
\begin{equation*}
\lim_{n\rightarrow\infty} a_n=A\qquad\hbox{or}\qquad
a_n\rightarrow A\text{ as }n\rightarrow\infty
\end{equation*}
A sequence is said to converge if it converges to some limit. Otherwise
it is said to diverge.
\end{block}
\unote{Definition~\eref{text}{def:SRsequenceLimit}}
\end{frame}
%----------------------------------------------------------------------------------------
%-------------------------------------------------------------
\begin{frame}
\AnswerYes<1>\NoSpace<1>
\begin{block}{Convergence}
A sequence $\big\{a_n\big\}_{n=1}^\infty$ is said to \alert{converge} to the limit
$A$ if $a_n$ approaches $A$ as $n$ tends to infinity. If so, we write
\begin{equation*}
\lim_{n\rightarrow\infty} a_n=A\qquad\hbox{or}\qquad
a_n\rightarrow A\text{ as }n\rightarrow\infty
\end{equation*}
A sequence is said to converge if it converges to some limit. Otherwise
it is said to diverge.
\end{block}

\begin{itemize}
\item $\{1,2,3,4,5,6,7,8,\cdots\}$ (natural numbers)\\ This sequence  \sonslide<2->{\alert{diverges}, growing without bound, not approaching a real number.}\vfill
\item $\{3,1,4,1,5,9,2,6,\cdots\}$  (digits of $\pi$)\\ This sequence \sonslide<2->{\alert{diverges}, since it bounces around, not approaching a real number.}\vfill
\item $\{1,-1,1,-1,1,\cdots\}$  (powers of $-1:$ $(-1)^0$, $(-1)^1$, $(-1)^2$, etc.)\\ This sequence  \sonslide<2->{\alert{diverges}, since it bounces around, not approaching a real number.}
\end{itemize}
\end{frame}

%%----------------------------------------------------------------------------------------
%----------------------------------------------------------------------------------------
\begin{frame}[t]
\AnswerYes<1>
Does the sequence $a_n=\dfrac{n}{2n+1}$ converge or diverge?
\vspace{1em}

\sonslide<2>{
To study the behaviour of $\dfrac{n}{2n+1}$ as $n\rightarrow\infty$, it is a good
idea to write it as:
\[\frac{n}{2n+1}=\frac{1}{2+\frac{1}{n}}\]
As $n\rightarrow\infty$, the $\frac{1}{n}$ in the denominator tends to
zero, so that the denominator $2+\frac{1}{n}$ tends to $2$ and
$\frac{1}{2+\frac{1}{n}}$ tends to $\frac{1}{2}$. So
\begin{align*}
\lim_{n\rightarrow\infty}\frac{n}{2n+1}
=\lim_{n\rightarrow\infty}\frac{1}{2+\frac{1}{n}}
=\frac{1}{2+0}=\frac12
\end{align*}
}
\unote{Example~\eref{text}{eg:SRsequenceLimB}}
\end{frame}
%----------------------------------------------------------------------------------------
%----------------------------------------------------------------------------------------
\begin{frame}
\sStatusBar{1}{4}
\nsStatusBar{1}{3}
Consider the sequence $\ds\textcolor{W1}{a_n=\frac{1}{3^n+1}}$.
\hspace{1cm}
$\lim\limits_{n\to\infty}\textcolor{W1}{a_n}=\sonslide<4->{0}$
\begin{center}
\begin{tikzpicture}[yscale=0.6]
\myaxis{n}{.25}{7}{}{.25}{4}
\onslide<3-|handout:0>{
{\color{C2}
\draw[thick] plot[domain=1:6](\x,{15/(3^\x+1)})node[above right]{$f(x)=\frac{1}{3^x+1}$};
}}


\foreach \x in {1,2,3,4,5}{\xcoord{\x}{\x}
\onslide<2-|handout:0>{
\POWER{3}{\x}{\p}
\ADD{\p}{1}{\d}
\DIVIDE{15}{\d}{\y}
\draw[W1] (\x,\y) node[vertex, label=above right:{$a_{\x}$}]{};
}%slide3
}%foreach
\onslide<2->{\ycoord{15/4}{\frac14}
\ycoord{15/10}{\frac1{10}}
\ycoord{15/28}{\frac1{28}}
}

\end{tikzpicture}
\end{center}
\onslide<3->{\begin{block}{Theorem~\eref{text}{thm:SRxlimtoanlim}}
If \vspace{-1em}
\begin{equation*}
\lim_{x\rightarrow\infty} f(x) = L
\end{equation*}
and if $a_n=f(n)$ for all positive integers $n$, then
\begin{equation*}
\lim_{n\rightarrow\infty} a_n = L
\end{equation*}
\end{block}}
\end{frame}
%----------------------------------------------------------------------------------------
%----------------------------------------------------------------------------------------
\begin{frame}[t]{Cautionary Tale}
\sStatusBar{1}{8}
\nsStatusBar{1}{5}
Consider the sequence $\textcolor{W1}{b_n=\sin(\pi n)=\sonslide<6->{\{0,0,0,0,0,\ldots\}}}$\\[1em]
\pause


\onslide<5->{$\lim\limits_{n\to\infty}\textcolor{W1}{b_n}=\sonslide<7->{0}$}\hspace{2cm}
 \onslide<5->{$\lim\limits_{x\to\infty}\textcolor{C2}{f(x)}$ \sonslide<8->{ DNE}}
\begin{center}
\begin{tikzpicture}
\myaxis{n}{.25}{7}{}{1.25}{1.25}

\onslide<4->{
{\color{C2}
\draw[thick] plot[domain=0:6.25,smooth](\x,{sin(\x*3.141 r) })node[right]{$f(x)=\sin(\pi x)$};
}}


\foreach \x in {1,2,3,4,5}{\xcoord{\x}{\x}
\onslide<3-|handout:0>{
\draw[W1] (\x,0) node[vertex, label=above	:{$b_{\x}$}]{};
}%slide3
}%foreach


\end{tikzpicture}
\end{center}
\end{frame}

%----------------------------------------------------------------------------------------
\begin{frame}
\begin{block}{Theorem }
\alert{If $\lim\limits_{x\rightarrow\infty} f(x) = L$}
and if $a_n=f(n)$ for all natural $n$, then
$\lim\limits_{n\rightarrow\infty} a_n = L.
$
\end{block}\vfill

\begin{tikzpicture}[scale=0.6]
\myaxis{n}{0}{7}{}{0}{4}
{\color{C2}
\draw[thick] plot[domain=1:6](\x,{15/(3^\x+1)});
}
\foreach \x in {1,2,3,4,5}{\xcoord{\x}{\x}
\POWER{3}{\x}{\p}
\ADD{\p}{1}{\d}
\DIVIDE{15}{\d}{\y}
\draw[W1] (\x,\y) node[vertex, label=above right:{$a_{\x}$}]{};
}%foreach
\end{tikzpicture}\hfill
\begin{tikzpicture}[scale=0.75]
\myaxis{n}{0}{7}{}{1.25}{1.25}
{\color{C2}
\draw[thick] plot[domain=0:6.25,smooth](\x,{sin(\x*3.141 r) });
}
\foreach \x in {1,2,3,4,5}{\xcoord{\x}{\x}
\draw[W1] (\x,0) node[vertex, label=above	:{$b_{\x}$}]{};
}%foreach
\end{tikzpicture}
\end{frame}
%----------------------------------------------------------------------------------------

%----------------------------------------------------------------------------------------
\begin{frame}
\label{note3.1a}
\begin{block}{Arithmetic of Limits}
\unote{Theorem~\eref{text}{thm:SRlimarith}}
  Let $A$, $B$ and $C$ be real numbers and let the two sequences
$\big\{a_n\big\}_{n=1}^\infty$ and  $\big\{b_n\big\}_{n=1}^\infty$
converge to $A$ and $B$ respectively. That is, assume that
 \begin{align*}
  \lim_{n \to \infty} a_n&=A & \lim_{n \to \infty} b_n &=B
\end{align*}
  Then the following limits hold.
\begin{enumerate}[(a)]
 \item $\ds \lim_{n \to \infty} \big[a_n+b_n\big] = A+B$
 \item $\ds \lim_{n \to \infty} \big[a_n-b_n\big] = A-B$
\item $\ds \lim_{n \to \infty} C a_n = C A$.
\item $\ds \lim_{n \to \infty} a_n\,b_n = A\,B$
\item If $B \neq 0$, then $\ds \lim_{n \to \infty}\frac{a_n}{b_n} = \frac{A}{B}$
\end{enumerate}
\end{block}

\end{frame}

%----------------------------------------------------------------------------------------
%----------------------------------------------------------------------------------------
\begin{frame}[t]
\label{note3.1b}
\AnswerYes<1>\NoSpace<1>
 Evaluate the following limits:\vfill
\begin{itemize}
\item $\lim\limits_{n\to\infty}{e^{-n}}=\sonslide<2>{0}$\vfill
\item $\lim\limits_{n\to\infty}\frac{1+n}{n}=\sonslide<2>{1}$\vfill
\item $\lim\limits_{n\to\infty}\frac{1}{n^2}=\sonslide<2>{0}$\vfill
\item $\lim\limits_{n\to\infty}{2n^2}=\sonslide<2>{\infty}$\vfill
\item $\lim\limits_{n\to\infty}\left(\frac{1}{n^2}\right)\left(2n^2\right)=\sonslide<2>{2}$
\end{itemize}\vfill
\sonslide<2>{(As you might guess, the expression ``$\lim\limits_{n\rightarrow\infty} a_n = \infty$" means that $a_n$
grows without bound as $n\rightarrow\infty$.)}
\end{frame}
%----------------------------------------------------------------------------------------
\begin{frame}[t]
\AnswerYes<1>
\unote{Theorem~\eref{text}{thm:SRcontfn}}
\begin{block}{Continuous functions of limits}
If $\lim\limits_{n\to \infty}a_n=L$ and if the function $g(x)$ is continuous at $L$, then
\[\lim_{n \to \infty}g(a_n)=g(L)\]
\end{block}

Evaluate $\lim\limits_{n\to\infty}\left[\sin\left(\frac{\pi n}{2n+1}\right)\right]$

\sonslide<2>{
	\begin{align*}
	\lim_{n \to \infty} \left[\frac{\pi n}{2n+1}\right]&=\lim_{n \to \infty} \left[\frac{\pi}{2+\frac{1}{n}}\right]=\frac{\pi}{2}\\
	\lim\limits_{n\to\infty}\left[\sin\left(\frac{\pi n}{2n+1}\right)\right]&=\sin\left(\frac{\pi}{2}\right)=1
	\end{align*}
	}
\end{frame}
%----------------------------------------------------------------------------------------
%----------------------------------------------------------------------------------------
\begin{frame}[t]
\begin{block}{Squeeze Theorem  }
\unote{Theorem~\eref{text}{thm:SRsqueeze}}
If $\textcolor{C2}{a_n}\le c_n\le \textcolor{M4}{b_n}$ for all sufficiently large natural numbers $n$, and if
\begin{equation*}
\lim_{n\rightarrow\infty}\textcolor{C2}{a_n}=\lim_{n\rightarrow\infty}\textcolor{M4}{b_n}=L
\end{equation*}
then\begin{equation*}
\lim_{n\rightarrow\infty}c_n=L
\end{equation*}
\end{block}
\begin{center}
\begin{tikzpicture}
\myaxis{n}{.25}{10}{}{.25}{3.2}
\draw[thick, dashed] (0,2.9)--(10,2.9);
\foreach \x in {1,...,9}{\xcoord{\x}{\x}
\DIVIDE{5}{\x}{\y}
\draw[C2,opacity=0.75] (\x,{3-(61*\x-90)/(\x*(20*\x-30))}) node[vertex]{};
\draw[opacity=0.75] (\x,{3-5/(2.5*\x)}) node[vertex]{};
\draw[M4,opacity=0.75] (\x,{3-1/(5*\x-7.5)}) node[vertex]{};

}
\end{tikzpicture}
\end{center}
\end{frame}
%----------------------------------------------------------------------------------------
%----------------------------------------------------------------------------------------
\begin{frame}[t]
\only<1>{\AnswerYes}\AnswerSpace
Evaluate 
\[\lim_{n\to\infty}\left(\frac{2n+\cos n}{n+1} \right)\]

\sonslide<2->{
Use squeeze theorem:

\begin{align*}
-1&\leq \cos n \leq 1 \\
2n-1 & \leq 2n+\cos n \leq 2n+1\\
\frac{2n-1}{n+1} & \leq \frac{2n+\cos n}{n+1} \leq \frac{2n+1}{n+1}\\
\lim_{n\to\infty}\frac{2n-1}{n+1} & = \lim_{n\to\infty} \frac{2n+1}{n+1}=2\\
2 &=\lim_{n\to\infty}  \frac{2n+\cos n}{n+1} 
\end{align*}}

\end{frame}
%----------------------------------------------------------------------------------------
\begin{frame}[t]
\only<1>{\AnswerYes}\AnswerSpace
Let $a_n=(-n)^{-n}$. Evaluate $\lim\limits_{n \to \infty}a_n$.

\sonslide<2->{
First, we note $a_n=(-1)^{-n}\cdot (n^{-n})=\frac{(-1)^n}{n^n}$
because $(-1)^{-n} ={\big((-1)^{-1}\big)}^n=(-1)^n$.
%(Note that multiplying and dividing by 1 do the same thing; and multiplying and dividing by %negative 1 do the same thing.)

This sequence alternates between positive and negative terms. We can show that the positive terms tend to zero and the negative terms tend to zero.
   So, we can apply the squeeze theorem.
	\begin{align*}
	\text{Set }b_n&=\frac{-1}{n^n} \text{ and }c_n=\frac{1}{n^n}\\
	\text{Then, } b_n &<a_n<c_n \text{ for all natural $n$}\\
	\lim_{n \to \infty}b_n&=\lim_{n \to \infty}c_n=0\\
	\text{So, } \lim_{n \to \infty}a_n&=0
	\end{align*}
	}
\end{frame}
%----------------------------------------------------------------------------------------
